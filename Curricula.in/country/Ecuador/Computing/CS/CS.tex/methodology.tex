\section{Metodología de estudio}
El proceso educativo teórico-práctico, se lo desarrolla dentro de un marco de análisis, 
reflexión y propuestas conceptuales y prácticas, con la participación dinámica de los 
actores académicos. Se apoyará en actividades de investigación formativa y operativa, 
para lo cual se realizarán acercamientos a los sectores comunitarios organizados, 
empresas e instituciones de servicios  y productivas. 

Se priorizarán las actividades de taller y laboratorio, con el propósito 
establecer niveles de crecimiento intelectual y práctico. Se desarrollarán 
aprehensiones de los avances tecnológicos que estén en el sistema de producción 
empresarial, con  especial atención en los escenarios de   gestión- administración 
y comunicación  

Los/as estudiantes durante su proceso de formación recibirán conocimientos 
técnicos y operativos, a través de eventos  electivos, con el propósito de 
fortalecer su capacidad práctica en aspectos de habilidad y sistema motriz, 
considerando el nivel de crecimiento intelectual teórico-práctico. 

Las actividades académicas teóricas-prácticas y de investigación, tendrán un 
ciclo de estudios de 18 semanas por nivel. En cada semana se desarrollarán de 
25 a 30 horas de interacción docente-estudiantes y en  cada nivel se deberán 
cumplir de 25 a 30 créditos de acuerdo a los conocimientos requeridos para 
una formación de calidad, con sentido holístico y sistémica. 

Para otorgar el título universitario de tercer nivel, cada estudiante debe 
aprobar la totalidad del pensun con su equivalente a 257 créditos  en 
Mención Redes: 255 créditos en Informática Bancaria: 254  Software y 
haber realizado una trabajo de investigación como requisito a su titulación. 

\section{Evaluación de los aprendizajes}
Las actividades académicas serán evaluadas de manera permanente, sistemática y 
técnicas, considerando cuatro parámetros: participación en clase, tareas individuales 
y grupales, trabajos de investigación y pruebas de conocimientos.  Los tres primeros 
parámetros tendrán una calificación del 20\% cada uno y las pruebas de conocimientos 
el 40\%.  

Un estudiante aprueba un evento académico cuando la sumatoria logre el 70\%  (siete puntos) 
sobre el 100\% (diez puntos) y reprueba el evento cuando alcance el 40\%  (cuatro puntos). 
Entre el 50\% y el 60\%  (cinco o seis) el o la estudiante tendrá una prueba de 
recuperación y deberá obtener una calificación no menor a siete puntos sobre diez. 
 
\section{Organización académica y administrativa}
La carrera de Ingeniería  en  Ciencias  y Sistemas Informáticos será administrada en 
su parte académica por un coordinador de carrera, apoyado  en su gestión  y proyección 
educativa e institucional  por un director de  la unidad técnica. 

Los espacios, equipos y materiales requeridos  en la parte teórica-práctica y de 
investigación que demanda la  formación profesional, serán fortalecidos por el nivel 
ejecutivo y sus  diferentes instancias académica y administrativa de la universidad.   

Los/as  estudiantes estarán en predisposición para aprobar  todos los eventos académicos 
que contiene la planificación en un semestre, el en caso de no aprobar un evento en el semestre, 
podrán matricularse en el semestre inmediato superior y recibir la resolución de arrastre del 
evento. Cuando no aprobare dos o más eventos en el semestre, reprobará el semestre y 
recibirá una resolución para matricularse en el mismo semestre por los eventos no aprobados.   

\section{Laboratorios y talleres}
La Universidad  dispone de espacios y equipos tecnológicos  suficientes para el procesos 
educativo teórico-práctico,  el  análisis y tratamiento  de cada tema de investigación, 
planeación, modelación, desarrollo, validación, transferencia y comunicación de todo el 
sistema informático y computacional.  

\section{Convenios inter-institucionales}
La carrera de Ingeniería en Ciencias y Sistemas  Informáticos  a través de la Universidad, 
tendrá como el principal aliado el contexto productivo, financiero y de servicio, con ellos
 se realizarán importantes convenios de carácter académicos y de servicios, que permita un 
adecuado sistema de  investigación y estudio del entorno;  al mismo tiempo, reconozcan 
los medios y proyección de las organizaciones e instituciones de servicios, en el campo 
de  la ciencia y la tecnología, dando oportunidad a los estudiantes de   conocer y 
reflexionar sobre los principales temas  del sistema informático y su demanda estructural, 
de comunicación, procesamiento y transferencias  en los  medios productivos.
