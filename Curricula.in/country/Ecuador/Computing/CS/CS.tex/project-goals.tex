\section{Objetivos del proyecto}
\begin{enumerate}
\item Responde a las demandas del desarrollo tecnológico, que  genera la ciencia de la computación.

\item Proyecto técnico académico diseñado para ejecutarlo  en ciencias de computación, 
con procesos pedagógicos adecuados al sistema educativo  de pregrado y a los niveles de 
formación  humanística,  básica,  profesional, optativo  y de servicio comunitario.
\end{enumerate}

\subsection{Objetivo académico general}
Profesionales formados con los suficientes  conocimientos  científicos, técnicos, de 
acuerdo a los avances de la  ciencia de la computación, con  niveles de apropiación 
investigativa y capacidad para crear, emprender, aplicar y desarrollar sistemas de computación.

\subsection{Objetivos específicos de formación}
\begin{itemize}
\item Otorgar conocimientos teóricos, prácticos y de investigación en el proceso formativo, 
con énfasis a los avances de la ciencia de  computación.

\item Facilitar conocimientos de  computación desde una perspectiva analítica de autocrítica 
que oriente el desempeño de su práctica profesional y mejore los conceptos  de la 
ciencia en computación.

\item Posesionar a los estudiantes de técnicas y estrategias de investigación 
formativa y aplicada  en la ciencia de la computación, con niveles de estudios 
y exploración contextual.

\item Facilitar capacidad y habilidades  que permitan crear y desarrollar 
medios computacionales.

\item Sensibilizar el desarrollo de la ciencia de computación con los principios 
éticos y morales de la sociedad y del individuo.
\end{itemize}