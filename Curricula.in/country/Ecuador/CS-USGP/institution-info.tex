\newcommand{\fecha}{\today}
\newcommand{\YYYY}{2010\xspace}
\newcommand{\Semester}{2010-I\xspace}
\newcommand{\city}{Portoviejo-Manabí\xspace}
\newcommand{\country}{Ecuador\xspace}
\newcommand{\dictionary}{Español\xspace}
\newcommand{\GraphVersion}{2\xspace}
\newcommand{\CurriculaVersion}{2\xspace}
\newcommand{\OutcomesList}{a,b,c,d,e,f,g,h,i,j,k,l,m,HU,FH,TASDSH}

\newcommand{\University}{Universidad Particular San Gregorio de Portoviejo\xspace}
\newcommand{\InstitutionURL}{http://www.sangregorio.edu.ec\xspace}
\newcommand{\underlogotext}{}
\newcommand{\FacultadName}{???\xspace}
\newcommand{\DepartmentName}{Ciencia de la Computación\xspace}
\newcommand{\SchoolFullName}{Carrera Profesional de Ciencia de la Computación\xspace}
\newcommand{\SchoolFullNameBreak}{\SchoolFullName}
%\newcommand{\SchoolFullName}{Programa Profesional de Ingeniería Informática (Ciencia de la Computación)\xspace}
\newcommand{\SchoolShortName}{Ciencia de la Computación\xspace}
\newcommand{\SchoolAcro}{CPCC\xspace}
\newcommand{\SchoolURL}{http://www.sangregorio.edu.ec}

\newcommand{\AcademicDegreeIssued}{Bachiller en Ciencia de la Computación\xspace}
\newcommand{\TitleIssued}{Ingeniero en Ciencia de la Computación\xspace}
\newcommand{\AcademicDegreeAndTitle}%
{\begin{description}%
\item [Titulo Profesional: ] \TitleIssued%
\end{description}%
}

\newcommand{\doctitle}{Proyecto Técnico Académico de Creación: \SchoolFullName{\Large\footnote{\SchoolURL}}\xspace}
\newcommand{\AbstractIntro}{Este documento representa el Proyecto Técnico Académico de creación de la \SchoolFullName 
de la \University en la ciudad de \city-\country(\textit{\InstitutionURL}).}

\newcommand{\OtherKeyStones}%
{Un pilar que merece especial consideración en el caso de la \University es el 
aspecto de valores humanos debido a que forman parte fundamental de la misión de la universidad.\xspace}

\newcommand{\profile}{%
El perfil profesional de este programa profesional puede ser mejor entendido a partir de
\OnlyMainDoc{la Fig. \ref{fig.cs} (Pág. \pageref{fig.cs})}\OnlyPoster{las figuras del lado derecho}. 
Este profesional tiene como centro de su estudio a la computación. Es decir, tiene a la computación 
como fin y no como medio. De acuerdo a la definición de esta área, este profesional está llamado 
directamente a ser un impulsor del desarrollo de nuevas técnicas computacionales que 
puedan ser útiles a nivel local, nacional e internacional.

Nuestro perfil profesional está orientado a ser generador de puestos de empleo a través de la innovación permanente. 
Nuestra formación profesional tiene 3 pilares fundamentales: 
Formación Humana, un contenido de acuerdo a normas internacionales y una orientación marcada a la innovación.
}

\newcommand{\HTMLFootnote}{Generado por <A HREF='http://socios.spc.org.pe/ecuadros/'>Ernesto Cuadros-Vargas</A> <ecuadros AT spc.org.pe><BR>basado en el modelo de la <A HREF='http://www.spc.org.pe/'>Sociedad Peruana de Computación</A> y en la Computing Curricula de <A HREF='http://www.computer.org/'>IEEE-CS</A>/<A HREF='http://www.acm.org/'>ACM</A>}