\section{Análisis de Demanda}\label{sec:cs-analisis-de-demanda}
Nuestro egresado puede desempeñarse en el mercado laboral peruano sin ningún problema ya que en general la exigencia peruana esta mucho más orientada al uso de herramientas. Sin embargo, es poco común que los propios profesionales de esta carrera se pregunten: ?`que tipo de formación debería tener si yo quisiera crear esas herramientas además de saber usarlas?. Ambos perfiles, usuario y creador, son bastante diferentes pues no sería posible usar algo que todavía no fue creado. En otras palabras, los creadores de tecnología son los que \underline{dan origen a nuevos puestos de trabajo} y abren la posibilidad de que otros puedan usar esa tecnología. Ambos tipos de profesional son igualmente importantes en nuestra sociedad.

Debido a la formación basada en la investigación, nuestro profesional debe siempre ser un innovador en su centro de trabajo. Esta misma formación permite que el egresado piense también en crear su propia empresa de desarrollo de software. Considerando que Perú es un país con un costo de vida mucho menor que Norte América ó Europa, una posibilidad que se muestra interesante es la exportación de software.

Este perfil profesional también posibilita que nuestros egresados se queden en nuestro país pues, producir en Perú y vender en USA, es más rentable que salir al extranjero y comercializarlo allá.

El campo ocupacional de un egresado es amplio y está en continua expansión y cambio. Prácticamente toda empresa u organización hace uso de servicios de computación de algún tipo, y la buena formación básica de nuestros egresados hace que puedan responder a los requerimientos de las mismas exitosamente. Este egresado no sólo podrá dar soluciones a los problemas existentes sino que deberá proponer innovaciones tecnológicas que impulsen la empresa hacia un progreso constante.

A medida que la informatización básica de las empresas del país se está completando, la necesidad de personas capacitadas para resolver los problemas de mayor complejidad aumenta y el plan de estudios que hemos desarrollado tiene como objetivo satisfacer esta demanda considerando en mediano y largo plazo. El campo para las tareas de investigación y desarrollo de problemas complejos en computación es también muy amplio y está creciendo día a día en el país. 

%infraestructura, turnos libres en la tarde
%Proyeccion de docentes, al cabo de cuatro años
%biblioteca compartida con sistemas
%grado academico, menciones