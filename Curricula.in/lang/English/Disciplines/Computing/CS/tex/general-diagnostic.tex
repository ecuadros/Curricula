\section{Diagnóstico General}

\subsection{Análisis de Currículos}
Nuestra \SchoolFullName inicia sus actividades el 2010 con un currículo basado en la propuesta internacional de 
Ciencia de la Computación de IEEE y ACM en su versión 2008 \cite{ComputerScience2008}.

Esta propuesta ha permitido que nuestros alumnos sean formados de acuerdo a las recomendaciones internacionales para esta línea
siendo su formación compatible con lo que se enseña a nivel mundial.

Además, el hecho de estar alineados a una propuesta internacional facilita enormemente la movilidad estudiantil y de docentes,
el acceso a estudios de postgrado en el extranjero, la acreditación internacional. Todos estos son factores muy importantes en el 
desarrollo académico y profesional de una Escuela Profesional tan dinámica como lo es la computación hoy en dia.

A nivel internacional la siguiente propuesta para el área de Ciencia de la Computación fue publicada en el documento \cite{CS2013} donde 
inclusive uno de nuestros profesores participó en la elaboración de dicho documento como miembro del comité internacional que la formuló.

Este último documento \cite{CS2013} y el Curriculo de inicial de 2010 sirvieron como insumos para el presente currículo 
por lo cual estamos seguros que está totalmente actualizado con relación a las tendencias internacionales del área.

\subsection{Propuestas curriculares de la Escuela en otras universidades nacionales y extranjeras}
La referencia más sólida a nivel mundial en cuanto a la propuesta de carreras de 
computación para nivel de pregrado es la que fue propuesta en conjunto por la 
\ac{ACM}, \ac{IEEE-CS} y la \ac{AIS}. Estas tres organizaciones propusieron la 
Computing Curricula en el documento denominado: {\it Joint Task Force for Computing Curricula 2005, 
Computing Curricula 2005. Overview Report}\cite{ComputingCurricula2005}.

La \acl{CC} es un término de origen estadounidense \ac{CS}. Este término es 
conocido también como informática en el ámbito europeo\footnote{El término 
europeo es derivado del vocablo francés {\it Informatique}.}.

Según el diccionario de la Real Academia de la Lengua Española (http://www.rae.es) 
ambos términos también son sinónimos.

A nivel internacional, la computación presenta 5 perfiles claramente definidos: 
\begin{itemize}
\item Ciencia de la Computación (\textit{Computer Science}) \\ \cite{CS2013},
\item Ingeniería de Computación (\textit{Computer Engineering}) \cite{ComputerEngineering2004},
\item Sistemas de Información (\textit{Information Systems}) \cite{InformationSystemsCurricula2010, InformationSystems2002Journal},
\item Ingeniería de Software (\textit{Software Engineering}) \cite{SoftwareEngineering2004},
\item Tecnología de la Información (\textit{Information Technology}) \cite{InformationTechnology2005}
\end{itemize}

La Figura \ref{fig.cs} es tomada de la definición propuesta en la \textit{Computing Curricula} 
\cite{ComputingCurricula2005} en el área de \ac{CC} cuyo última versión es la denominada CS2013 (cs2013.org) \cite{CS2013}.
La \ac{CC} cubre la mayor parte entre el extremo superior y el extremo inferior, porque el 
profesional en \ac{CC} no trata ``solamente con el hardware'' que utiliza un software o de 
``solamente la organización'' que hace uso de la información que la computación le puede proveer. 

\begin{figure}[ht]
   \centering
   \includegraphics[width=13cm]{\OutputFigsDir/\currentarea}
   \caption{Campo acción de la Ciencia de la Computación}
   \label{fig.cs}
\end{figure}

%\begin{quote}
Las Ciencias de la Computación cubren un amplio rango, desde sus fundamentos teóricos y algorítmicos hasta 
los \'ultimos desarrollos en robótica, visión por computadora, sistemas inteligentes, bioinformática, y 
otras áreas emocionantes. Podemos pensar que el trabajo de un científico de la computación pertenece 
a las siguientes tres categorías:

\begin{itemize}
\item \textbf{Diseño e implementación de software}. Los científicos de computación se encargan de 
desafiantes labores de programación. También supervisan otros programadores, haciéndolos concientes 
de nuevas aproximaciones.

\item \textbf{Instrumentación de nuevas formas para usar computadoras}. El progreso en las áreas 
de ciencias de la computación como redes, bases de datos, e interfaces humano-computadora permitieron 
el desarrollo de la www y actualmente se trabaja en el desarrollo de metasistemas Grid. Además, 
los investigadores trabajan ahora en hacer que los robots sean ayudantes prácticos y demuestren 
inteligencia, utilizan las bases de datos para crear nuevos conocimientos, y están utilizando 
computadoras para decifrar los secretos de nuestro ADN.

\item \textbf{Desarrollo de formas efectivas de resolver problemas de computación.} 
Por ejemplo, los científicos de la computación desarrollan las mejores formas posibles 
de almacenar información en bases de datos, enviar datos a través de la red, y 
desplegar imágenes complejas. Sus bases teóricas les permiten determinar el 
mejor desempeño posible, y su estudio de algoritmos les ayuda a desarrollar 
nuevas aproximaciones para proveer un mejor desempeño.
\end{itemize}

Las Ciencias de la Computación cubren todo el rango desde la teoría hasta la programación. Mientras otras disciplinas pueden producir titulados mejor preparados para trabajos específicos, las ciencias de la computación ofrecen un amplio fundamento que permite a sus titulados adaptarse a nuevas tecnologías y nuevas ideas.
%\end{quote}

El profesional en \ac{CC} se preocupa por casi todo en medio de estas áreas. En dirección hacia el hardware, este profesional llega a desarrollar software que permite el funcionamiento de dispositivos {\it devices}. En dirección a aspectos organizacionales, el profesional de \ac{CC} ayuda a que los sistemas de información operen correctamente en las organizaciones. Él genera la tecnología que permite que otras áreas como los sistemas de información se desarrollen adecuadamente.

El profesional en \ac{CC} diseña y desarrolla todo tipo de software, desde infraestructura de plataformas (sistemas operativos, programas de comunicación, etc.) hasta aplicación de tecnologías (navegadores de Internet, bases de datos, motores de búsqueda, etc.). Este profesional crea estas capacidades, pero no está orientado al uso de las mismas. Por lo tanto, el área sombreada (fig. \ref{fig.cs}) para \ac{CC} se estrecha y finaliza en la medida que nos movamos hacia la aplicación y configuración de productos.


\subsection{Diagnóstico específico para la carrera}

\subsubsection{Los grandes retos en el desarrollo de la carrera profesional}
Uno de los caminos que se espera del área de computación en nuestro país pueda producir software a gran escala como las 
grandes empresas productoras de software a nivel mundial y convertirse en referente a nivel nacional, luego latinoamericano y 
finalmente a nivel mundial. En el ámbito de la computación, es común observar que los países cuentan con
Asociaciones de Productores de Software cuyas políticas están orientadas a la exportación. Siendo así, 
no tendría sentido preparar a nuestros alumnos sólo para el mercado local o nacional. 
Siendo un reto que nuestros egresados deben estar preparados para desenvolverse en el mundo globalizado que nos ha tocado vivir.

Es necesario recordar que la mayor innovación de productos comerciales de versiones recientes utiliza tecnología que se conocía 
en el mundo académico hace 20 años o más. Un ejemplo claro son las bases de datos que soportan datos y consultas espaciales 
desde hace muy pocos años. Sin embargo, utilizan estructuras de datos que ya existían hace algunas décadas. 
Es lógico pensar que la gente del área académica no se dedique a estudiar en profundidad la última versión de un 
determinado software cuando esa tecnología ya la conocían hace mucho tiempo. Por esa misma razón es raro en el 
mundo observar que una universidad tenga convenios con una transnacional de software para dictar solamente esa 
tecnología pues, nuestra función es generar esa tecnología y no sólo saber usarla.

\subsubsection{Aporte al desarrollo de las personas, instituciones y sociedad}
Nuestros futuros profesionales deben estar orientados a crear nuevas empresas de base tecnológica que puedan 
incrementar las exportaciones de software peruano. Este nuevo perfil está orientado a generar industria innovadora. 
Si nosotros somos capaces de exportar software competitivo también estaremos en condiciones de atraer nuevas inversiones. 
Las nuevas inversiones generarían más puestos de empleo bien remunerados y con un costo bajo en relación a otros tipos de industria. 
Bajo esta perspectiva, podemos afirmar que esta carrera será un motor que impulsará al desarrollo del país de forma decisiva 
con una inversión muy baja en relación a otros campos.

Tampoco debemos olvidar que los alumnos que ingresan hoy saldrán al mercado dentro de 5 años aproximadamente y, 
en un mundo que cambia tan rápido, no podemos ni debemos enseñarles tomando en cuenta solamente el mercado actual. 
Nuestros profesionales deben estar preparados para resolver los problemas que habrá dentro de 10 o 15 años y 
eso sólo es posible a través de la investigación en los diferentes problemas de nuestra región.


\subsubsection{Empleabilidad de egresados}
Nuestro egresado podrá prestar sus servicios profesionales en empresas e instituciones públicas y privadas que requieran 
sus capacidades en función del desarrollo que oferta, entre ellas:

\begin{itemize}
\item Empresas dedicadas a la producción de software con calidad internacional.
\item Empresas, instituciones y organizaciones que requieran software de calidad para mejorar sus actividades y/o servicios ofertados.
\end{itemize}


Nuestro egresado puede desempeñarse en el mercado laboral sin ningún problema ya que, en general, la exigencia del 
mercado y campo ocupacional está más orientada al uso de herramientas. Sin embargo, es poco común que los propios 
profesionales de esta carrera se pregunten: ?`qué tipo de formación debería tener si yo quisiera crear esas herramientas además de saber usarlas?.  
Ambos perfiles (usuario y creador) son bastante diferentes pues no sería posible usar algo que todavía no fue creado. 
En otras palabras, los creadores de tecnología son los que dan origen a nuevos puestos de trabajo y abren 
la posibilidad de que otros puedan usar esa tecnología.

Debido a la formación basada en la investigación, nuestro profesional debe siempre ser un innovador donde trabaje. 
Esta misma formación permite que el egresado piense también en crear su propia empresa de desarrollo de software. 
Considerando que países como el nuestro tienen un costo de vida mucho menor que Norte América ó Europa, una 
posibilidad que se muestra interesante es la exportación de software pero eso requiere que la calidad del 
producto sea al mismo nivel de lo ofrecido a nivel internacional.

Este perfil profesional también posibilita que nuestros egresados se queden en nuestro país; 
producir software en nuestro país y venderlo fuera es más rentable que salir al extranjero y comercializarlo allá.

El campo ocupacional de un egresado es amplio y están en continua expansión y cambio. 
Prácticamente toda empresa u organización hace uso de servicios de computación de algún tipo, y la buena 
formación básica de nuestros egresados hace que puedan responder a los requerimientos de las mismas exitosamente. 
Este egresado, no sólo podrá dar soluciones a los problemas existentes sino que deberá proponer innovaciones 
tecnológicas que impulsen la empresa hacia un progreso constante.

A medida que la informatización básica de las empresas del país avanza, la necesidad de personas capacitadas 
para resolver los problemas de mayor complejidad aumenta y el plan de estudios que hemos desarrollado tiene 
como objetivo satisfacer esta demanda considerándola a mediano y largo plazo. El campo para las tareas de 
investigación y desarrollo de problemas complejos en computación es también muy amplio y están creciendo día a día a nivel mundial.

Debido a la capacidad innovadora de nuestro egresado, existe una mayor la probabilidad de registrar 
patentes con un alto nivel inventivo lo cual es especialmente importante en nuestros países.

\subsubsection{Competencias requeridas para el mercado laboral}
En un área tan globalizada como lo es la computación el mercado laboral no podría ser solamente local o nacional.
Hoy en dia, en Arequipa, es posible observar empresas de la India, de California, de Alemania, de Uruguay que operan desde nuestra ciudad.
Esto obliga a que la preparación de nuestros egresados tenga que ser competitiva a nivel mundial.

También existen aquellos defensores de que el Perú debe formar profesionales en esta área solo para problemas locales.
Sin embargo, consideramos que es una de las razones de nuestro atraso pues continuamos siendo un país de usuarios de 
tecnología pero no podemos producirla con calidad internacional aun debido al número muy reducido de profesional con perfil internacional. 

Un análisis muy reciente al respecto que nos ayuda a entender mejor como Perú es observado desde fuera puede ser visto en un 
artículo reciente publicado en Boston\footnote{http://bit.ly/2i5lzNM} por el Mag Eddy Wong.
En esta publicación de indica de forma clara que las universidades peruanas han venido formando solamente usuarios de tecnología y 
que además existe un énfasis en mantener asociada la palabra Ingeniería a estas carreras.
Esta formación de usuarios crea una dependencia enorme de nuestro país y dificulta enormemente el desarrollo de la industria nacional.
Es cierto que existen más de 100 carreras de esta línea pero los profesionales unicamente son orientados a hacer uso de la tecnología que es importada a nuestro país.
De parte de las empresas del sector de software existe mucha dificultad para encontrar personal capacitado para crear software de calidad internacional.
Esto puede ser corroborado directamente con el Presidente de la Asociación Peruana de Productores de Software (APESOFT) 
quien con frecuencia reclama este problema al ámbito académico.

\subsubsection{Situación de egresados}
La primera promoción egreso a inicios de 2015 con un total de 6 egresados los cuales fueron aceptados en su totalidad 
para ir a estudiar maestrías con beca a Brasil en área de Ciencia de la Computación.
El hecho de sea el 100\% de egresados en esta situación es un reflejo claro del nivel que se ha podido alcanzar a 
pesar de las dificultades propias de una universidad nacional en nuestro país.
En este momento estos 6 alumnos ya están culminando con éxito sus estudios de maestría e iniciando los de doctorado.

La segunda promoción egresó a inicios de 2016 con un aproximado de 8 alumnos que han seguido un camino similar al primer 
grupo estudiando maestrías becados por concurso a tiempo completo en Arequipa, Holanda, y Brasil.

Estos niveles de competitividad no habrían sido posibles si la formación hubiese sido de usuarios o 
solamente orientada al mercado local o nacional. 

Las competencias que más destacan en nuestros egresados son:
\begin{itemize}
\item Habilidad aprender a aprender de forma autónoma,
\item Capacidad de adaptación rápida a grupos de investigación de nivel internacional,
\item Capacidad de adaptación rápida al trabajo con nuevas tecnologías,
\item Competencia para trabajo multidisciplinario,
\item Competencia para comunicarse de forma efectiva en inglés y en varios casos también en portugues.
\end{itemize}

Como varios de nuestros alumnos han salido a otros países, es nuestra responsabilidad que tengan el 
espacio adecuado en nuestro medio para poder integrarlos y mejorar la calidad académica existente.

\subsubsection{Oportunidades de empleo}
Nuestro egresado podrá prestar sus servicios profesionales en empresas e instituciones públicas y privadas 
que requieran sus capacidades en función del desarrollo que oferta, entre ellas:

\begin{itemize}
\item Empresas dedicadas a la producción de software con calidad internacional.
\item Empresas, instituciones y organizaciones que requieran software de calidad para mejorar sus actividades y/o servicios ofertados.
\end{itemize}

Nuestro egresado puede desempeñarse en el mercado laboral sin ningún problema ya que, en general, la 
exigencia del mercado y campo ocupacional está mucho más orientada al uso de herramientas. Sin embargo, 
es poco común que los propios profesionales de esta carrera se pregunten: ?`qué tipo de formación debería 
tener si yo quisiera crear esas herramientas además de saber usarlas?. Ambos perfiles (usuario y creador) 
son bastante diferentes pues no sería posible usar algo que todavía no fue creado. En otras palabras, 
los creadores de tecnología son los que \underline{dan origen a nuevos puestos de trabajo} y abren la 
posibilidad de que otros puedan usar esa tecnología.

Debido a la formación basada en la investigación, nuestro profesional debe siempre ser un innovador 
donde trabaje. Esta misma formación permite que el egresado piense también en crear su propia empresa 
de desarrollo de software. Considerando que países como el nuestro tienen un costo de vida mucho menor 
que Norte América ó Europa, una posibilidad que se muestra interesante es la exportación de software 
pero eso requiere que la calidad del producto sea al mismo nivel de lo ofrecido a nivel internacional.

Este perfil profesional también posibilita que nuestros egresados se queden en nuestro país; producir 
software en nuestro país y venderlo fuera es más rentable que salir al extranjero y comercializarlo allá.

El campo ocupacional de un egresado es amplio y está en continua expansión y cambio. Prácticamente 
toda empresa u organización hace uso de servicios de computación de algún tipo, y la buena formación 
básica de nuestros egresados hace que puedan responder a los requerimientos de las mismas exitosamente. 
Este egresado, no sólo podrá dar soluciones a los problemas existentes sino que deberá proponer innovaciones 
tecnológicas que impulsen la empresa hacia un progreso constante.

A medida que la informatización básica de las empresas del país avanza, la necesidad de personas 
capacitadas para resolver los problemas de mayor complejidad aumenta y el plan de estudios que hemos 
desarrollado tiene como objetivo satisfacer esta demanda considerandola a mediano y largo plazo. El campo 
para las tareas de investigación y desarrollo de problemas complejos en computación es también muy amplio 
y está creciendo día a día a nivel mundial.

Debido a la capacidad innovadora de nuestro egresado, existe una mayor la probabilidad de registrar 
patentes con un alto nivel inventivo lo cual es especialmente importante en nuestros países.


% \subsubsection{Aporte al desarrollo del conocimiento}
% Las necesidades de los Científicos e Ingenieros con relación a computación han impulsado mucho la 
% investigación y la innovación en computación. A medida que las computadoras aumentan su poder en la 
% solución de problemas, la ciencia computacional ha crecido tanto en amplitud e importancia. 
% Es una disciplina por derecho propio y se considera que es una de las cinco con mayor crecimiento.
% Una increíble variedad de sub-campos han surgido bajo el paraguas de la Ciencia Computacional, 
% incluyendo la biología computacional, química computacional, mecánica computacional, 
% arqueología computacional, finanzas computacionales, sociología computacional y forense.
