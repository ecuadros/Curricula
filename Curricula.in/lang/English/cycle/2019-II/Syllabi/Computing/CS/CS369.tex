\begin{syllabus}

\course{CS369. Computación Evolutiva}{Electivo}{CS369}

\begin{justification}
    La Computación Evolutiva comprende un conjunto de metodologías de búsqueda y optimización cuya base primordial es el Paradigma Neodarwiniano que agrupa la Herencia Genética (Mendel), el Seleccionismo (Weismann) y la Evolución de las Especies (Darwin) que, cuando llevadas a implementaciones computacionales, ofrecen una herramienta poderosa de optimización global para una determinada función objetivo. Son bastante robustos cuando se supone la existencia de muchos óptimos locales. De esta forma, estos algoritmos pueden aplicarse en diversos problemas de optimización.
\end{justification}

\begin{goals}
\item Que el alumno sea capaz de entender y aplicar el Paradigma Neodarwiniano para solucionar problemas complejos de optimización.
\item Entendimiento a detalle del principio, fundamentos teóricos, funcionamiento, implementación, intepretación de resultados y operación de los algoritmos de la Computación Evolutiva más populares y utilizados por la comunidad científica y profesional.
\item Conocimiento del estado del arte en Computación Evolutiva
\item Capacidad de tratar un problema real de optimización utilizando Computación Evolutiva
\end{goals}

\begin{outcomes}
\ExpandOutcome{a}{4}
\ExpandOutcome{b}{5}
\ExpandOutcome{i}{4}
\ExpandOutcome{j}{3}
\end{outcomes}

\begin{unit}{Introducción a la Optimización}{weise09, Rozenberg:2012}{4}{2}
\begin{topics}
        \item Definiciones de Optimización: principio de estabilidad, optimización global.
        \item Optimización Clásica: Definición del problema de optimización, concepto de convexidad, optimización numérica y combinatoria.
				\item Técnicas de optimización clásica: optimización lineal, algoritmo simplex, optimización no lineal, algoritmos \textit{steepest descent, conjugate gradient}, algoritmos de búsqueda, programación dinámica, 
        \item Heurísticas: definición, \textit{Tabu search}, \textit{Hill Climbing} \textit{Simulated Annealing}, \textit{Evolutionary Algorithms}
\end{topics}	
\begin{learningoutcomes}
  \item Entender los principios básicos de la optimización
  \item Entender e implementar algoritmos básicos de Optimización aplicados a problemas \textit{benchmark}. 
	\item Entender la necesidad de uso de heurísticas
\end{learningoutcomes}
\end{unit}

\begin{unit}{Computación Evolutiva: Conceptos básicos}{Rozenberg:2012,weise09,fogel95,koza98,mel98,michalewicz96}{8}{3}
\begin{topics}
        \item Computación Evolutiva: definiciones
        \item Ideas precursoras: El origen de las ideas, L'Eclerc, Lamarck, Darwin, Weismann, Mendel, Baldwin, Paradigma Neodarwiniano
        \item Conceptos básicos de Computación Evolutiva: genes, cromosomas, individuos, población.
        \item Paradigmas de la Computación Evolutiva: Programación Evolutiva, Estrategias Evolutivas, Algoritmos Genéticos, \textit{Learning Classifier Systems}, Programación Genética.
\end{topics}
\begin{learningoutcomes}
  \item Entender los principios básicos que rigen la computación evolutiva
	\item Conocer el contexto en que surgió la computación evolutiva.
\end{learningoutcomes}
\end{unit}

\begin{unit}{Algoritmo Genético Canónico}{Rozenberg:2012,holland75,goldberg89,mel98,michalewicz96}{8}{3}
\begin{topics}
   \item Algoritmo Genético: definición, componentes.
   \item Algoritmo Genético Canónico: procedimiento elemental, ciclo de un AG, representación (codificación binaria, real a binario, decodificación binario a real), inicialización de la población, evaluación y aptitud, selección (proporcional, torneo), operadores genéticos (cruces, mutaciones), el dilema \textit{exploiting-exploring}, ajustes en la aptitud, ajustes en la selección.
	 \item Monitoreo de un AG: curvas \textit{best-so-far, online, off-line}
	 \item Convergencia
   \item Teoría de \textit{Schemata}: Máscaras, esquemas, definiciones y propiedades, \textit{Schemata theorem}: impacto de la selección, cruce de 1 punto y mutación, teorema fundamental de los algoritmos genéticos, hipótesis de los bloques constructores.
\end{topics}
\begin{learningoutcomes}
   \item Entender los algoritmos genéticos tradicionales.
	 \item Analizar y evaluar ventajas y desventajas del modelo genético tradicional.
	 \item Implementar un ejemplo de algoritmo genético tradicional y analizar su comportamiento.
\end{learningoutcomes}
\end{unit}

\begin{unit}{Algoritmos Evolutivos en Optimización Numérica}{Rozenberg:2012,michalewicz96,michalewicz2000,smith2000}{8}{3}
\begin{topics}
   \item Problemas con restricciones: definiciones, espacios válido e inválido.
   \item Tratamiento de las restricciones: Penalización, reparación, uso de codificadores, operadores especializados.
	 \item Uso de codificación real: binario vs. real, algoritmo evolutivo con codificación real.
	 \item Modelo GENOCOP: tratamiento de restricciones lineales, inicialización, operadores, inicialización, modelo GENOCOP III para restricciones no lineales: reparación de individuos.
\end{topics}
\begin{learningoutcomes}
  \item Comprensión de las formas de tratar problemas de optimización con restricciones.
	\item Entender y analizar los algoritmos evolutivos con codificación real.
	\item Evaluar la aplicación de computación evolutiva en problemas de optimización numérica
\end{learningoutcomes}
\end{unit}

\begin{unit}{Algoritmos Evolutivos en Optimización Combinatoria}{Rozenberg:2012,mel98,vargas03}{8}{3}
\begin{topics}
  \item Espacios discretos y finitos
	\item Algoritmos Evolutivos discretos: definición, modelo discreto generalizado
  \item Algoritmos Evolutivos de orden: representación de soluciones, operadores de orden: cruces, mutaciones
	\item Aplicaciones: \textit{Quadratic assignment Problem} -- QAP, \textit{Travelling Salesman Problem} -- TSP
	\item Problemas de Planificación: variables típicas, carácteristicas, representación, codificadores, evaluación de una planificación. 
\end{topics}
\begin{learningoutcomes}
  \item Comprender e identificar el uso de Computación Evolutiva en problemas de optimización combinatoria
  \item Evaluar la aplicación de computación evolutiva en problemas reales discretos
\end{learningoutcomes}
\end{unit}

\begin{unit}{Paralelización y Multi objetivos}{Rozenberg:2012,Cantu-Paz2000,coello07}{8}{5}
\begin{topics}
  \item PEA -- Algoritmos Evolutivos en Paralelo: arquitecturas de paralelización, arquitecturas \textit{master-slave}, \textit{coarse-grained}, \textit{fine-grained} e híbridas 
  \item Análisis de la ejecución de una implementación \textit{master-slave}.
  \item Optimización de Multiples Objetivos: Definición formal, criterio de Pareto, Algoritmos Evolutivos Multi Objetivos (MOEA) sin uso de Pareto, MOEA con uso de Pareto: MOGA, NSGA, NPGA, NPGA2, PESA, SPEA, SPEA-II, Algoritmo Microgenético. 
	\item MOEA -- Métricas de desempeño, investigación futura
\end{topics}
\begin{learningoutcomes}
  \item Comprender y analizar la capacidad de paralelización de los modelos evolutivos
  \item Analizar la aplicabilidad de Computación Evolutiva en problemas de múltiples objetivos
  \item Implementación de modelos paralelos y multiobjetivo
\end{learningoutcomes}
\end{unit}

\begin{unit}{Algoritmos Genéticos Avanzados}{Rozenberg:2012,tarek06,koza92,Reynolds94,Storn95,AbsDaCruz2007}{16}{5}
\begin{topics}
      \item HEA -- Algoritmos Evolutivos Híbridos: Por qué hibridizar?, formas de hibridización, búsqueda local y aprendizaje.
      \item GP -- Programación Genética: definición, representación, ciclo de la GP. 
      \item CA -- Algoritmos Culturales: Evolución Cultural, componentes, procedimiento, espacio de creencia, operadores culturales.
      \item CoEv -- Coevolución: carácteristicas, modelo competitivo, modelo cooperativo.
      \item DE -- Evolución Diferencial: inicialización, operaciones, selección, DE vs. GA, variantes de DE, \textit{Dynamic DE}
      \item QIEA -- Algoritmos Evolutivos con Inspiración Quántica: Computación quántica, algoritmos con inspiración quántica, QIEA-{\bf B}, QIEA-{\bf R}
\end{topics}
\begin{learningoutcomes}
  \item Reconocer y analizar la necesidad de usar Algoritmos Evolutivos más avanzados
  \item Implementación de modelos avanzados de computación evolutiva
\end{learningoutcomes}
\end{unit}



\begin{coursebibliography}
\bibfile{Computing/CS/CS369}
\end{coursebibliography}

\end{syllabus}
