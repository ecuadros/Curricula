\begin{syllabus}

\course{MA100. Mathematics I}{Obligatorio}{MA100}
% Source file: ../Curricula.in/lang/English/cycle/2021-I/Syllabi/BasicSciences/MA100.tex

\begin{justification}
The course aims to develop in students the skills to deal with models in science and engineering related to single variable differential calculus skills. In the course it is studied and applied concepts related to calculation limits, derivatives and integrals of real and vector functions of single real variables to be used as base and support for the study of new contents and subjects. 
Also seeks to achieve reasoning capabilities and applicability to interact with real-world problems by providing a mathematical basis for further professional development activities.
\end{justification}

\begin{goals}
\item Apply knowledge of mathematics.
\item Apply engineering knowledge.
\end{goals}

\begin{outcomes}{V1}
   \item \ShowOutcome{a}{3}
   \item \ShowOutcome{j}{3}
\end{outcomes}

\begin{specificoutcomes}{V1}
   \item \ShowSpecificOutcome{a}{17}{}
   \item \ShowSpecificOutcome{a}{18}{}
   \item \ShowSpecificOutcome{a}{19}{}
   \item \ShowSpecificOutcome{a}{20}{}
   \item \ShowSpecificOutcome{a}{21}{}
   \item \ShowSpecificOutcome{j}{4}{}
   \item \ShowSpecificOutcome{j}{5}{}
\end{specificoutcomes}

\begin{competences}{V1}
   \item \ShowCompetence{C1}{a}
   \item \ShowCompetence{C20}{j}
   \item \ShowCompetence{C24}{j}
\end{competences}

\begin{unit}{Vectors and complex numbers}{}{StewartCOVar,Larson}{20}{C1}
   \begin{topics}
      \item Operations with complex numbers
      \item Theorem Moivre
   \end{topics}

   \begin{learningoutcomes}
      \item  Define and operate with complex numbers, calculating their polar and exponential shape.
      \item  Use Moivre theorem to simplify complex calculations.
      \item Operate with vectors by characterizing them by their direction and magnitude.  Represent a function from the relation of sets, given verbally, graphically and algebraically, in a Venn diagram and/or in the Cartesian plane providing, if possible, its correspondence rule and its main characteristics.
   \end{learningoutcomes}
\end{unit}

\begin{unit}{Functions of a variable}{}{StewartCOVar,Larson}{10}{C20}
   \begin{topics}
      \item Definition, characteristics and graphic representation.
      \item Function algebra.
      \item Linear, polynomial, sinusoidal, exponential and logarithmic functions.
      \item Modeling of situations close to reality using functions.
   \end{topics}

   \begin{learningoutcomes}
      \item Model real situations of the near environment using constant, linear, quadratic and polynomial functions, and others resulting from operations ( f+-*/g, fog , af(bx -c)+d ) between elementary functions, with emphasis on calculation, graphing and interpretation of slope and concavity in an applied context 
      \item Model real-life situations in the immediate environment using sine wave functions.
      \item Use the exponential, logarithmic and logistic functions to model real situations of the near environment that adjust to their behavior, recognizing their characteristics (growth, decrease, asymptotic behavior).
      \item Recognizes and builds trigonometric functions.
      \item Aplicar reglas para transformar funciones.
   \end{learningoutcomes}
\end{unit}

\begin{unit}{Derivatives of functions}{}{StewartCOVar,Larson}{20}{C1}
   \begin{topics}
      \item Definition of derivative as rate of change and as slope of the tangent to the curve at a point.
      \item Referral rules.
      \item Applications of derivadees in related speed problems.
      \item Applications of derivatives in function optimization problems.
   \end{topics}

   \begin{learningoutcomes}
      \item Solve problems using the derivative of a function as a ratio of change between its two variables or as the slope of the tangent line at a point, applying the derivation rules to simple functions.
      \item Approximate functions using the differentials. $df=f'(x)dx$, applying the derivation rules to calculate derivatives of compound and implicit functions with Leibniz notation.
      \item To solve real context problems of the near environment that involve the calculation of related speeds by deriving simple, compound functions and implicitly taking into account the use of differentials.
      \item Solve optimization problems by analyzing the behavior of a function through its first and second derivatives (growth, decrease, concavity, extremes)
   \end{learningoutcomes}
\end{unit}

\begin{unit}{Integral}{}{StewartCOVar,Larson}{22}{C20}
   \begin{topics}
      \item Indefinite integral and integration methods (substitution, integration by parts, trigonometric substitutions and decomposition by partial fractions).
      \item Riemann sum to estimate areas.
      \item Calculation theorems (TFC1, TFC2, TCN).
      \item Calculation of area between curves and average value.
      \item Differential equations that are solved by separable variables.
   \end{topics}

   \begin{learningoutcomes}
      \item Solve undefined integrals by various methods (substitution, integration by parts, trigonometric substitution, decomposition into partial fractions).
      \item Estimate the area under a curve by dividing it into Riemann rectangles and sums, with interpretations in physics and other everyday contexts.  
      \item Apply the calculation theorems (TFC1, TFC2, TCN) to solve undefined integrals using different integration methods.
      \item Solve area and average value problems of a function, with the corresponding physical interpretations of the integral in kinematics.
      \item Model real situations using differential equations and solve them using variable separation method (Newton's Cooling Law, Population Dynamics (Logistics, learning curve), etc.).
      \item It defines a complex number and represents it in various ways. It uses Moivre's formula to calculate operations with complexes.
   \end{learningoutcomes}
\end{unit}

\begin{coursebibliography}
\bibfile{BasicSciences/MA100}
\end{coursebibliography}

\end{syllabus}
