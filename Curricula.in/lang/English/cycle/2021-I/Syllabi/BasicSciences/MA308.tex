% Responsable : Luis Díaz Basurco
% Sumilla de  : Estadísticas y Probabilidades
% Versión     : 1

\begin{syllabus}

\course{MA308. Exploratory Spatial Data Analysis  }{Obligatorio}{MA308}
% Source file: ../Curricula.in/lang/English/cycle/2021-I/Syllabi/BasicSciences/MA308.tex

\begin{justification}
   Provee de una introducción a la teoría de las probabilidades e inferencia estadistica  con aplicaciones, necesarias en el análisis de datos, diseño de modelos aleatorios y toma de decisiones.
\end{justification}

\begin{goals}
\item That the student learns to use the tools of statistics to make decisions in situations of uncertainty.
\item That the student learns to draw conclusions from experimental data.
\item The student will be able to extract useful conclusions about a whole population based on collected information.
\end{goals}

\begin{outcomes}{V1}
   \item \ShowOutcome{a}{2}
   \item \ShowOutcome{i}{2}
   \item \ShowOutcome{j}{3}
\end{outcomes}

\begin{competences}{V1}
    \item \ShowCompetence{C1}{a} 
    \item \ShowCompetence{CS6}{i}
    \item \ShowCompetence{CS2}{j}
\end{competences}


\begin{unit}{}{Descriptive statistics}{Mendenhall97}{6}{C1}
\begin{topics}
      \item Data presentation.
      \item Central location measurements.
      \item Dispersion measures.
   \end{topics}

   \begin{learningoutcomes}
      \item Present summary and description of data. [\Usage]
   \end{learningoutcomes}
\end{unit}

\begin{unit}{}{Probabilities}{Meyer70}{6}{C1}
\begin{topics}
      \item Sample spaces and events.
      \item Axioms and probability properties.
      \item Conditional probability.
      \item Independence.
      \item Bayes' Theorem.
   \end{topics}
   \begin{learningoutcomes}
      \item Identify random spaces. [\Usage]
      \item design probabilistic models. [\Usage]
      \item Identify events as a result of a random experiment. [\Usage]
      \item Calculate the probability of occurrence of an event. [\Usage]
      \item Find the probability using conditionality, independence and Bayes. [\Usage]
   \end{learningoutcomes}
\end{unit}

\begin{unit}{}{Random variable}{Meyer70,Devore98}{6}{CS6}
\begin{topics}
      \item Definition and types of random variables.
      \item Distribution of probabilities.
      \item Density functions.
      \item Expected value.
      \item Moments.
   \end{topics}

   \begin{learningoutcomes}
      \item Identify random variables that describe a sample space. [\Usage]
      \item Build the density distribution or function. [\Usage]
      \item Characterize joint density distributions or functions. [\Usage]
   \end{learningoutcomes}
\end{unit}

\begin{unit}{}{Discrete and continuous probability distribution}{Meyer70,Devore98}{6}{CS6}
\begin{topics}
      \item Basic probability distributions.
      \item Basic Probability Densities.
      \item Random variable functions.
   \end{topics}

   \begin{learningoutcomes}
      \item Calculate probability of a random variable with distribution or density function. [\Usage]
      \item Identify the density distribution or function that describes a random problem. [\Usage]
      \item Testing distribution properties or density functions. [\Usage]
   \end{learningoutcomes}
\end{unit}

\begin{unit}{}{Joint probability distribution}{Meyer70,Devore98}{6}{CS2}
\begin{topics}
      \item Random variables distributed together.
      \item Expected values, covariance and correlation.
      \item The statistics and their distributions.
      \item Distribution of sample averages.
      \item Distribution of a linear combination.

   \end{topics}
   \begin{learningoutcomes}
      \item Find the joint distribution of two discrete or continuous random variables. [\Usage]
      \item Find the marginal or conditional distributions of joint random variables. [\Usage]
      \item Determine dependence or independence of random variables. [\Usage]
      \item Proving properties that are a consequence of the Central Limit Theorem. [\Usage]
   \end{learningoutcomes}
\end{unit}

\begin{unit}{}{Statistical inference}{Meyer70,Devore98}{6}{CS2}
\begin{topics}
      \item Statistical estimation 
      \item Hypothesis testing
      \item Hypothesis testing using ANOVA
   \end{topics}

   \begin{learningoutcomes}
      \item Test whether an estimator is unbiased, consistent, or sufficient. [\Usage]
      \item Find confidence intervals to estimate parameters. [\Usage]
      \item Make parameter decisions based on hypothesis testing. [\Usage]
      \item Test hypotheses using ANOVA. [\Usage]
   \end{learningoutcomes}
\end{unit}

\begin{coursebibliography}
\bibfile{BasicSciences/MA308}
\end{coursebibliography}

\end{syllabus}
