
\begin{syllabus}

\course{MA203. Statistics and Probabilities}{Obligatorio}{MA203}
% Source file: ../Curricula.in/lang/English/cycle/2021-I/Syllabi/BasicSciences/MA203.tex

\begin{justification}
It provides an introduction to probability theory and statistical inference with applications, needs in data analysis, design of random models and decision making.
\end{justification}

\begin{goals}
\item An ability to design and conduct experiments, as well as to analyze and interpret data.
\item An ability to identify, formulate, and solve real problems.
\end{goals}

\begin{outcomes}{V1}
   \item \ShowOutcome{a}{2}
   \item \ShowOutcome{j}{3}
\end{outcomes}

\begin{competences}{V1}
    \item \ShowCompetence{C1}{a} 
    \item \ShowCompetence{CS2}{j}
\end{competences}


\begin{unit}{Variable Type}{}{Sheldon,Menden}{6}{C1}
\begin{topics}
      \item Variable Type: Continuous, discrete
   \end{topics}

   \begin{learningoutcomes}
      \item Classify the relevant variables identified according to their type: continuous (interval and ratio), categorical (nominal, ordinal, dichotomous).
      \item Identify the relevant variables of a system using a process approach.
   \end{learningoutcomes}
\end{unit}

\begin{unit}{Descriptive Statistics}{}{Sheldon,Menden}{6}{C1}
\begin{topics}
      \item Central Tendency (Mean, median, mode)
      \item Dispersion (Range, standard deviation, quartile)
      \item Graphics: histogram, boxplot, etc.: Communication ability.
   \end{topics}
   \begin{learningoutcomes}
      \item Use central tendency measures and dispersion measures to describe the data gathered.
      \item Use graphics to communicate the characteristics of the data gathered.
   \end{learningoutcomes}
\end{unit}

\begin{unit}{Inferential Statistics}{}{Sheldon,Menden}{6}{CS2}
\begin{topics}
      \item Determination of the sample size
      \item Confidence interval
      \item Type I and type II error
      \item Distribution type
      \item Hypothesis test (t-student, means, proportions and ANOVA)
      \item Relationships between variables: correlation, regression.
   \end{topics}

   \begin{learningoutcomes}
      \item Propose questions and hypotheses of interest.
      \item Analyze the data gathered using different statistical tools to answer questions of interest.
      \item Draw conclusions based on the analysis performed.
   \end{learningoutcomes}
\end{unit}





\begin{coursebibliography}
\bibfile{BasicSciences/MA203}
\end{coursebibliography}

\end{syllabus}
