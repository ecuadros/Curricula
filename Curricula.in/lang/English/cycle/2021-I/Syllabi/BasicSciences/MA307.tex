\begin{syllabus}

\course{MA307. Mathematics applied to computing}{Obligatorio}{MA307}
% Source file: ../Curricula.in/lang/English/cycle/2021-I/Syllabi/BasicSciences/MA307.tex

\begin{justification}
   This course is important because it develops topics of Linear Algebra and Ordinary Differential Equations useful in all areas of computer science where one works with linear systems and dynamic systems.
\end{justification}

\begin{goals}
\item That the student has the mathematical basis for the modeling of linear systems and dynamic systems needed in the area of Computer Graphics and Artificial Intelligence.
\end{goals}

\begin{outcomes}{V1}
  \item \ShowOutcome{a}{1}
  \item \ShowOutcome{i}{1}
  \item \ShowOutcome{j}{2}
\end{outcomes}

\begin{specificoutcomes}{V1}
      \item \ShowSpecificOutcome{a}{30}{}
      \item \ShowSpecificOutcome{j}{8}{}
\end{specificoutcomes}

\begin{competences}{V1}
    \item \ShowCompetence{C1}{a} 
    \item \ShowCompetence{C20}{i}
    \item \ShowCompetence{CS2}{j}
\end{competences}

\begin{unit}{}{Linear Spaces}{Strang03, Apostol73}{0}{C1}
\begin{topics}
      \item Vector spaces.
      \item Independence, base and dimension.
      \item Dimensions and orthogonality of the four subspaces.
      \item Approximations by least squares.
      \item Projections
      \item Orthogonal and Gram-Schmidt bases
   \end{topics}
   \begin{learningoutcomes}
      \item Identify spaces generated by linearly independent vectors. [\Usage]
      \item Build orthogonal vector arrays. [\Usage]
      \item Approximate functions by trigonometric polynomials. [\Usage]
   \end{learningoutcomes}
\end{unit}

\begin{unit}{}{Linear transformations}{Strang03, Apostol73}{0}{C20}
\begin{topics}
      \item Concept of linear transformation.
      \item Matrix of a linear transformation.
      \item Change of base.
      \item Diagonalization and pseudo-inversion
   \end{topics}
   \begin{learningoutcomes}
      \item Determining the core and image of a transformation. [\Usage]
      \item Building the matrix of a transformation. [\Usage]
      \item Determine the base change matrix. [\Usage]
      \end{learningoutcomes}
\end{unit}

\begin{unit}{}{Auto-values and auto-vectors}{Strang03, Apostol73}{0}{C24}
\begin{topics}
      \item Diagonalization of a matrix.
      \item Symmetrical matrices.
      \item Positive defined matrices.
      \item Similar matrices.
      \item The decomposition of singular value.
  \end{topics}
 \begin{learningoutcomes}
      \item Finding the diagonal representation of a matrix. [\Usage]
      \item Determining similarity between matrices. [\Usage]
      \item Reducing a real quadratic shape to a diagonal. [\Usage]
   \end{learningoutcomes}
\end{unit}

\begin{unit}{}{Systems of differential equations}{Zill02,Apostol73}{0}{C1}
\begin{topics}
      \item Exponential of a matrix.
      \item Theorems of existence and uniqueness for homogeneous linear systems with constant coefficients.
      \item Non-homogeneous linear systems with constant coefficients.
   \end{topics}
\begin{learningoutcomes}
      \item Finding the overall solution for a non-homogeneous linear system. [\Usage]
      \item Solving problems involving systems of differential equations. [\Usage]
   \end{learningoutcomes}
\end{unit}

\begin{unit}{}{Fundamental theory}{Hirsh74}{0}{C20}
\begin{topics}
      \item Dynamic systems.
      \item The fundamental theorem.
      \item Existence and uniqueness.
      \item The flow of a differential equation.
   \end{topics}
   \begin{learningoutcomes}
      \item Discuss the existence and uniqueness of a differential equation. [\Usage]
      \item Analyze the continuity of solutions. [\Usage]
      \item Study the prolongation of a solution. [\Usage]

   \end{learningoutcomes}
\end{unit}

\begin{unit}{}{Equilibrium stability}{Zill02, Hirsh74}{0}{C24}
\begin{topics}
      \item Stability.
      \item Liapunov features.
      \item Gradient systems.
   \end{topics}
   \begin{learningoutcomes}
      \item Analyze the stability of a solution. [\Usage]
      \item Finding Liapunov's function for balance points. [\Usage]
      \item Drawing the phase portrait a gradient flow. [\Usage]
    \end{learningoutcomes}
\end{unit}

\begin{coursebibliography}
\bibfile{BasicSciences/MA307}
\end{coursebibliography}

\end{syllabus}
