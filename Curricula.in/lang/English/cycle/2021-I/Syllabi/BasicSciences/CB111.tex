\begin{syllabus}

\course{CB111. Computational Physics}{Obligatorio}{CB111}
% Source file: ../Curricula.in/lang/English/cycle/2021-I/Syllabi/BasicSciences/CB111.tex

\begin{justification}
   The course develops the knowledge and skills to recognize, evaluate and apply the effects of physical phenomena related to mechanics in the field of engineering. In industry in general, the control of processes, the operation of machines, their maintenance, etc., are always governed by some kind of physical manifestation. Because of this, it is important for the student to understand the foundations of physical phenomena, the laws that govern them, their manifestation and the way to detect them. This course will allow the student to understand and identify the physical phenomena related to mechanics in order to control their effects on some technical process. 
\end{justification}

\begin{goals}
\item Ability to apply science knowledge.
\item Ability to design and conduct experiments.
\item Ability to apply computer and mathematical knowledge.
\item Ability to develop research principles at an international level.
\end{goals}

\begin{competences}{V1}
  \item \ShowCompetence{C1}{a} 
  \item \ShowCompetence{C20}{i}
  \item \ShowCompetence{CS2}{j}
\end{competences}

\begin{unit}{Work, Energy and Power}{}{YoFreed,Hewitt}{6}{C1}
   \begin{topics}
      \item Definition of work and the relationship between net work and kinetic energy.
      \item Power and Efficiency.
   \end{topics}
   
   \begin{learningoutcomes}
      \item Determine the variables that affect the opposition to translation and opposition to rotation (moment of inertia) and calculate the kinetic energy of translation and rotation.
      \item Calculate the work of a force, apply the Net Work and Energy Theorem to a real life system, and determine the power and efficiency.
   \end{learningoutcomes}
\end{unit}
   
\begin{unit}{}{Kinematics}{YoFreed,Hewitt}{6}{C20}
   \begin{topics}
      \item Spatial and temporal reference systems.
      \item Average speed, average acceleration, linear and angular.
      \item Position, velocity and acceleration vectors, linear and angular
      \item Relationship between linear and angular kinematics.
      \end{topics}
   \begin{learningoutcomes}
      \item Understand the concepts of spatial and temporal reference system kinematics and trajectory and determine position, velocity, linear and angular acceleration, according to a physical or graphical context.
      \item Decompose the linear acceleration, according to a coordinate system, in order to describe the position and in radial and tangential acceleration.
      \item It determines position, speed and acceleration, using differential and integral calculus. 
   \end{learningoutcomes}
\end{unit}
   
\begin{unit}{Newton's three laws}{Newton's three laws}{YoFreed,Hewitt}{6}{C24}
   \begin{topics}
      \item Newton's 3 laws and their application to particles.
      \item Moment of a force.
      \item Rotation of a rigid body.
   \end{topics}

   \begin{learningoutcomes}
      \item To propose the rotation and translation equations for a solid and apply Newton's laws. 
      \item Analyze the characteristics of the friction force. Calculate the net radial force and the net centripetal force.
      \item Calculate the center of mass and analyze the relationship between the variables of net force, time and speed change. 
   \end{learningoutcomes}
\end{unit}

\begin{coursebibliography}
\bibfile{BasicSciences/CB111}
\end{coursebibliography}

\end{syllabus}
