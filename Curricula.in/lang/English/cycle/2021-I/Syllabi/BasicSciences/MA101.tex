\begin{syllabus}

\course{MA101. Math II}{Obligatorio}{MA101}
% Source file: ../Curricula.in/lang/English/cycle/2021-I/Syllabi/BasicSciences/MA101.tex

\begin{justification}
  The course is focused on developing skills in problem understanding, comprehension and application of mathematical models. To this end, an active and participatory methodology is developed with rational use of technology and collaborative work spaces. The sessions are theoretical and practical associated to contextualized situations that motivate the student to get involved in their understanding and solution.
  The course aims to address the following main topics which will be monitored every week, these topics are Vectors, Functions of Several Variables, Partial Derivatives, Double Integrals, Series and Ordinary Differential Equations of first order and second or more order
\end{justification}

\begin{goals}
  \item Ability to apply knowledge of mathematics.
  \item Ability to apply engineering knowledge.
  \item Ability to apply computer and mathematical knowledge
\end{goals}

\begin{outcomes}{V1}
    \item \ShowOutcome{a}{3}  
    \item \ShowOutcome{j}{3}
\end{outcomes}

\begin{competences}{V1}
    \item \ShowCompetence{C1}{a}
    \item \ShowCompetence{C20}{j}
\end{competences}

\begin{unit}{Vectors}{}{StewartMVar,DennisZ}{24}{C1,C20}
   \begin{topics}      
    \item Components, canonical, force or speed problems.
    \item Angle between two vectors, calculate work for a constant force, moment of a force, volume.
    \item Equation of line and plane, Drawing planes, Distance between points, planes and lines.
    \item Calculate work by constant force, moment of a force, volume.
    \item Drawing functions of two and three variables, contour lines.
  \end{topics}

   \begin{learningoutcomes}
    \item Express a vector by its components and use vector operations to interpret the results geometrically, using standard or canonical linear combinations of unit vectors.
    \item Understand the three-dimensional rectangular coordinate system and analyze vectors in space; finding the angle between two vectors and the perpendicular vector between two vectors.
    \item Apply knowledge about vector properties in physical and chemical properties. 
    \item Give a set of parametric equations for a line in space.
    \item Give a linear equation to represent a plane in space, using it to draw the plane given by the linear equation.
    \item Find the distances between points, planes and lines in space.
    \end{learningoutcomes}
\end{unit}

\begin{unit}{Derivatives and Integrals}{}{StewartMVar,DennisZ}{12}{C1,C20}
  \begin{topics}
    \item Interpreting the directional derivatives, error analysis, chain rule.
    \item Directional derivative, gradient of a two-variable function, application.
    \item Absolute and relative extremes / criteria of the second partial derivatives.
    \item Areas, volumes and average values.
    \item Double integrals using polar coordinates.
   \end{topics}
  
  \begin{learningoutcomes}
    \item Understand the notation for a multi-variable function, helping you to draw the graph in space.
    \item Make contour plots of a two-variable function.
    \item Find and use the partial derivatives of a function of two or more variables, to understand the concepts of increments and differentials.
    \item Use a differential as an approximation and use the chain rule for multivariate functions.
    \item Find and use the directional derivatives of a two-variable function, using it to find the gradient of a two-variable function.
    \item Find absolute and relative ends of a two-variable function, using the criterion of the second partial derivatives.
    \item Solve optimization problems with unrestricted and restricted multivariate functions, using the Lagrange multiplier method.
    \item Evaluate and use an iterated integral to find the area of a flat region in Cartesian coordinates.
    \end{learningoutcomes}

\end{unit}

\begin{unit}{Series and Successions}{}{StewartMVar,DennisZ}{24}{C1,C20}
   \begin{topics}
    \item Succession - limit of a succession - recognition of patterns of a succession.
    \item Infinite geometric series - integral and P series criteria.
    \item Quotient criterion / Taylor and Maclaurin polynomials.
    \item Taylor / Maclaurin series.
 \end{topics}

   \begin{learningoutcomes}
    \item Find the mass, center of mass and moments of inertia of a flat sheet using a double integral.
    \item Determine if a succession converges or diverges, using limits and L'Hospital's rule.
    \item Understand the definition of an infinite series using properties to find whether they are convergent or divergent.
    \item Use criteria and properties of the infinite series to determine whether it is convergent or divergent.
    \item Find polynomial approximations of functions using Taylor and Maclaurin polynomials to elementary functions.
    \item Understand the definition of a power series to calculate the radius and range of convergence.
    \item Find a Taylor or Maclaurin series for a function.
     \end{learningoutcomes}
\end{unit}

\begin{unit}{Differential Equations}{}{StewartMVar,DennisZ}{30}{C1,C20}
   \begin{topics}
    \item Definitions and terminologies / Problems with initial values.
    \item Separable variable - Linear equations.
    \item Linear Models of Growth (Population), Decay (Bacteria - Half-life - Mixtures - Newton's Law.)
    \item Exact Equations - Solutions by substitution.
    \item Nonlinear Models (Falling Chain - Logistic Population Growth - Leaking Cylindrical Tank - Inverted Cone, Solar Collector, Immigration Model.
    \item Radioactive Series - Mixed - Mesh.
    \item Nutrient concentration - Newton's Law.
    \item Problems with initial values - homogeneous and non-homogeneous.
    \item Annulator method - Cauchy Euler equation.
      \end{topics}

   \begin{learningoutcomes}
    \item Understand the definitions and terminology of differential equations with and without initial values
    \item Explain 1st and 2nd order differential equation models.
    \item Solve first-order differential equations by the separable variables method.
    \item Solve the homogeneous and non-homogeneous first-order linear differential equations using the integral factor.
    \item Solve exact first-order differential equations with and without initial values, using the integration factor.
    \item Obtain the general solution of a homogeneous second order linear equation with constant coefficients.
    \item Solve the Euler equation of second order, applying to analyze applications in mechanical vibrations and oscillations in electrical circuits.
   \end{learningoutcomes}
\end{unit}

\begin{coursebibliography}
\bibfile{BasicSciences/MA101}
\end{coursebibliography}

\end{syllabus}
