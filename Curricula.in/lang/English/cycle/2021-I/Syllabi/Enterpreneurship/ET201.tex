\begin{syllabus}

\course{ET201. Entrerpreneurship I}{Obligatorio}{ET201}
% Source file: ../Curricula.in/lang/English/cycle/2021-I/Syllabi/Enterpreneurship/ET201.tex

\begin{justification}
   This is the first course in the area of training for
   technological basis, aims to provide the future professional 
   of knowledge, attitudes and skills that will
   allow a business plan to be drawn up for a technology-based company.
   The course is divided into the following units:
   Introduction, Creativity, From Idea to Opportunity, The Canvas Model, Customer Development and Lean Startup, Legal Aspects and Marketing, Company Finance and Presentation.
   
   The aim is to take advantage of the creative and innovative potential and effort of the students in the creation of new companies.
\end{justification}

\begin{goals}
   \item That the student knows how to prepare a business plan to start a technology-based company.
   \item That the student is able to carry out, using business models, the conception and presentation of a business proposal.
\end{goals}

\begin{outcomes}{V1}
   \item \ShowOutcome{d}{3}
   \item \ShowOutcome{f}{3}
   \item \ShowOutcome{m}{2}
\end{outcomes}

\begin{specificoutcomes}{V1}
    \item \ShowSpecificOutcome{d}{4}{}
    \item \ShowSpecificOutcome{f}{1}{}
    \item \ShowSpecificOutcome{f}{2}{}
    \item \ShowSpecificOutcome{m}{1}{}
\end{specificoutcomes}
   
\begin{competences}{V1}
    \item \ShowCompetence{C2}{d}
    \item \ShowCompetence{C10}{f}
    \item \ShowCompetence{C17}{f}
    \item \ShowCompetence{C18}{i}
    \item \ShowCompetence{C19}{i}
    \item \ShowCompetence{C20}{k}
    \item \ShowCompetence{C23}{k}
    \item \ShowCompetence{CS5}{m}
\end{competences}
   
\begin{unit}{}{Introduction}{byers10,osterwalder10,garzozi14}{5}{C2}
\begin{topics}
    \item Entrepreneurship, entrepreneurship and technological innovation.
    \item Business models.
    \item Team building.
\end{topics}

\begin{learningoutcomes} 
    \item Identify characteristics of entrepreneurs. [\Familiarity]
    \item Introducing business models. [\Familiarity]
\end{learningoutcomes} 
\end{unit}
   
\begin{unit}{}{Creativity}{byers10,blank12,garzozi14}{5}{C10}
    \begin{topics}
      \item Vision.
      \item Mission.
      \item The Value Proposition.
      \item Creativity and invention.
      \item Types and sources of innovation.
      \item Strategy and Technology.
      \item Scale and scope.
    \end{topics}

    \begin{learningoutcomes} 
      \item Correctly setting out the company's vision and mission. [\Usage]
      \item Characterize an innovative value proposition. [\Assessment]
      \item Identify the various types and sources of innovation. [\Familiarity]
    \end{learningoutcomes} 
\end{unit}
   
\begin{unit}{}{From Idea to Opportunity}{byers10,osterwalder10,ries11,garzozi14}{5}{C17}
\begin{topics}
    \item Company Strategy.
    \item Barriers .
    \item Sustainable competitive advantage.
    \item Alliances.
    \item Organizational learning.
    \item Product development and design.
\end{topics}

\begin{learningoutcomes} 
    \item Knowing business strategies. [\Familiarity]
    \item Characterize barriers and competitive advantages. [\Familiarity]
\end{learningoutcomes} 
\end{unit}
   
   \begin{unit}{}{The Canvas Model}{osterwalder10,blank12,garzozi14}{20}{C18}
      \begin{topics}
         \item Creating a new business.
         \item The business plan.
         \item Canvas.
         \item Elements of the Canvas.
      \end{topics}
   
     \begin{learningoutcomes} 
         \item Get to know the elements of the Canvas model. [\Usage]
         \item Develop a business plan based on the Canvas model. [\Usage]
       \end{learningoutcomes} 
   \end{unit}
   
   \begin{unit}{}{Customer Development and Lean Startup}{blank12,ries11,garzozi14}{20}{C19}
      \begin{topics}
         \item Acceleration versus incubation.
         \item Customer Development.
         \item Lean Startup.
      \end{topics}
   
      \begin{learningoutcomes} 
         \item Knowing and applying the Customer Development model. [\Usage]
         \item Knowing and applying the Lean Startup model. [\Usage]
       \end{learningoutcomes} 
   \end{unit}
   
   \begin{unit}{}{Legal Aspects and Marketing}{byers10,ries11,congreso96, congreso97,garzozi14}{5}{C20}
      \begin{topics}
         \item Legal and tax aspects for the incorporation of the company.
         \item Intellectual Property.
         \item Patents.
         \item Copyrights and trademarks.
         \item Marketing objectives and market segments.
         \item Market research and customer search.
      \end{topics}
   
     \begin{learningoutcomes} 
         \item Knowing the legal aspects necessary for the formation of a technology company. [\Familiarity]
         \item Identify market segments and marketing objectives. [\Familiarity]
      \end{learningoutcomes} 
   \end{unit}
   
   \begin{unit}{}{Company Finances}{byers10,blank12,garzozi14}{5}{C23}
      \begin{topics}
         \item Cost model.
         \item Utility Model.
         \item Price.
         \item Financial Plan.
         \item Ways of financing.
         \item Sources of capital.
         \item Venture Capital.
      \end{topics}
   
      \begin{learningoutcomes} 
         \item Define a cost and profit model. [\Assessment]
         \item Knowing the various sources of funding. [\Familiarity]
      \end{learningoutcomes} 
   \end{unit}
   
   \begin{unit}{}{Presentation}{byers10,blank12,garzozi14}{5}{CS5}
      \begin{topics}
         \item The Elevator Pitch.
         \item Presentation.
         \item Negotiation.
       \end{topics}
   
      \begin{learningoutcomes} 
         \item Knowing the different ways to present business proposals. [\Familiarity]
         \item Make the presentation of a business proposal. [\Usage]
      \end{learningoutcomes} 
   \end{unit}
   
   \begin{coursebibliography}
   \bibfile{Enterpreneurship/ET201}
   \end{coursebibliography}
   
   \end{syllabus}
   
   %\end{document}
   