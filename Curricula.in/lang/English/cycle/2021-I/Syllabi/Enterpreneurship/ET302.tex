\begin{syllabus}

\course{ET302. Entrerpreneurship III}{Obligatorio}{ET302}
% Source file: ../Curricula.in/lang/English/cycle/2021-I/Syllabi/Enterpreneurship/ET302.tex

\begin{justification}
This course is part of the training area of technology-based companies, 
aims to address all processes and good practices in the 
project management recommended by the \textit{Project Management Institute} (PMI) 
contained in the \textit{Project Management Body of Knowledge 2012} (PMBOK)  
applied in particular to technology-based projects such as 
construction, development, integration and implementation of 
application software.

The future professional who intends to venture into a 
software in the competitive globalised market, it must necessarily 
know the hard skills and practice the soft skills that are 
considered in the PMBOK. All contracts for the supply of goods 
(Hardware) or intangible (Software) as well as the services of 
consulting should be handled as small projects.

We believe it is of utmost importance to impart the fundamentals and experiences 
associated with project management to future professionals, 
we must consider that currently the client companies 
(national or international) that demand solutions require 
consulting companies to carry out system projects 
and information technology with PMI standards, 
more and more turns out to be a condition of exigibility to be able to win 
tenders and sign contracts for the supply of technology solutions, 
It also requires that the project leader, in addition to his or her training and 
experience to bring the project to a successful conclusion is a PMP.
\end{justification}

\begin{goals}
\item That the student masters the concepts related to the management of computer projects.
\item To provide the student with the techniques and tools that allow him/her to successfully manage projects of various magnitudes.
\item That the student builds his business plan oriented to get an international investor who can promote and project the company to an international environment.
\end{goals}

%% (1) familiar  (2)usar (3)evaluar
\begin{outcomes}{V1}
    \item \ShowOutcome{d}{2}
    \item \ShowOutcome{f}{2}
    \item \ShowOutcome{m}{3}
\end{outcomes}

\begin{specificoutcomes}{V1}
   \item \ShowSpecificOutcome{d}{4}{}
   \item \ShowSpecificOutcome{f}{3}{}
   \item \ShowSpecificOutcome{f}{4}{}
   \item \ShowSpecificOutcome{m}{1}{}
\end{specificoutcomes}


\begin{competences}{V1}
    \item \ShowCompetence{C17}{f} 
    \item \ShowCompetence{C18}{d}
    \item \ShowCompetence{C19}{m}
    \item \ShowCompetence{C20}{m}
    \item \ShowCompetence{C21}{m}
    \item \ShowCompetence{C22}{m}
    \item \ShowCompetence{C23}{m}
    \item \ShowCompetence{C24}{m}
\end{competences}

%% Nivel = 1(Familiarity),  2(Usage),  3(Assessment) 
\begin{unit}{Conceptual Framework of Project Management}{}{pmbo08,pmep09}{15}{C19}
\begin{topics}
      \item Introduction.
      \item Purpose of the PMBOK guide, 'What is a project', 'What is project management', The structure of the PMBOK guide, Areas of expertise, context of project management.
      \item Project Life Cycle and Organization.
      \item Project life cycle, project stakeholders, organizational influences.
   \end{topics}

   \begin{learningoutcomes}
      \item To know the conceptual framework in which the projects are developed. [\Usage]
   \end{learningoutcomes}
\end{unit}

\begin{unit}{Standard for the management of a project}{}{pmbo08,pmep09}{15}{C20}
\begin{topics}
      \item Project Management Processes for a Project.
      \item Project management processes, project management process groups, process interactions, correspondence of project management processes.
   \end{topics}

   \begin{learningoutcomes}
      \item Know the standards of project management applied to projects. [\Usage]
   \end{learningoutcomes}
\end{unit}

\begin{unit}{Project management knowledge areas}{}{pmbo08,pmep09}{60}{C23}
\begin{topics}
      \item Introduction.
      \item Project Integration Management.
      \item Project Scope Management.
      \item Project Time Management.
      \item Project Cost Management.
      \item Project Quality Management.
      \item Project Human Resources Management.
      \item Project Communications Management.
      \item Project Risk Management.
      \item Project Procurement Management.
   \end{topics}

   \begin{learningoutcomes}
      \item Understand the nature of project management and its importance to project success. [\Assessment]
      \item Acquire the necessary knowledge to successfully manage projects in terms of: Time, Costs, Scope, Risks, Quality, HR, Procurement, Communications and Integration. [\Usage]
      \item Appreciate the importance of good project management. [\Assessment]
      \item Demonstrate skills in making effective presentations. [\Usage]
      \item Develop skills to manage multidisciplinary work teams. [\Usage]
   \end{learningoutcomes}
\end{unit}


\begin{coursebibliography}
\bibfile{Enterpreneurship/ET302}
\end{coursebibliography}

\end{syllabus}
