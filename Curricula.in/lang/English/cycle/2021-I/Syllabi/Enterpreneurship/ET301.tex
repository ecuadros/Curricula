\begin{syllabus}

\course{ET301. Entrerpreneurship II}{Obligatorio}{ET301}
% Source file: ../Curricula.in/lang/English/cycle/2021-I/Syllabi/Enterpreneurship/ET301.tex

\begin{justification}
   The aim of this course is to provide the future professional with knowledge, attitudes and skills that will enable him/her to form his/her own software development and/or IT consultancy company. The course is divided into three units: Project Assessment, Services Marketing and Negotiations. In the first unit, the student will be able to analyze and make decisions regarding the viability of a project and/or business.

   In the second unit, the aim is to prepare the student to carry out a satisfactory marketing plan of the good or service that his company can offer to the market. The third unit seeks to develop the negotiating skills of the participants through experiential and practical training and theoretical knowledge that will allow them to close contracts where both the client and the supplier are winners. We consider these issues to be extremely critical in the launch, consolidation and eventual re-launching stages of a technology-based company.
\end{justification}

\begin{goals}
\item That the student understands and applies the terminology and fundamental concepts of economic engineering that allow him/her to value a project in order to make the best economic decision.
\item That the student acquires the bases to form his own technology-based company.
\end{goals}

%% (1) familiar  (2)usar (3)evaluar

--COMMON-CONTENT--

%% Nivel = 1(Familiarity),  2(Usage),  3(Assessment) 
\begin{unit}{}{Project Valuation}{blank06}{20}{C19}
\begin{topics}
      \item Introduction.
       \item Decision-making process.
       \item The value of money over time.
       \item Interest Rate and Rate of Return.
       \item Simple interest and compound interest.
       \item Cost identification.
       \item Net Cash Flow.
       \item Return on Investment (ROI).
      \item Net Present Value (NPV).
       \item Project Valuation.
   \end{topics}
   \begin{learningoutcomes}
      \item To allow the student to make decisions on how best to invest the available funds, based on the analysis of both economic and non-economic factors that determine the viability of a venture. [\Assessment]
   \end{learningoutcomes}
\end{unit}

\begin{unit}{}{Marketing de Servicios}{kotler06,love09}{30}{C20}
\begin{topics}
      \item Introduction.
      \item Importance of marketing in service companies.
      \item The Strategic Process.
      \item The Marketing Plan.
      \item Strategic marketing and operational marketing.
      \item Segmentation, targeting and positioning of services in competitive markets.
      \item Product life cycle.
      \item Aspects to be considered in the setting of prices in services.
      \item The role of advertising, sales and other forms of communication.
      \item Consumer behaviour in services.
      \item Fundamentals of Service Marketing.
      \item Creation of the service model.
      \item Service quality management.
   \end{topics}
   \begin{learningoutcomes}
      \item Brindar las herramientas al alumno para que pueda identificar, analizar y aprovechar las oportunidades de marketing que generan valor en un emprendimiento. [\Usage]
      \item To achieve that the student knows, understands and identifies criteria, abilities, methods and procedures that allow an adequate formulation of marketing strategies in specific sectors and media such as a technology-based company. [\Usage]
   \end{learningoutcomes}
\end{unit}

\begin{unit}{}{Negociaciones}{fish96,dasi06}{10}{C18}
\begin{topics}
      \item Introduction. What is a negotiation?.
      \item Theory of negotiation needs.
      \item The negotiation process.
      \item Trading styles.
      \item Game theory.
      \item The Harvard method of negotiation.
   \end{topics}
   \begin{learningoutcomes}
      \item Know the key points in the negotiation process. [\Usage]
      \item Establish an effective negotiation methodology. [\Usage]
      \item To develop skills and abilities that allow to carry out a successful negotiation. [\Usage]
   \end{learningoutcomes}
\end{unit}

\begin{coursebibliography}
\bibfile{Enterpreneurship/ET301}
\end{coursebibliography}

\end{syllabus}

%\end{document}
