\begin{syllabus}

\course{FG350. Leadership and Performance}{Obligatorio}{FG350}
% Source file: ../Curricula.in/lang/English/cycle/2021-I/Syllabi/GeneralEducation/FG350.tex

\begin{justification}
	At present, the different organizations in the world demand from their members the exercise of leadership, this means assuming the challenges assigned with efficiency and eagerness to serve, being these demands necessary for the search of a more just and reconciled society. 
	This challenge involves the need to form our students with a correct knowledge of themselves, with the capacity to judge reality objectively and to propose orientations that seek to positively modify the environment.
\end{justification}

\begin{goals}
\item Develop knowledge, criteria, skills and attitudes to exercise leadership, in order to achieve effectiveness and service in the challenges assigned, thus contributing to building a better society.
\end{goals}

\begin{outcomes}{V1}
    \item \ShowOutcome{d}{2}
    \item \ShowOutcome{f}{2}
    \item \ShowOutcome{ñ}{2}
\end{outcomes}

\begin{competences}{V1}
    \item \ShowCompetence{C17}{f}
    \item \ShowCompetence{C18}{d}
    \item \ShowCompetence{C24}{ñ}
\end{competences}

\begin{unit}{}{First Unit: Foundations of Leadership}{Cardona, Ferreiro,Dianine,DSouza,Sonnenfeld}{15}{C18,C24}
\begin{topics}
	\item Leadership Theories: 
	\item Definition of Leadership.
	\item Fundamentals of Leadership.
	\item Integral Vision of the Human Being and Reasons for Action.
	\item The practice of Virtue in the exercise of Leadership.
\end{topics}
\begin{learningoutcomes}
	\item Analyze and understand the theoretical bases of the Leadership exercise.[\Familiarity]
	\item Based on what is understood, assume the right attitude to put it into practice.[\Familiarity]
	\item Initiate a process of self-knowledge oriented to discover leadership traits in itself.[\Familiarity]
\end{learningoutcomes}
\end{unit}

\begin{unit}{}{Second Unit: Virtues and skills training}{Wilkinson, Huete, Cardona, Chinchilla}{15}{C17,C18,C24}
\begin{topics}
	
	\item Competence Theory.
	\item Recognition of Competencies.
	\item Development Plan.
	\item Mental Models.
	\item Emotional Needs.
	\item Emotional Profiles.
	\item Motivational Vices.

\end{topics}
\begin{learningoutcomes}
	\item To know and develop leadership skills, focused on achieving effectiveness, without neglecting the duty of service to others.[\Familiarity]
	\item Recognize personal and group tendencies necessary for the exercise of Leadership.[\Familiarity]
\end{learningoutcomes}
\end{unit}

\begin{unit}{}{Third Unit: Team Management and Leadership}{Goleman, CardonaP,Hersey, Hunsaker, Hawkins, Ginebra}{18}{C18,C24}
\begin{topics}
	\item The personal relationship with the team.
	\item Integral leadership.
	\item Accompaniment and discipleship.
	\item Fundamentals of Unity.
\end{topics}
\begin{learningoutcomes}
	\item Develop teamwork skills[\Familiarity]
\end{learningoutcomes}
\end{unit}

\begin{coursebibliography}
\bibfile{GeneralEducation/FG350}
\end{coursebibliography}

\end{syllabus}
