\begin{syllabus}

\course{FG203. Oratory}{Obligatorio}{FG203}
% Source file: ../Curricula.in/lang/English/cycle/2021-I/Syllabi/GeneralEducation/FG203.tex

\begin{justification}
	In a competitive society such as ours, it is required that the person be an effective communicator and know how to use his or her potential to solve problems and face the challenges of the modern world within the work, intellectual and social activity. Having knowledge is not enough, the important thing is to know how to communicate it and to the extent that the person knows how to use his or her communicative faculties, what he or she has to do in his or her personal and professional development will derive in success or failure. Therefore it is necessary to achieve a good saying, to resort to knowledge, strategies and resources, which every speaker must have, to reach the interlocutor with clarity, precision and conviction.
\end{justification}

\begin{goals}
\item At the end of the course, the student will be able to organize and assume the word from the speaker's perspective, in any situation, in a more correct, coherent and adequate way, through the use of knowledge and linguistic skills, seeking at all times their personal and social realization through their expression, based on truth and constant preparation.
\end{goals}

\begin{outcomes}{V1}
    \item \ShowOutcome{f}{2}
    \item \ShowOutcome{n}{2}
    \item \ShowOutcome{ñ}{2}
\end{outcomes}

\begin{competences}{V1}
    \item \ShowCompetence{C17}{f,n,ñ}
    \item \ShowCompetence{C24}{f,n,ñ}
\end{competences}

\begin{unit}{}{First Unit: Generalities of Oratory}{Monroe,Rodriguez}{3}{C24}
\begin{topics}
	\item Oratory
	\item The function of the word.
	\item The process of communication.
	\item Rational and emotional basis of public speaking
		\begin{subtopics}
			\item Oral expression in participation.
		\end{subtopics}

	\item Sources of knowledge for public speaking: levels of general culture.
\end{topics}
\begin{learningoutcomes}
	\item Understanding: to interpret, exemplify and generalize the basis of oratory as a theoretical and practical foundation. [\Usage].
\end{learningoutcomes}
\end{unit}

\begin{unit}{}{Second Unit: The Speaker}{Rodriguez}{4}{C17}
\begin{topics}
	\item Qualities of a good speaker.
	\item Rules for first speeches.
	\item The human body as an instrument of communication:
		\begin{subtopics}
			\item Body expression in speech
			\item The voice in the speech.
	   	\end{subtopics}
	\item Speakers with history and their example.
\end{topics}
\begin{learningoutcomes}
	\item Understanding: Interpreting, exemplifying and generalizing knowledge and skills of oral communication through the experience of great speakers and your own. [\Usage].
	\item Application: Implementing, using, choosing and performing the knowledge acquired to express yourself in public in an efficient, intelligent and pleasant way. [\Usage].
\end{learningoutcomes}
\end{unit}

% begin{unit}{}{Tercera Unidad: Elementos técnicos y complementarios del orado}{Rodriguez}{1}{C17}
% begin{topics}
% 	\item Las fichas, apuntes, citas.
% 	\item Recursos técnicos.
% end{topics}
% begin{learningoutcomes}
% 	\item Conocimiento y aplicación: reconocer y utilizar material de apoyo en forma adecuada y correcta para hacer más eficiente su discurso. [\Usage].
% end{learningoutcomes}
% end{unit}
% 
% begin{unit}{}{Cuarta Unidad: El Discurso}{Rodriguez,MonroeEhninger76, Altamirano}{9}{C17, C24}
% begin{topics}
% 	\item El discurso - Los primeros discursos en clase.
% 	\item Clases de discurso.
% 	\item El propósito del discurso. Discursos informativos. Discursos persuasivos. Discursos sociales, de entretenimiento.
% 	\item El auditorio.	
% 	\item Análisis de discursos famosos.
% end{topics}
% begin{learningoutcomes}
% 	\item Síntesis:  crear, elaborar hipótesis, discernir y experimentar al producir sus propios discursos de manera correcta, coherente y oportuna teniendo en cuenta su propósito y hacia quién los dirige.  [\Usage].
% 	\item Evaluación: ponderar, juzgar, relacionar y apoyar sus propios discursos y los de sus compañeros [\Usage].
% end{learningoutcomes}
% end{unit}
% 
% begin{unit}{}{Quinta Unidad: Formas deliberativas en público}{Rodriguez, MonroeEhninger76, McEntee, Fonseca}{6}{C17}
% begin{topics}
% 	\item La participación
% 		 begin{subtopics}
% 			\item La argumentación: organización, tipos de razonamiento.
% 	   	end{subtopics}
% 	\item Formas de grupo.
% 		 begin{subtopics}
% 			\item La mesa redonda.
% 			\item El conversatorio.
% 			\item El simposio.
% 			\item El foro.
% 			\item Panel de discusión.
% 			\item El debate.
% 	   	end{subtopics}
% 	\item Práctica.
% end{topics}
% begin{learningoutcomes}
% 	\item Comprensión: Interpretar, ejemplificar y generalizar
% conocimientos y habilidades de la comunicación oral mediante el conocimiento y aplicación de formas deliberativas orales. [\Usage].
% 	\item Aplicación: Implementar, usar, elegir y desempeñar los conocimientos adquiridos para  expresarse en público en forma eficiente, inteligente y agradable. [\Usage].
% 	\item Síntesis: Crear, elaborar hipótesis, discernir y experimentar al producir sus propios discursos de manera correcta, coherente y oportuna teniendo en cuenta su propósito y hacia quién los dirige.  [\Usage].
% 	\item Evaluación: El alumno puede ponderar, juzgar, relacionar y apoyar sus propios discursos y los de sus compañeros. [\Usage].
% end
% end{learningoutcomes}
% end{unit}
% 
% begin{unit}{}{Sexta Unidad: Discurso Final}{Altamirano, Rodriguez, MonroeEhninger76}{6}{C17}
% begin{topics}
% 	\item Redacción de discursos: Plan de redacción.
% 		begin{subtopics}
% 			\item Planteamiento del tema
% 			\item Selección de información
% 			\item Análisis del auditorio
% 			\item Elaboración del esquema
% 	   	end{subtopics}
% 	\item Sustentación del discurso: Debate.
% end{topics}
% begin{learningoutcomes}
% 	\item Aplicación: Implementar, usar, elegir y desempeñar los conocimientos adquiridos para  elaborar y expresar  en público un discurso en  forma eficiente, inteligente y agradable. [\Usage].
% 	\item Síntesis: Crear, elaborar hipótesis, discernir y experimentar al producir sus propios discursos de manera correcta, coherente y oportuna teniendo en cuenta su propósito y hacia quién los dirige [\Usage].
% 	\item Evaluación:  ponderar, juzgar, relacionar y apoyar sus propios discursos y los de sus compañeros [\Usage].
% end{learningoutcomes}
% end{unit}

\begin{coursebibliography}
\bibfile{GeneralEducation/FG203}
\end{coursebibliography}
\end{syllabus}
