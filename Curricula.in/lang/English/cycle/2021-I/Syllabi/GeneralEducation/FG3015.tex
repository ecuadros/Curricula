\begin{syllabus}

\course{FG3015. Andean and Amazonian Economy}{Electivo}{FG3015}
% Source file: ../Curricula.in/lang/English/cycle/2020-I/Syllabi/GeneralEducation/FG3015.tex

\begin{justification}
The course brings students closer to the study of the non-capitalist economic systems of the peasant and indigenous populations of the Andes and the Amazon of Peru. In addition, students will learn about the relationship between these societies and the environment and the main economic problems that they face today.
\end{justification}

\begin{goals}
\item Capacidad de interpretar información.
\end{goals}

\begin{outcomes}{V1}
    \item \ShowOutcome{d}{2}
    \item \ShowOutcome{e}{2}
    \item \ShowOutcome{n}{2}
    
\end{outcomes}

\begin{competences}{V1}
    \item \ShowCompetence{C10}{d,n}
    \item \ShowCompetence{C17}{d}
    \item \ShowCompetence{C18}{n}
    \item \ShowCompetence{C21}{e}
\end{competences}

\begin{unit}{Culturas de Gobernanza y Distribución de Poder}{}{Lessig15}{12}{4}
   \begin{topics}
      \item ?`Cómo se relaciona la economía con la política?.
      \item El rol de las Instituciones.
      \item Análisis de casos.
   \end{topics}
   \begin{learningoutcomes}
      \item Desarrollo del innterés por conocer sobre temas actuales en la sociedad peruana y el mundo.
   \end{learningoutcomes}
\end{unit}

\begin{coursebibliography}
\bibfile{GeneralEducation/FG3015}
\end{coursebibliography}

\end{syllabus}
