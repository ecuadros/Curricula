\begin{syllabus}

\course{ID202. English  IV}{Obligatorio}{ID202}
% Source file: ../Curricula.in/lang/English/cycle/2021-I/Syllabi/ForeignLanguages/ID104.tex

\begin{justification}
A fundamental part of the integral formation of a professional is the ability to communicate in a foreign language in addition to the native language itself. It not only broadens its cultural horizon but also allows a more humane and comprehensive view of life. In the case of foreign languages, English is undoubtedly the most practical because it is spoken around
all the world. There is no country where it is not spoken. In addition to being vital to your professional career
\end{justification}

\begin{goals}
\item Increase the level of conversation in different subjects, in the students. As well as the ability to write and read documentation of all kinds.
\item Bring the student to a more intense expression in the language domain.
\end{goals}

\begin{outcomes}{V1}
\item \ShowOutcome{f}{2}
\end{outcomes}

\begin{competences}{V1}
\item \ShowCompetence{C25}{f}
\end{competences}

\begin{unit}{Do and don't!}{}{Soars022S, Cambridge06, MacGrew99}{0}{C25}
   \begin{topics}
      \item Mode Auxiliaries should, must and have got to.
      \item Affirmative, negative and interrogative sentences with modals.
      \item Terms for formal letters.
      \item Parts of short answers.
      \item Expressions for occupations.
   \end{topics}

   \begin{learningoutcomes}
      \item At the end of the eighth unit, each of the students, understanding the grammar of should and must auxiliaries, is able to express a greater number of actions in an obligatory and suggestive way.
            Also be able to express ideas describing occupations. Assumes the need to write formal letters
   \end{learningoutcomes}
\end{unit}

\begin{unit}{Going places!}{}{Soars022S, Cambridge06, MacGrew99}{0}{C25}
   \begin{topics}
      \item Present and Future Present Time with Will
      \item First conditional
      \item Collocations
      \item Vocabulary of prepositions of place and time
      \item Expressions of connection of ideas
   \end{topics}

   \begin{learningoutcomes}
      \item At the end of the ninth unit, students having identified how to express present recognize the difference between future forms and apply them properly. They describe conditions accurately. They assume expressions to show place location. They use expressions of time and connectors to unite several ideas.
   \end{learningoutcomes}
\end{unit}

\begin{unit}{Scared to death!}{}{Soars022S, Cambridge06, MacGrew99}{0}{C25}
   \begin{topics}
      \item Infinitive and gerund verb patterns
      \item What + Infinitive
      \item Something + infinitive
      \item Expressions of feelings
      \item Exclamations of surprise
   \end{topics}

   \begin{learningoutcomes}
      \item At the end of the tenth unit of students, the chapters recognize and use the patterns of times in the past properly. They use exclamation marks. And describe feelings. They will use conjunctions to unite type ideas.
   \end{learningoutcomes}
\end{unit}

\begin{unit}{Things that changed the world!}{}{Soars022S, Cambridge06, MacGrew99}{0}{C25}
   \begin{topics}
       \item Passive Voice
       \item Affirmative Prayers, Negatives and Questions
       \item Use of participles, verbs and nouns that go together
       \item Signals. Signs and notes
       \item Summaries
       \item Expressions to indicate prohibition
    \end{topics}

   \begin{learningoutcomes}
      \item At the end of the eleventh unit ,the students having identified the idea of passive actions describe actions appropriately in diverse situations that involve it. They recognize and apply participations. They assume the idea of respecting public signs and signals. They express ideas of habits. They make summaries.
   \end{learningoutcomes}
\end{unit}

\begin{unit}{Dreams and reality!}{}{Soars022S, Cambridge06, MacGrew99}{0}{C25}
   \begin{topics}
       \item Second Conditional
       \item Auxiliar of mode "might"
       \item Phrase Verbs
       \item Social expressions vocabulary
       \item Adverbs
       \item Expressions to give advice
   \end{topics}

   \begin{learningoutcomes}
      \item At the end of the twelfth unit, students, starting from understanding the idea of Conditionals and expressing the possibility of elaborating sentences using the necessary elements. They will also assimilate the need for verbal phrases (2 word verbs). They will acquire vocabulary to describe social expressions.
   \end{learningoutcomes}
\end{unit}

\begin{unit}{Making a living!}{}{Soars022S, Cambridge06, MacGrew99}{0}{C25}
   \begin{topics}
       \item Present Perfect Continuous
       \item Present Continuous
       \item Occupations
       \item Word formation
       \item Adverbs
       \item Expressions of use on the phone
   \end{topics}

   \begin{learningoutcomes}
      \item At the end of the thirteenth unit, they structure sentences with actions that include present and past in appropriate contexts. They emphasize the difference between types of occupations. Use appropriate expressions for telephone conversations.
   \end{learningoutcomes}
\end{unit}

\begin{unit}{All you need is love!}{}{Soars022S, Cambridge06, MacGrew99}{0}{C25}
   \begin{topics}
       \item Past Perfect and Past Simple
       \item Report Expressions
       \item Expressions of words in different contexts
       \item Short and formal farewells
       \item Love Stories
   \end{topics}

   \begin{learningoutcomes}
      \item At the end of the fourteenth unit, students having learned the fundamentals of structuring past perfect time, differentiate it from the simple past. They emphasize the difference between words in different contexts. Describe farewell ideas. They use expressions to write love stories. They assume the idea of giving and doing interviews.
   \end{learningoutcomes}
\end{unit}

\begin{coursebibliography}
\bibfile{ForeignLanguages/ID101}
\end{coursebibliography}

\end{syllabus}
%\end{document}
