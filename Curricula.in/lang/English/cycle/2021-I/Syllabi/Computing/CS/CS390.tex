\begin{syllabus}

\course{CS390. Ingenieríade Software II}{Obligatorio}{CS390}

\begin{justification}
Los tópicos de este curso extienden las ideas del diseño y desarrollo de software desde la secuencia de introducción
a la programación para abarcar los problemas encontrados en proyectos de gran escala. Es una visión m más amplia y
completa de la Ingeniería de Software apreciada desde un punto de vista de Proyectos.
\end{justification}

\begin{goals}
    \item Capacitar a los alumnos para formar parte y definir equipos de desarrollo de software que afronten problemas de envergadura real.
    \item Familiarizar a los alumnos con el proceso de administración de un proyecto de software de tal manera que sea capaz de crear, mejorar y utilizar herramientas ymétricas que  le permitan realizar la estimación y seguimiento de un proyecto de software.
    \item Crear , evaluar e implementar un plan de prueba para segmentos de código de tamaño medio , Distinguir entre los diferentes tipos de pruebas , sentar las bases para crear, mejorar los procedimientos de prueba y las herramientas utilizadas con ese propósito.
    \item Seleccionar con justificación un apropiado conjunto de herramientas para soportar el desarrollo de un rango de productos de software.
    \item Crear , mejorar y utilizar los patrones existentes para el mantenimiento de software . Dar a conocer las características y patrones de diseño para la reutilización de software.
    \item Identificar y discutir diferentes sistemas especializados , crear , mejorar y utilizar los patrones especializados para el diseño, implementación , mantenimiento y prueba de sistemas especializados.
\end{goals}

\begin{outcomes}
    \item \ExpandOutcome{b}{2}
    \item \ExpandOutcome{c}{2}
	\item \ExpandOutcome{f}{2}
	\item \ExpandOutcome{i}{3}
	\item \ExpandOutcome{k}{2}
\end{outcomes}

\begin{competences}
    \item \Competence{C7}{b,k} 
    \item \Competence{C8}{b,c,k} 
    \item \Competence{C11}{c}
	\item \Competence{C12}{c,i}
	\item \Competence{C13}{c,i}
	\item \Competence{C18}{k}
	\item \Competence{CS1}{c}
	\item \Competence{CS2}{b,c}
	\item \Competence{CS4}{b,c,i}
	\item \Competence{CS5}{b,c,i}
	\item \Competence{CS10}{i,k}
\end{competences}

\begin{unit}{\SESoftwareConstruction}{}{20}{b,c,i,k}
	\begin{topics}
		\item \SESoftwareConstructionTopicCoding
		\item \SESoftwareConstructionTopicCodingStandards
		\item \SESoftwareConstructionTopicIntegration
		\item \SESoftwareConstructionTopicDevelopment
		\item \SESoftwareConstructionTopicPotential
	\end{topics}
	\begin{learningoutcomes}
		\item \SESoftwareConstructionLODescribeTechniques[\Assessment]
		\item \SESoftwareConstructionLOBuild[\Assessment]
		\item \SESoftwareConstructionLODescribeSecure[\Assessment]
		\item \SESoftwareConstructionLOSelectAndDefined[\Assessment]
		\item \SESoftwareConstructionLOCompareAndStrategies[\Assessment]
		\item \SESoftwareConstructionLODescribeTheAnalyzing[\Assessment]
		\item \SESoftwareConstructionLODescribeTheAnalyzingChanges[\Assessment]
		\item \SESoftwareConstructionLORewrite[\Assessment]
		\item \SESoftwareConstructionLOWriteAThatNon[\Assessment]
	\end{learningoutcomes}
\end{unit}

\begin{unit}{\SESoftwareVerificationandValidation}{Pressman2005,Sommerville2008,Larman2008}{20}{b,c,i,k}
	\begin{topics}
		\item \SESoftwareVerificationandValidationTopicVerification%
		\item \SESoftwareVerificationandValidationTopicInspections%
		\item \SESoftwareVerificationandValidationTopicTestingTypes%
		\item \SESoftwareVerificationandValidationTopicTestingFundamentals%
		\item \SESoftwareVerificationandValidationTopicDefect%
		\item \SESoftwareVerificationandValidationTopicLimitationsOfTesting%
		\item \SESoftwareVerificationandValidationTopicStaticApproaches%
		\item \SESoftwareVerificationandValidationTopicTest%
		\item \SESoftwareVerificationandValidationTopicValidation%
		\item \SESoftwareVerificationandValidationTopicObjectOrientedTesting%
		\item \SESoftwareVerificationandValidationTopicVerificationAndValidationOf%
		\item \SESoftwareVerificationandValidationTopicFaultLogging%
		\item \SESoftwareVerificationandValidationTopicFaultEstimation%
	\end{topics}
	\begin{learningoutcomes}%Usage Familiarity Assessment
		\item \SESoftwareVerificationandValidationLODistinguishBetween [\Familiarity] %
		\item \SESoftwareVerificationandValidationLODescribeTheTools [\Familiarity] %
		\item \SESoftwareVerificationandValidationLOUndertake [\Usage] %
		\item \SESoftwareVerificationandValidationLODescribeAnd [\Familiarity] %
		\item \SESoftwareVerificationandValidationLODescribeTechniquesSignificant [\Familiarity] %
		\item \SESoftwareVerificationandValidationLOCreateAndSet [\Usage] %
		\item \SESoftwareVerificationandValidationLODescribeHowGood [\Familiarity] %
		\item \SESoftwareVerificationandValidationLOUseAToolSoftware [\Usage] %
		\item \SESoftwareVerificationandValidationLODiscussTheTesting [\Familiarity] %
		\item \SESoftwareVerificationandValidationLOEvaluateAFor [\Usage] %
		\item \SESoftwareVerificationandValidationLOCompareStatic [\Familiarity] %
		\item \SESoftwareVerificationandValidationLOIdentifyTheOfDevelopment [\Familiarity] %
		\item \SESoftwareVerificationandValidationLODiscussTheThe [\Usage] %
		\item \SESoftwareVerificationandValidationLODescribeTechniquesVerification [\Familiarity] %
		\item \SESoftwareVerificationandValidationLODescribeApproachesEstimation [\Familiarity] %
		\item \SESoftwareVerificationandValidationLOEstimateThe [\Usage] %
		\item \SESoftwareVerificationandValidationLOConductAn [\Usage] %
	\end{learningoutcomes}
\end{unit}

\begin{unit}{\SEToolsandEnvironments}{Pressman2014, Sommerville2010}{20}{b,c,i,k}
	\begin{topics}
		\item \SEToolsandEnvironmentsTopicSoftware
		\item \SEToolsandEnvironmentsTopicRelease
		\item \SEToolsandEnvironmentsTopicRequierements
		\item \SEToolsandEnvironmentsTopicTesting
		\item \SEToolsandEnvironmentsTopicProgramming
		\item \SEToolsandEnvironmentsTopicTool
	\end{topics}
	\begin{learningoutcomes}%Usage Familiarity Assessment
		\item \SEToolsandEnvironmentsLODescribeTheCentralized
		\item \SEToolsandEnvironmentsLODescribeHowCanToK
		\item \SEToolsandEnvironmentsLOIdentifyConfiguration
		\item \SEToolsandEnvironmentsLODescribeHowAnd
		\item \SEToolsandEnvironmentsLODescribeTheAre
		\item \SEToolsandEnvironmentsLODemonstrateTheUse
	\end{learningoutcomes}
\end{unit}



\begin{coursebibliography}
\bibfile{Computing/CS/CS392}
\end{coursebibliography}

\end{syllabus}
