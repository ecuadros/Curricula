\begin{syllabus}

	\course{CB309. Bioinformatics}{Obligatorio}{CB309}
	% Source file: ../Curricula.in/lang/English/cycle/2021-I/Syllabi/BasicSciences/CB309.tex
	
	\begin{justification}
	The use of computational methods in the biological sciences has become one of the key tools for the field of molecular biology, being a fundamental part of research in this area.
	
	In Molecular Biology, there are several applications that involve both DNA, protein analysis or sequencing of the human genome, which depend on computational methods. Many of these problems are really complex and deal with large data sets.
	
	This course can be used to see concrete use cases of several areas of knowledge of Computer Science such as Programming Languages (PL), Algorithms and Complexity (AL), Probabilities and Statistics, Information Management (IM), Intelligent Systems (IS).
	\end{justification}
	
	\begin{goals}
	\item That the student has a solid knowledge of molecular biological problems that challenge computing.
	\item That the student is able to abstract the essence of the various biological problems to pose solutions using their knowledge of Computer Science
	\end{goals}
	
	--COMMON-CONTENT--
	
	\begin{unit}{Introduction to Molecular Biology}{}{Clote2000,Setubal1997}{4}{CS1}
		 \begin{topics}
			\item ...
			\item ...
			\item ...
		 \end{topics}
		 \begin{learningoutcomes}
				\item ... [\Familiarity]
				\item ... [\Assessment]
		 \end{learningoutcomes}
	\end{unit}
	
	\begin{coursebibliography}
	\bibfile{BasicSciences/CB309}
	\end{coursebibliography}
	
	\end{syllabus}
	