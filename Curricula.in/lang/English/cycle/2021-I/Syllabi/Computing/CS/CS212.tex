\begin{syllabus}

\course{CS212. Analysis and Design of Algorithms}{Obligatorio}{CS212}
% Source file: ../Curricula.in/lang/English/cycle/2021-I/Syllabi/Computing/CS/CS212.tex

\begin{justification}
An algorithm is, essentially, a well-defined set of rules or instructions that allow solving a computational problem.
The theoretical study of the performance of the algorithms and the resources used by them, usually time and space, allows us to evaluate if an algorithm is suitable for solving a specific problem, comparing it with other algorithms for the same problem or even delimiting the boundary between Viable and impossible.
This matter is so important that even Donald E. Knuth defined Computer Science as the study of algorithms.
This course will present the most common techniques used in the analysis and design of efficient algorithms,
with the purpose of learning the fundamental principles of the design, implementation and analysis of algorithms for the solution of computational problems
\end{justification}

\begin{goals} 
\item Develop the ability to evaluate the complexity and quality of algorithms proposed for a given problem.
\item Study the most representative, introductory algorithms of the most important classes of problems treated in computation.
\item Develop the ability to solve algorithmic problems using the fundamental principles of algorithm design learneds.
\item Be able to answer the following questions when a new algorithm is presented: How good is the performance ?, Is there a better way to solve the problem?
\end{goals}

\begin{outcomes}{V1}
    \item \ShowOutcome{a}{3} 
    \item \ShowOutcome{b}{3} 
\end{outcomes}

\begin{specificoutcomes}{V1}
    \item \ShowSpecificOutcome{a}{8}{}
    \item \ShowSpecificOutcome{a}{9}{}
    \item \ShowSpecificOutcome{b}{1}{}
    \item \ShowSpecificOutcome{b}{3}{}
    \item \ShowSpecificOutcome{b}{4}{}
\end{specificoutcomes}
.
\begin{outcomes}{V2}
    \item \ShowOutcome{1}{2} 
    \item \ShowOutcome{2}{2} 
\end{outcomes}

\begin{competences}{V1}
    \item \ShowCompetence{C1}{a} 
    \item \ShowCompetence{C2}{b} 
    \item \ShowCompetence{C3}{b} 
    \item \ShowCompetence{C5}{a} 
    \item \ShowCompetence{C6}{a} 
    \item \ShowCompetence{C9}{a} 
\end{competences}

\begin{competences}{V2}
    \item \ShowCompetence{C1}{1} 
    \item \ShowCompetence{C2}{2} 
    \item \ShowCompetence{C3}{2} 
    \item \ShowCompetence{C5}{1} 
    \item \ShowCompetence{C6}{1} 
    \item \ShowCompetence{C9}{2} 
\end{competences}

\begin{unit}{\ALBasicAnalysis}{}{KT2005, DPV2006, CLRS2009, S2013, K1997}{10}{a}
\begin{topics}%
    \item \ALBasicAnalysisTopicDifferences % Differences between the best, the expected and the worst case of an algorithm.
    \item \ALBasicAnalysisTopicAsymptotic % Asymptotic analysis of upper and expected complexity of dimensions.
    \item \ALBasicAnalysisTopicComplexity % Complexity classes such as constant, logarithmic, linear, quadratic, and exponential.
    \item Asymptotic Notation
    \item \ALBasicAnalysisTopicAnalysis % Analysis of iterative and recursive algorithms.
    \item Inductive proofs and correctness of algorithms
    \item \ALBasicAnalysisTopicSome % Some versions of the Master Theorem.
\end{topics}
\begin{learningoutcomes}
    \item \ALBasicAnalysisLOExplain [\Assessment] % Explain to refer to "better", "expected" and "worse" behavioral case of an algorithm.
    \item \ALBasicAnalysisLODetermine [\Assessment] % Informally determine the time and complexity space of simple algorithms.
    \item \ALBasicAnalysisLOList [\Assessment] % List and contrast of standard classes of complexity.
    \item \ALBasicAnalysisLOExplainThe [\Assessment] % Explain the use of the big, large, and small omega notation to describe the amount of work done by an algorithm.
    \item Analyze worst-case running times of algorithms using asymptotic analysis [\Assessment]
    \item \ALBasicAnalysisLOUseRecurrence [\Assessment] % Use recurring relations to determine the complexity time of recursively defined algorithms.
    \item \ALBasicAnalysisLOSolve [\Assessment] % Solve basic recurrence relations, for example. Using some form of the Master Theorem
    \item Argue the correctness of algorithms using inductive proofs [\Assessment]
\end{learningoutcomes}
\end{unit}

\begin{unit}{\ALAlgorithmicStrategies}{}{KT2005, DPV2006, CLRS2009, A1999}{30}{a,b}
\begin{topics}%
    \item \ALAlgorithmicStrategiesTopicBrute % Brute force algorithms
    \item \ALAlgorithmicStrategiesTopicGreedy % Voracious Algorithms.
    \item \ALAlgorithmicStrategiesTopicDivide % Divide and conquer
    \item \ALAlgorithmicStrategiesTopicDynamic % Dynamic Programming.
\end{topics}
\begin{learningoutcomes}
    \item \ALAlgorithmicStrategiesLOFor [\Assessment] % For each of the strategies (brute force, greedy algorithm, divide and conquer, reverse recursion and dynamic programming), it identifies a practical example in which it can be applied.
    \item \ALAlgorithmicStrategiesLOUseA [\Assessment] % It uses a voracious approach to solve a specific problem and determines if the chosen rule guides it to an optimal solution.
    \item \ALAlgorithmicStrategiesLOUseAConquer [\Assessment] % Use a divide and solve algorithm to solve a given problem
    \item \ALAlgorithmicStrategiesLOUseDynamic [\Assessment] % Use dynamic programming to solve a given problem
    \item \ALAlgorithmicStrategiesLODetermineAn [\Assessment] % Determines the right algorithmic approach for a problem.
\end{learningoutcomes}
\end{unit}

\begin{unit}{\ALFundamentalDataStructuresandAlgorithms}{}{KT2005, DPV2006, CLRS2009, S2011, GT2009}{6}{a,b}
\begin{topics}%
    \item \ALFundamentalDataStructuresandAlgorithmsTopicGraphsAnd % "Graphs and algorithms in graphs:
									      % begin{subtopic} 
										  % \item Shortest Path Algorithms (Dijkstra and Floyd Algorithms)
										  % \item Minimum expansion tree (Prim and Kruskal algorithm)
                                            % end{subtopic}"
    \item Cache oblivious algorithms 
    \item Number theory and cryptography
\end{topics}
\begin{learningoutcomes}
    \item \ALFundamentalDataStructuresandAlgorithmsLODiscussFactors [\Familiarity] % Discuss factors other than computational efficiency that influence the choice of algorithms, such as programming time, maintainability, and the use of application-specific patterns in the input data.
    \item \ALFundamentalDataStructuresandAlgorithmsLOSolveProblems [\Assessment] % Solve problems using basic algorithms of graphs, including search by depth and search by amplitude.
    \item \ALFundamentalDataStructuresandAlgorithmsLODemonstrate [\Assessment] % Demonstrate ability to evaluate algorithms, to select from a range of possible options, to provide a justification for that selection, and to implement the algorithm in a specific context.
    \item \ALFundamentalDataStructuresandAlgorithmsLOSolveProblemsAlgorithms [\Assessment] % Solve problems using graph algorithms, including shorter single source path and shorter path of all pairs,and at least a minimal expansion tree algorithm..
\end{learningoutcomes}
\end{unit}

\begin{unit}{\ALBasicAutomataComputabilityandComplexity}{}{KT2005, DPV2006, CLRS2009}{2}{a,b}
\begin{topics}%
    \item \ALBasicAutomataComputabilityandComplexityTopicIntroduction % Introduction to the classes P and NP and the problem P vs. NP.
    \item \ALBasicAutomataComputabilityandComplexityTopicIntroductionTo % Introduction and examples of problems NP- Complete and NP-Complete classes.
    \item Reductions
\end{topics}
\begin{learningoutcomes}
    \item \ALBasicAutomataComputabilityandComplexityLODefine [\Familiarity] % Define the classes P and NP
    \item \ALBasicAutomataComputabilityandComplexityLOExplainTheNp [\Familiarity] % Explain the meaning of NP-Completitud
\end{learningoutcomes}
\end{unit}

\begin{unit}{\ALAdvancedDataStructuresAlgorithmsandAnalysis}{}{KT2005, DPV2006, CLRS2009, T1983, R1992}{12}{a,b}
\begin{topics}%
    \item \ALAdvancedDataStructuresAlgorithmsandAnalysisTopicGraphs % Graphs (eg topological ordering, finding strongly connected components)
    \item \ALAdvancedDataStructuresAlgorithmsandAnalysisTopicRandomized % Random algorithms.
    \item \ALAdvancedDataStructuresAlgorithmsandAnalysisTopicAmortized % Amortized analysis.
    \item \ALAdvancedDataStructuresAlgorithmsandAnalysisTopicProbabilistic % Probabilistic Analysis.
    \item Approximation Algorithms 
    \item Linear Programming
\end{topics}
\begin{learningoutcomes}
    \item \ALAdvancedDataStructuresAlgorithmsandAnalysisLOUnderstand [\Familiarity]% Understand the mapping of real-world problems to algorithmic solutions (eg, gran) to solve real problems.
    \item \ALAdvancedDataStructuresAlgorithmsandAnalysisLOSelectAnd [\Usage] % Select and apply advanced techniques of analysis (example, amortized, probabilistic, etc.) for algorithms.
\end{learningoutcomes}
\end{unit}

\begin{coursebibliography}
\bibfile{Computing/CS/CS212}
\end{coursebibliography}

\end{syllabus}
