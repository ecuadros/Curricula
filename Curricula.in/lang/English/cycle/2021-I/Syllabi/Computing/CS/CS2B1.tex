\begin{syllabus}

\course{CS2B1. Platform Based Development}{Obligatorio}{CS2B1}
% Source file: ../Curricula.in/lang/English/cycle/2021-I/Syllabi/Computing/CS/CS2B1.tex

\begin{justification}
The world has changed due to the use of fabric and related technologies, rapid, timely and personalized access to the
information, through web technology, ubiquitous and pervasive; they have changed the way we do things, how do we think? and how does the industry develop?
Web technologies, ubiquitous and pervasive are based on the development of web services, web applications and mobile applications,
which are necessary to understand the architecture, design, and implementation of web services, web applications and mobile applications.
\end{justification}

\begin{goals}
    \item That the student is able to design and implement services, web applications using tools and languages such as HTML, CSS, JavaScript (including AJAX), back-end scripting and a database, at an intermediate level.
    \item That the student is able to develop mobile applications, administration of web servers in a Unix system and an introduction to web security, at an intermediate level.
\end{goals}

--COMMON-CONTENT--

\begin{unit}{\PBDIntroduction}{}{fielding2000fielding, grove2009web,annuzzi2013introduction,Cornez2015}{5}{g}
\begin{topics}%
    \item \PBDIntroductionTopicOverview
    \item \PBDIntroductionTopicProgramming
    \item \PBDIntroductionTopicOverviewOf
    \item \PBDIntroductionTopicProgrammingUnder
\end{topics}
\begin{learningoutcomes}
    \item \PBDIntroductionLODescribeHowDevelopment [\Familiarity]
    \item \PBDIntroductionLOListCharacteristics [\Familiarity]
    \item \PBDIntroductionLOWriteAnd [\Familiarity]
    \item \PBDIntroductionLOListTheDisadvantages [\Familiarity]
\end{learningoutcomes}
\end{unit}

\begin{unit}{\PBDWebPlatforms}{}{fielding2000fielding}{5}{c,g,i}
\begin{topics}%
    \item \PBDWebPlatformsTopicWeb
    \item \item Web Platform constraints: Client-Server, Stateless-Stateful, Cache, Uniform Interface, Layered System, Code on Demand, ReST.
    \item \PBDWebPlatformsTopicWebPlatform
    \item \PBDWebPlatformsTopicSoftware
    \item \PBDWebPlatformsTopicWebStandards
\end{topics}
\begin{learningoutcomes}
    \item \PBDWebPlatformsLODesignAndSimple [\Familiarity]
    \item \PBDWebPlatformsLODescribeTheTheOn [\Familiarity]
    \item \PBDWebPlatformsLOCompareAndProgramming [\Familiarity]
    \item \PBDWebPlatformsLODescribeTheSoftware [\Familiarity]
    \item \PBDWebPlatformsLODiscussHowImpact [\Familiarity]
    \item \PBDWebPlatformsLOReview [\Familiarity]
\end{learningoutcomes}
\end{unit}

\begin{unit}{Desarrollo de servicios y aplicaciones web}{}{freeman2011head}{25}{c,d,g,i}
   \begin{topics}
    \item Describe, identify and debug issues related to web application development
    \item Design and development of interactive web applications using HTML5 and Python
    \item Use MySQL for data management and manipulate MySQL with Python
    \item Design and development of asynchronous web applications using Ajax techniques
    \item Using dynamic client side Javascript scripting language and server side python scripting language with Ajax
    \item Apply XML / JSON technologies for data management with Ajax
    \item Use framework, services and Ajax web APIs and apply design patterns to web application development
   \end{topics}
   \begin{learningoutcomes}
      \item Server-side python scripting language: variables, data types, operations, strings, functions, control statements, arrays, files and directory access, maintain state. [\Usage]
      \item Web programming approach using embedded python. [\Usage]
      \item Accessing and Manipulating MySQL. [\Usage]
      \item The Ajax web application development approach. [\Usage]
      \item DOM and CSS used in JavaScript. [\Usage]
      \item Asynchronous Content Update Technologies. [\Usage]
      \item XMLHttpRequest objects use to communicate between clients and servers. [\Usage]
      \item XML and JSON. [\Usage]
      \item XSLT and XPath as mechanisms for transforming XML documents. [\Usage]
      \item Web services and APIs (especially Google Maps). [\Usage]
      \item Macros Ajax  for the development of contemporary web applications. [\Usage]
      \item Design patterns used in web applications. [\Usage]
   \end{learningoutcomes}
\end{unit}

\begin{unit}{\PBDMobilePlatforms}{}{martin2017clean, annuzzi2013introduction}{5}{c,d,g,i}
\begin{topics}%
    \item \PBDMobilePlatformsTopicMobile
    \item Design Principles: Segregation of Interfaces, Single Responsability, Separation of concerns, Dependency Inversion.
    \item \PBDMobilePlatformsTopicChallenges
    \item \PBDMobilePlatformsTopicLocation
    \item \PBDMobilePlatformsTopicPerformance
    \item \PBDMobilePlatformsTopicMobilePlatform
    \item \PBDMobilePlatformsTopicEmerging
\end{topics}
\begin{learningoutcomes}
    \item \PBDMobilePlatformsLODesignAndMobile [\Familiarity]
    \item \PBDMobilePlatformsLODiscussTheMobile [\Familiarity]
    \item \PBDMobilePlatformsLODiscussThePower [\Familiarity]
    \item \PBDMobilePlatformsLOCompareAndProgrammingPurpose [\Familiarity]
\end{learningoutcomes}
\end{unit}

\begin{unit}{Mobile Applications for Android Handheld Systems}{}{annuzzi2013introduction,Cornez2015}{25}{c,d,g,i}
\begin{topics}
    \item The Android Platform
    \item The Android Development Environment
    \item Application Fundamentals
    \item The Activity Class
    \item The Intent Class
    \item Permissions
    \item The Fragment Class
    \item User Interface Classes
    \item User Notifications
    \item The BroadcastReceiver Class
    \item Threads, AsyncTask \& Handlers
    \item Alarms
    \item Networking (http class)
    \item Multi-touch \& Gestures
    \item Sensors
    \item Location \& Maps
\end{topics}

\begin{learningoutcomes}
    \item Students identify necessary software and install it on their personal computers.
    \item Students perform various tasks to familiarize themselves with the Android platform and Environment for development. [\Usage]
    \item Students build applications that trace the lifecycle callback methods emitted by the Android platform and demonstrate the behavior of Android when device configuration changes (for example, when the device moves from vertical to horizontal and vice versa ). [\Usage]
    \item Students build applications that require starting multiple activities through both standard and custom methods. [\Usage]
    \item Students build applications that require standard and custom permissions. [\Usage]
    \item Students build an application that uses a single code base, but creates different user interfaces depending on the screen size of a device. [\Usage]
    \item Students construct a to-do list manager using the user interface elements discussed in class. The application allows users to create new items and to display them in a ListView. [\Usage]
    \item Students build an application that uses location information to collect latitude, length of places they visit. [\Usage]
\end{learningoutcomes}
\end{unit}

\begin{coursebibliography}
\bibfile{Computing/CS/CS2B1}
\end{coursebibliography}

\end{syllabus}
