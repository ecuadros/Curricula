\begin{syllabus}

\course{CS361. Topics in Artificial Intelligence}{Electivo}{CS361}
% Source file: ../Curricula.in/lang/English/cycle/2021-I/Syllabi/Computing/CS/CS361.tex

\begin{justification}
  It provides a set of tools to solve problems that are difficult to solve with traditional algorithmic methods. Including heuristics, planning, formalisms in the representation of knowledge and reasoning, machine learning techniques, techniques applicable to action and reaction problems: as well as the learning of natural language, artificial vision and robotics among others. 
\end{justification}

\begin{goals}
\item Take an advanced course in Artificial Intelligence suggested by the ACM/IEEE curriculum.
\end{goals}

--COMMON-CONTENT--

\begin{unit}{}{State of the art survey}{Russell03,Haykin99,Goldberg89}{60}{a,h}
\begin{topics}
  \item Intelligent Systems.
  \item Automated Reasoning.
  \item Knowledge Based Systems.
  \item Machine Learning. \cite{Russell03},\cite{Haykin99}
  \item Planning Systems.
  \item Natural Language Processing.
  \item Agents.
  \item Robotics.
  \item Symbolic Computing.
  \item Genetic Algorithms. \cite{Goldberg89}
\end{topics}
\begin{learningoutcomes}
  \item To deepen in several techniques related to Artificial Intelligence. [\Usage]
\end{learningoutcomes}
\end{unit}

\begin{coursebibliography}
\bibfile{Computing/CS/CS369}
\end{coursebibliography}

\end{syllabus}
