\begin{syllabus}

	\course{CS272. Databases II}{Obligatorio}{CS272}
	% Source file: ../Curricula.in/lang/English/cycle/2021-I/Syllabi/Computing/CS/CS272.tex
	
	\begin{justification}
	Information Management (IM) plays a leading role in almost every area where computers are used. This area includes the capture, digitization, representation, organization, transformation and presentation of information; Algorithms to improve the efficiency and effectiveness of access and update of stored information, data modeling and abstraction, and physical file storage techniques.
	
	It also covers information security, privacy, integrity and protection in a shared environment. Students need to be able to develop conceptual and physical data models, determine which IM methods and techniques are appropriate for a given problem, and be able to select and implement an appropriate IM solution that reflects all applicable constraints, including scalability and Usability.
	\end{justification}
	
	\begin{goals}
	\item To make the student understand the different applications that the databases have, in the different areas of knowledge.
	\item Show appropriate ways of storing information based on their various approaches and their subsequent retrieval of information.
	\end{goals}
	
	--COMMON-CONTENT--
	
	\begin{unit}{\IMPhysicalDatabaseDesign}{}{burleson04,celko05}{10}{b,j}
	\begin{topics}%
		\item ...
		\item ...
		\item ...
	\end{topics}
	\begin{learningoutcomes}
		\item ... [\Usage]
		\item ... [\Usage]
		\item ... [\Usage]
	\end{learningoutcomes}
	\end{unit}
	
	\begin{coursebibliography}
	\bibfile{Computing/CS/CS272}
	\end{coursebibliography}
	
	\end{syllabus}
	