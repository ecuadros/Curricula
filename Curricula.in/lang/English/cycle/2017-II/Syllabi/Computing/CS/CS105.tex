\begin{syllabus}

\course{CS105. Estructuras Discretas I}{Obligatorio}{CS105}

\begin{justification}
Las estructuras discretas son fundamentales para la ciencia de la
computación. Es evidente que las estructuras discretas son usadas en
las áreas de estructura de datos y algoritmos , sin embargo son
también importantes en otras, como por ejemplo en la
verificación, en criptografía y métodos formales.
\end{justification}

\begin{goals}
\item Desarrollar Operaciones asociadas con conjuntos, funciones y relaciones.
\item Relacionar ejemplos prácticos al modelo apropiado de conjunto, función o relación.
\item Conocer las diferentes técnicas de conteo más utilizadas.
\item Describir como las herramientas formales de lógica simbólica son utilizadas.
\item Describir la importancia y limitaciones de la lógica de predicados.
\item Bosquejar la estructura básica y dar ejemplos de cada tipo de prueba descrita en esta unidad.
\item Relacionar las ideas de inducción matemática con la recursividad y con estructuras definidas recursivamente.
\item Enunciar, identificar y habituarse a los conceptos más importantes de Conjuntos Parcialmente Ordenados y Látices
\item Analizar, comentar y aceptar las nociones básicas de Álgebras Booleanas.
\end{goals}

\begin{outcomes}
\ExpandOutcome{a}{3}
\ExpandOutcome{i}{3}
\ExpandOutcome{j}{3}
\end{outcomes}

\begin{unit}{\DSFunctionsRelationsAndSetsDef}{Kolman97,Grassmann97,Johnsonbaugh99}{13}{4}
    \DSFunctionsRelationsAndSetsAllTopics
    \DSFunctionsRelationsAndSetsAllObjectives
\end{unit}

\begin{unit}{\DSBasicLogicDef}{Grassmann97,Iranzo05,Paniagua03,Johnsonbaugh99}{14}{4}
    \DSBasicLogicAllTopics
    \DSBasicLogicAllObjectives
\end{unit}

\begin{unit}{\DSProofTechniquesDef}{Scheinerman01,Brassard97,Kolman97,Johnsonbaugh99}{14}{4}
   \DSProofTechniquesAllTopics
   \DSProofTechniquesAllObjectives
\end{unit}

\begin{unit}{\ARDigitalLogicAndDataRepresentationDef}{Kolman97, Grimaldi97, Gersting87}{19}{3}
\begin{topics}
      \item Conjuntos Parcialmente Ordenados.
      \item Elementos extremos de un conjunto parcialmente ordenado.
      \item Látices.
      \item Álgebras Booleanas.
      \item Funciones Booleanas.
      \item \ARDigitalLogicAndDataRepresentationTopicIntroduction
      \item \ARDigitalLogicAndDataRepresentationTopicLogic
   \end{topics}
   \begin{learningoutcomes}
      \item \DSProofTechniquesObjONE
      \item \DSProofTechniquesObjTWO
      \item \DSProofTechniquesObjTHREE
   \end{learningoutcomes}
\end{unit}



\begin{coursebibliography}
\bibfile{Computing/CS/CS105}
\end{coursebibliography}

\end{syllabus}

%\end{document}
