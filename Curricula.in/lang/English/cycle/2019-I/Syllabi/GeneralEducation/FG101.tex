\begin{syllabus}

\course{FG101A. Comunicación Integral}{Obligatorio}{FG101A} % Common.pm

\begin{justification}
Para lograr una eficaz comunicación en el ámbito personal y profesional, es prioritario el manejo adecuado de la Lengua en forma oral y escrita. Se justifica, por lo tanto, que los alumnos de la Universidad Católica San Pablo conozcan, comprendan y apliquen los aspectos conceptuales y operativos de su idioma, para el desarrollo de sus habilidades comunicativas fundamentales: Escuchar, hablar, leer y escribir.
En consecuencia el ejercicio permanente y el aporte de los fundamentos contribuyen grandemente en la formación académica y, en el futuro, en el desempeño de su profesión

In order to achieve effective communication in the personal and professional field, the proper handling of the Language in oral and written form is a priority. It is therefore justified that the students of UTEC University know, understand and apply the conceptual and operational aspects of their language, for the development of their fundamental communicative skills: Listening, speaking, reading and writing.
Consequently the permanent exercise and the contribution of the fundamentals contribute greatly in the academic formation and, in the future, in the performance of his profession.
\end{justification}

\begin{goals}
\item Desarrollar capacidades comunicativas a través de la teoría y práctica del lenguaje que ayuden al estudiante a superar las exigencias académicas del pregrado y contribuyan a su formación humanística y como persona humana.

\item Develop communicative skills through the theory and practice of language that help the student to overcome the academic requirements of the undergraduate and contribute to his humanistic training and human person.
\end{goals}

\begin{outcomes}{V1}
    \item \ShowOutcome{n}{2}
    \item \ShowOutcome{ñ}{2}
\end{outcomes}

\begin{competences}{V1}
    \item \ShowCompetence{C24}{n,ñ}
\end{competences}

\begin{unit}{}{Primera Unidad}{Real}{16}{C17,C20}
\begin{topics}
      \item La comunicación, definición, relevancia. Elementos. Proceso. Funciones. Clasificación.Comunicación oral y escrita.
      \item El lenguaje: definición. Características y funciones. Lengua: niveles. Sistema. Norma. Habla. El signo lingüístico: definición, características.
      \item Multilingüismo en el Perú. Variaciones dialectales en el Perú.
      \item La palabra: definición, clases y estructura. Los monemas: lexema y morfema. El morfema: clases. La etimología.
      \item El Artículo académico: Definición, estructura, elección del tema, delimitación del tema.

      \item The communication, definition, relevance. Elements. Process. Functions. Classification. Oral and written communication.
      \item The language: definition. Features and functions. Language: levels. System. Rule. Speaks. The linguistic sign: definition, characteristics.
      \item Multilingualism in Peru. Dialect variations in Peru.
      \item The word: definition, classes and structure. The monemas: lexema and morpheme. The morpheme: classes. Etymology.
      \item The Academic Article: Definition, structure, choice of topic, delimitation of the topic.
\end{topics}

\begin{learningoutcomes}
   \item Reconocer y valorar la comunicación como un proceso de comprensión e intercambio de mensajes, diferenciando sus elementos, funciones y clasificación [\Usage].
   \item Analizar las características, funciones y elementos del lenguaje y de la lengua [\Usage].
   \item Identificar las características del multilingüismo en el Perú, valorando su riqueza idiomática [\Usage].
   \item Identificar las cualidades de la palabra y sus clases [\Usage].

   \item Recognize and value communication as a process of understanding and exchanging messages, differentiating its elements, functions and classification[\Usage].
   \item Analyze the characteristics, functions and elements of language and language [\Usage].
   \item Identify the characteristics of multilingualism in Peru, valuing its idiomatic richness [\Usage].
   \item Identify the qualities of the word and its classes [\Usage].
\end{learningoutcomes}
\end{unit}

\begin{unit}{}{Segunda Unidad}{Real, Gatti}{16}{C17, C24}
\begin{topics}
   \item Párrafo: Idea principal, secundaria y global.
   \item El texto: definición, características. Cohesión y coherencia.
   \item Organización del texto: La referencia (deixis); anáfora, catáfora, elipsis. Conectores lógicos y textuales.
   \item Tipos de texto: descriptivo (procesos), expositivo, argumentativo.
   \item Funciones de elocución en el texto: generalización, identificación, nominalización, clasificación,  ejemplificación, definición.
   \item Textos discontinuos: gráficos, tablas y diagramas.
   \item Búsqueda de información. Fuentes de información. Referencias y citas. Registro de información: fichas, notas, resúmenes, etc. Aparato crítico: concepto y finalidad. Normas APA u otro.

   \item Paragraph: Main, secondary and global idea.
   \item The text: definition, characteristics. Cohesion and coherence.
   \item Organization of the text: The reference (dejis); Anaphora, cataphora, ellipsis. Logical and textual connectors.
   \item Types of text: descriptive (processes), expository, argumentative.
   \item Functions of elocution in the text: generalization, identification, nominalization, classification, exemplification, definition.
   \item Discontinuous texts: graphs, tables and diagrams.
   \item Search for information. Information sources. References and citations. Record of information: index cards, notes, summaries, etc. Critical apparatus: concept and purpose. APA Standards or other.
\end{topics}
\begin{learningoutcomes}
   \item Redactar textos expositivos resaltando la idea principal y secundaria [\Usage].
   \item Redactar textos expositivos con adecuada cohesión y coherencia, haciendo uso de referencias y conectores textuales [\Usage].
   \item Interpretar textos discontinuos  valorando su importancia para la comprensión del mensaje [\Usage].

   \item Redactar textos expositivos resaltando la idea principal y secundaria [\Usage].
   \item Redactar textos expositivos con adecuada cohesión y coherencia, haciendo uso de referencias y conectores textuales [\Usage].
   \item Interpretar textos discontinuos  valorando su importancia para la comprensión del mensaje [\Usage].
\end{learningoutcomes}
\end{unit}

\begin{unit}{}{Tercera Unidad}{Lobato}{12}{C17}
\begin{topics}
   \item La oración: definición y clases. La oración enunciativa, interrogativa, imperativa, exclamativa, optativa. La proposición y la frase. La oración simple y compuesta. Coordinación y subordinación. El sintagma: estructura y clases: nominal, verbal, adjetival, preposicional, adverbial.
   \item Elaboración de un glosario de términos técnicos, abreviaturas y siglas relacionadas con la especialidad (actividad permanente a lo largo del semestre).
   \item Redacción del artículo académico: Resumen, palabras clave, introducción, desarrollo, conclusiones, bibliografiaTecnologia (Normas APA u otro que la Escuela profesional requiera).
\end{topics}
\begin{learningoutcomes}
   \item Reconocer y analizar  la estructura oracional valorando su importancia y utilidad en la redacción de textos [\Usage].
   \item Registrar y emplear terminología propia de la especialidad [\Usage].
\end{learningoutcomes}
\end{unit}
y
\begin{unit}{}{Cuarta Unidad}{Vivaldi}{12}{C17, C20, C24}
\begin{topics}
   \item Redacción de correspondencia: carta - solicitud, informe, memorando, hoja de vida.
   \item El discurso oral: propósitos, partes. Escuchar: propósitos y condiciones. Vicios de dicción: barbarismo, solecismo, cacofonía, redundancia, anfibología, monotonía. Régimen preposicional.
   \item Comunicación en grupo Proceso, dinámica, estructura Formas (Técnicas): Mesa redonda,  panel, foro y debate.
   \item Revisión final del artículo académico. Presentación y exposición oral de trabajos de producción intelectual.
\end{topics}
\begin{learningoutcomes}
   \item Redactar textos académicos y funcionales atendiendo los distintos momentos de su producción, su estructura, finalidad y formalidad [\Usage].
   \item Demostrar habilidades como emisor o receptor en distintas situaciones de comunicación con corrección idiomática [\Usage].
   \item Aplicar las diferentes formas (técnicas) de comunicación en grupo reconociendo su importancia para la solución de problemas, toma de decisiones o discusión [\Usage].
\end{learningoutcomes}
\end{unit}

\begin{coursebibliography}
\bibfile{GeneralEducation/FG101}
\end{coursebibliography}

\end{syllabus}
