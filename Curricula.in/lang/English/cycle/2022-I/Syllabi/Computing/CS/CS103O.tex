\begin{syllabus}

\course{CS103O. Algoritmos y Estructuras de Datos}{Obligatorio}{CS103O}

\begin{justification}
El fundamento teórico de todas las ramas de la informática descansa sobre los algoritmos y estructuras de datos, este curso brindará a los participantes una introducción a estos témas, formando así­ una base que servirá para los siguientes cursos en la carrera.
\end{justification}

\begin{goals}
\item Hacer que el alumno entienda la importancia de los algoritmos para la solución de problemas.
\item Introducir al alumno hacia el campo de la aplicación de las estructuras de datos.
\end{goals}

\begin{outcomes}
\ExpandOutcome{a}{4}
\ExpandOutcome{b}{4}
\ExpandOutcome{c}{3}
\ExpandOutcome{d}{3}
\ExpandOutcome{i}{4}
\ExpandOutcome{j}{4}
\ExpandOutcome{k}{3}
\end{outcomes}

\begin{unit}{\PFFundamentalDataStructuresDef}{Cormen2009,Fager2014}{8}{a,b}
   \PFFundamentalDataStructuresAllTopics
   \PFFundamentalDataStructuresAllObjectives
\end{unit}

\begin{unit}{\PFRecursionDef}{Cormen2009,Fager2014}{4}{5}
    \PFRecursionAllTopics
    \PFRecursionAllObjectives
\end{unit}

\begin{unit}{\ALFundamentalAlgorithmsDef}{Cormen2009,Fager2014}{12}{a,b,c}
    \ALFundamentalAlgorithmsAllTopics
    \ALFundamentalAlgorithmsAllObjectives
\end{unit}

\begin{unit}{Grafos}{Cormen2009,Fager2014}{12}{5}
   \begin{topics}
    \item Concepto de Grafos.
    \item Grafos Dirigidos y Grafos no Dirigidos.
    \item Utilización de los Grafos.
    \item Medida de la Eficiencia. En tiempo y espacio.
    \item Matrices de Adyacencia.
    \item Matrices de Adyacencia etiquetada.
    \item Listas de Adyacencia.
    \item Implementación de Grafos usando Matrices de Adyacencia.
    \item Implementación de Grafos usando Listas de Adyacencia.
    \item Inserción, Búsqueda y Eliminación de nodos y aristas.
    \item Algoritmos de búsqueda en grafos.
   \end{topics}
   \begin{learningoutcomes}
      \item  Adquirir destreza para realizar una implementación correcta.
      \item  Desarrollar los conocimientos para decidir cuando es mejor usar una técnica de implementación que otra.
   \end{learningoutcomes}
\end{unit}

\begin{unit}{Matrices Esparzas}{Cormen2009,Fager2014}{8}{a,b,c,j}
   \begin{topics}
    \item  Conceptos  Iniciales.
    \item  Matrices poco densas
    \item  Medida de la Eficiencia en Tiempo  y en Espacio
    \item  Creación de la matriz esparza estática vs Dinámicas.
    \item  Métodos de inserción, búsqueda y eliminación
   \end{topics}

\begin{learningoutcomes}
      \item Comprender el uso y implementacion de matrices esparzas.
   \end{learningoutcomes}
\end{unit}

\begin{unit}{Arboles Equilibrados}{Cormen2009,Fager2014}{16}{a,b,c,j}
   \begin{topics}
        \item Árboles AVL.
	\item Medida de la Eficiencia.
	\item Rotaciones Simples y Compuestas
	\item Inserción, Eliminación y Búsqueda.
	\item Árboles B , B+ B* y Patricia.
   \end{topics}

   \begin{learningoutcomes}
      \item Comprender las funciones básicas de estas estructuras complejas con el fin de adquirir la capacidad para su implementación.
   \end{learningoutcomes}
\end{unit}



\begin{coursebibliography}
\bibfile{Computing/CS/CS103O}
\end{coursebibliography}

\end{syllabus}
