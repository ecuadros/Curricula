\begin{syllabus}

\course{FG211. Professional Ethics}{Obligatorio}{FG211}
% Source file: ../Curricula.in/lang/English/cycle/2021-I/Syllabi/GeneralEducation/FG211.tex

\begin{justification}
Ethics is a constitutive part inherent to the human being, 
and as such it must be reflected in the daily and professional 
actions of the human person. 

It is indispensable that the person assumes an active role in 
society because the economic-industrial, political and social 
systems are not always in function of values and principles, 
being these in reality the pillars on which all the action of 
professionals should be based.
\end{justification}

\begin{goals}
\item That the student broadens his own personal criteria for moral discernment in professional work, so that he not only takes into account the relevant technical criteria but also incorporates moral questions and adheres to correct professional ethics, so that he is capable of making a positive contribution to the economic and social development of the city, region, country and global community.[\Usage]
\end{goals}

--COMMON-CONTENT--

\begin{unit}{}{Moral objectivity}{ACM1992,Schmidt1995,Loza2000,Argandon2006}{12}{C10,C21}
\begin{topics}
	\item Be professional and be moral.
	\item Moral objectivity and the formulation of moral principles.
	\item The professional and his values.
	\item The moral conscience of the person.
	\item The contribution of the DSI to the professional's work.
	\item The common good and the principle of subsidiarity.
	\item Moral principles and private property.
	\item Justice: some basic concepts.
\end{topics}
\begin{learningoutcomes}
	\item To present the student with the importance of having principles and values in today's society.[\Usage]
	\item To present some of the principles that could contribute to society if applied and lived day by day. [\Usage]
	\item To present to the students the contribution of the Social Doctrine of the Church in their professional work. [\Usage]
\end{learningoutcomes}
\end{unit}

\begin{unit}{}{Leadership, Individual Responsibility}{ACM1992,Manzone2007,Schmidt1995,Perez1998,Nieburh2003}{12}{C20,C22}
\begin{topics}
	\item The individual responsibility of the worker in the company.
	\item Leadership and professional ethics in the work environment.
	\item General principles on collaboration in immoral acts.
	\item The professional in the face of bribery: 'victim or collaborator'.

\end{topics}
\begin{learningoutcomes}
	\item To present the student with the role of individual social responsibility and leadership in the company. [\Familiarity]
	\item To know the judgment of ethics in the face of corruption and bribery as a form of work relationship. [\Familiarity]
	\item To present the profession as a form of personal fulfillment, and as a consequence. []
\end{learningoutcomes}
\end{unit}

\begin{unit}{}{Ethics and New Technologies}{ACM1992,IEEE2004,Hernandez2006}{12}{C10,C20,C21}
\begin{topics}
	\item Professional ethics versus general ethics.
	\item Work and profession in the current times.
	\item Ethics, science and technology.
	\item Ethical values in organizations related to the use of information.
	\item Ethical values in the Information Society era.
\end{topics}
\begin{learningoutcomes}
	\item To present the student with the interrelations between ethics and the disciplines of the latest technological era. [\Familiarity]
\end{learningoutcomes}
\end{unit}

\begin{unit}{}{Practical applications}{Comunicaciones2002,Hernandez2006,ACM1992}{12}{C21,C22}
\begin{topics}
    \item Computer ethics.
	\begin{subtopics}
	    \item Ethics and software.
	    \item Free software.
	\end{subtopics}
    \item Telecommunications regulation and ethics.
	\begin{subtopics}
	    \item Internet ethics.
	\end{subtopics}
    \item Copyright and patents.
    \item Ethics in consulting services.
    \item Ethics in technological innovation processes.
    \item Ethics in technology management and technology-based companies.
\end{topics}
\begin{learningoutcomes}
	\item To present the student with some aspects that confront ethics with the work of emerging disciplines in the information society. [\Familiarity]
\end{learningoutcomes}
\end{unit}

\begin{coursebibliography}
\bibfile{GeneralEducation/FG211}
\end{coursebibliography}

\end{syllabus}
