\begin{syllabus}

\course{CS2100. Algorithms and Data Structures}{Obligatorio}{CS2100}
% Source file: ../Curricula.in/lang/English/cycle/2020-I/Syllabi/Computing/CS/CS210.tex

\begin{justification}
The theoretical foundation of all branches of computing rests on algorithms and data structures, this course will provide participants with an introduction to these topics, thus forming a basis that will serve for the following courses in the career.
\end{justification}

\begin{goals}
\item Make the student understand the importance of algorithms for solving problems.
\item Introduce the student to the field of application of data structures.
\end{goals}

\begin{outcomes}{V1}
    \item \ShowOutcome{a}{2}
    \item \ShowOutcome{b}{2}
    \item \ShowOutcome{c}{2}
\end{outcomes}

\begin{outcomes}{V2}
    \item \ShowOutcome{1}{2}
    \item \ShowOutcome{2}{2}
    \item \ShowOutcome{6}{2}
\end{outcomes}

\begin{competences}{V1}
    \item \ShowCompetence{C1}{a}
    \item \ShowCompetence{C2}{b}
    \item \ShowCompetence{C5}{c}
    \item \ShowCompetence{CS2}{b}
\end{competences}

\begin{competences}{V2}
    \item \ShowCompetence{C1}{1} 
    \item \ShowCompetence{C2}{2}
    \item \ShowCompetence{C5}{2}
    \item \ShowCompetence{CS2}{2,6}    
\end{competences}

%%%%%  Sin Usar Macros %%%%%%%
\begin{unit}{Graphs}{}{Cormen2009,Fager2014,Knuth97,Knuth98}{12}{a,b,c}
   \begin{topics}
    \item Graph Concept
    \item Directed Graphs and  Non-directed Graphs.
    \item Using Graphs.
    \item Measurement of efficiency ,in time and space.
    \item Adjacency matrices.
    \item Tag adjacent matrices.
    \item Adjacency Lists.
    \item Implementation of graphs using adjacency matrices.
    \item Graph Implementation using adjacency lists
    \item Insertion, search and deletion of nodes and edges.
    \item Graph search algorithms.
   \end{topics}
   \begin{learningoutcomes}
      \item  Acquire Dexterity to Perform Correct Implementation. [\Usage]
      \item  Develop knowledge to decide when it is better to use one implementation technique than another. [\Usage]   
   \end{learningoutcomes}
\end{unit}

\begin{unit}{Scatter Matrices}{}{Cormen2009,Fager2014,Knuth97,Knuth98}{8}{a,b,c}
   \begin{topics}
    \item  Initial concepts.
    \item  Dense Matrices
    \item  Measurement of Efficiency in Time and Space
    \item  Static scatter vs. dynamic matrix creation.
    \item  Insert, search, and delete methods.
   \end{topics}

\begin{learningoutcomes}
      \item Understand the use and implementation of scatter matrices.[\Assessment]
   \end{learningoutcomes}
\end{unit}

\begin{unit}{Balanced Trees}{}{Cormen2009,Fager2014,Knuth97,Knuth98}{16}{a,b,c}
   \begin{topics}
        \item AVL Trees.
	\item Measurement of Efficiency.
	\item Simple and Composite Rotations
	\item Insertion, deletion and search.
	\item Trees B , B+ B* y Patricia.
   \end{topics}

   \begin{learningoutcomes}
      \item Understand the basic functions of these complex structures in order to acquire the capacity for their implementation. [\Assessment]
   \end{learningoutcomes}
\end{unit}

\begin{coursebibliography}
\bibfile{Computing/CS/CS210}
\end{coursebibliography}

\end{syllabus}
