\begin{syllabus}

\course{CS1D01. Discrete Structures I}{Obligatorio}{CS1D01}
% Source file: ../Curricula.in/lang/English/cycle/2020-I/Syllabi/Computing/CS/CS1D1.tex

\begin{justification}

Discrete structures provide the theoretical foundations necessary for computation. These fundamentals are not only useful to develop computation from a theoretical point of view as it happens 
in the course of computational theory, but also is useful for the practice of computing; In particular in applications such as verification,
cryptography, formal methods, etc.

\end{justification}

\begin{goals}
\item Apply Properly concepts of finite mathematics (sets, relations, functions) to represent data of real problems.
\item Model real situations described in natural language, using propositional logic and predicate logic.
\item Determine the abstract properties of binary relations.
\item Choose the most appropriate demonstration method to determine the veracity of a proposal and construct correct mathematical arguments.
\item Interpret mathematical solutions to a problem and determine their reliability, advantages and disadvantages.
\item Express the operation of a simple electronic circuit using Boolean algebra.
\end{goals}

\begin{outcomes}{V1}
    \item \ShowOutcome{a}{2}
    \item \ShowOutcome{j}{2}
\end{outcomes}

\begin{outcomes}{V2}
    \item \ShowOutcome{1}{2}
    \item \ShowOutcome{6}{2}
\end{outcomes}

\begin{competences}{V1}
    \item \ShowCompetence{C1}{a}
    \item \ShowCompetence{C20}{j}
\end{competences}

\begin{competences}{V2}
    \item \ShowCompetence{C1}{1}
    \item \ShowCompetence{C20}{6}
\end{competences}

    \begin{unit}{\DSSetsRelationsandFunctions}{}{Grimaldi03,Rosen2007,howToProve}{22}{a,j}
    \begin{topics}
        \item \DSSetsRelationsandFunctionsTopicSets
        %\item \DSSetsRelationsandFunctionsTopicRelations
        \item Relations:
            \begin{subtopics}
                \item Reflexivity, simmetry, transitivity
                \item Equivalence relations
                \item Partial order relations and sets
                \item Extremal elements of a partially ordered sets
            \end{subtopics}
        \item \DSSetsRelationsandFunctionsTopicFunctions
    \end{topics}
    \begin{learningoutcomes}
    \item \DSSetsRelationsandFunctionsLOExplainWith [\Assessment]
    \item \DSSetsRelationsandFunctionsLOPerformThe [\Assessment]
    \item \DSSetsRelationsandFunctionsLORelate [\Assessment]
    \end{learningoutcomes}
    \end{unit}

    \begin{unit}{\DSBasicLogic}{}{Rosen2007,Grimaldi03,howToProve}{14}{a,j}
    \begin{topics}
        \item \DSBasicLogicTopicPropositional%
        \item \DSBasicLogicTopicLogical%
        \item \DSBasicLogicTopicTruth%
        \item \DSBasicLogicTopicNormal%
        \item \DSBasicLogicTopicValidity%
        \item \DSBasicLogicTopicPropositionalInference%
        \item \DSBasicLogicTopicPredicate%
        \item \DSBasicLogicTopicLimitations%
    \end{topics}
    \begin{learningoutcomes}
    \item \DSBasicLogicLOConvertLogical [\Usage ]
    \item \DSBasicLogicLOApplyFormal [\Usage ]
    \item \DSBasicLogicLOUseThe [\Usage]
    \item \DSBasicLogicLODescribeHowCan [\Familiarity]
    \item \DSBasicLogicLOApplyFormalAnd [\Usage ]
    \item \DSBasicLogicLODescribeTheLimitationsAnd [\Usage]
    \end{learningoutcomes}
    \end{unit}

\begin{unit}{\DSProofTechniques}{}{Rosen2007,Vel06, Scheinerman12,howToProve}{14}{a,j}
\begin{topics}
        \item \DSProofTechniquesTopicNotions%
        \item \DSProofTechniquesTopicThe%
        \item \DSProofTechniquesTopicDirect%
        \item \DSProofTechniquesTopicDisproving%
        \item \DSProofTechniquesTopicProof%s
        \item \DSProofTechniquesTopicInduction%
        \item \DSProofTechniquesTopicStructural%
        \item \DSProofTechniquesTopicWeak%
        \item \DSProofTechniquesTopicRecursive%
        \item \DSProofTechniquesTopicWell%
\end{topics}

\begin{learningoutcomes}
    %% itemizar cada learning outcomes [nivel segun el curso]
    \item \DSProofTechniquesLOIdentifyTheUsed [\Assessment]
    \item \DSProofTechniquesLOOutline [\Usage ]
    \item \DSProofTechniquesLOApplyEach [\Usage ]
    \item \DSProofTechniquesLODetermineWhich [\Assessment]
    \item \DSProofTechniquesLOExplainTheIdeas [\Familiarity ]
    \item \DSProofTechniquesLOExplainTheWeak [\Assessment]
    \item \DSProofTechniquesLOStateThe [\Familiarity]
\end{learningoutcomes}
\end{unit}

\begin{unit}{Data Representation}{}{Rosen2007,Grimaldi03,howToProve}{10}{a,j}
    \begin{topics}
        \item Numerical representation: sign-magnitude, floating point.
        \item Representation of other objects: sets, relations, functions.
    \end{topics}

    \begin{learningoutcomes}
        \item Explain numerical representations such as sign-magnitude and floating point. [\Assessment].
        \item Carry out arithmetic operations using different kinds of representations. [\Assessment].
        \item Explain the floating point standard IEEE-754 [\Familiarity].   
    \end{learningoutcomes}
\end{unit}

\begin{coursebibliography}
\bibfile{Computing/CS/CS1D1}
\end{coursebibliography}

\end{syllabus}

%\end{document}

