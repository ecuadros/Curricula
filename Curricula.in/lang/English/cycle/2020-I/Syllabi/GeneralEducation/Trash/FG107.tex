\begin{syllabus}

\course{FG107. Fundamentos Antropológicos de la Ciencia}{Obligatorio}{FG107}

\begin{justification}
El conocimiento de la realidad, la orientación de la propia vida y el desempeño profesional requiere del conocimiento de la persona humana y del sentido de la existencia. El curso procurará brindar al alumno una comprensión de la persona humana en su naturaleza, sus relaciones, y su inserción y actuación en la historia.
El curso de Antropología Filosófica y Teológica pertenece al área de Humanidades y se imparte en el tercer semestre del programa profesional de Educación. Para cumplir con sus objetivos, la asignatura tiene un carácter teórico y práctico. La asignatura es teórica en cuanto a que se impartirán contenidos generales que permitan comprender qué es la antropología filosófica y cómo está abierta a la Revelación, pero también adquiere un carácter práctico al evocar experiencias particulares del ser personal y al invitar al alumno a una asimilación vital del conocimiento de sí­ mismo en cuanto persona. Los contenidos que aborda son: La pregunta antropológica como parte de la experiencia común, el hombre como persona, la persona humana como unidad sustancial, manifestaciones del ser personal y antropocentrismo teologal.
\end{justification}

\begin{goals}
\item Comprender, desde una perspectiva que integra razón y fe, los elementos constitutivos de la persona humana y sus manifestaciones, cuestionarse sobre el horizonte de realización que dicha comprensión le abre y desear alcanzarlo, y aplicarlo en la vida desde la libertad de cada uno.[\Usage]
\end{goals}

\begin{outcomes}
    \item \ShowOutcome{g}{1}
    \item \ShowOutcome{n}{2}
    \item \ShowOutcome{o}{2}
\end{outcomes}
\begin{competences}
    \item \ShowCompetence{C10}{g}
    \item \ShowCompetence{C22}{n}
	\item \ShowCompetence{C24}{ñ}
\end{competences}

\begin{unit}{}{Introducción: Primera aproximación al tema}{Garcia2008,Guardini2002,Stein2007,Figari2000,ortega1980,marias1996}{6}{C22,C24}
\begin{topics}
	\item La situación actual del hombre requiere una respuesta verdadera. ``Dimisión de lo humano''. Visiones del hombre desde la modernidad.
	\item La pregunta antropológica como parte de la experiencia común.
	\item El ``hecho humano'' como punto de partida. Lugar del hombre en el mundo.
	\item Una aproximación existencial y situada al tema del hombre.
	\item Antropología ``filosófica y teológica'': la necesidad de integrar fe y razón.
\end{topics}

\begin{learningoutcomes}
	\item Comprender la importancia de conocerse a uno mismo y adquirir elementos para plantear las interrogantes fundamentales de la propia vida. [\Usage].
\end{learningoutcomes}
\end{unit}

\begin{unit}{}{El hombre como persona}{Garcia2008,Melendo2005,Garcia1994,Copleston1999,Verneaux1997,Krapiec1985}{9}{C22}
\begin{topics}
	\item La pregunta por el ``quién soy.
		\subitem Primeras evidencias de nuestro ser personal.
		\subitem Notas carácteristicas de la ``mismida'' personal.
	\item La noción de persona: orígenes históricos (cultura griega y romana).
	\item La noción de persona en el cristianismo
		\subitem El combate a las herejías.
		\subitem Concilio de Calcedonia. 
		\subitem La definición de Boecio.
	\item La definición tomista de persona.
		\subitem La noción metafísica de persona.
	\item De la persona divina a la persona humana.
	\item El personalismo y las aproximaciones más recientes a la persona humana.
	\item La dignidad personal.
	\item Incomunicabilidad y singularidad.
\end{topics}
\begin{learningoutcomes}
	\item Comprender los alcances de la noción de persona: orígenes históricos, fundamentación metafísica. [\Usage]
\end{learningoutcomes}
\end{unit}

\begin{unit}{}{La persona humana como unidad sustancial}{Garcia2008,Rivas2007,ponferrada1970,Stein2007,Verneaux1997,Calkins1990,Copleston1999,Krapiec1985}{9}{C22}
\begin{topics}
	\item Presentación y refutación de posturas contrarias a la unidad sustancial.
	\item Explicaciones dicotómica y tricotómica de la persona humana. 
	\item Explicación hilemórfica: El alma como forma del cuerpo, acto primero.
	\item Los tipos de alma en la psicología clásica.
	\item El alma humana. División de las funciones carácteristicas del alma humana.
	\item La vida vegetativa. La vida sensitiva: conocimiento sensible, apetitos sensibles y pasiones.
	\item La vida intelectiva: la inteligencia humana y la voluntad.
\end{topics}
\begin{learningoutcomes}
	\item Comprender a la persona en su estructura óntica de unidad bio-psico-espiritual.[\Usage]
\end{learningoutcomes}
\end{unit}

\begin{unit}{}{Manifestaciones del ser personal}{Garcia2008,Garcia2000,GarciaCuadrado2000,Figari2002,Mounier1988,ortega1980,Lepp1980,catolica1984,Melendo2005}{12}{C22}
\begin{topics}
	\item El carácter dinámico de la persona.
	\item ``Dinamismos fundamentale''.
	\item Necesidades de seguridad y significación.
	\item La libertad humana.
	\item El aspecto dialogal o relacional de la persona. 
	\item Las relaciones interpersonales. 
	\item El amor interpersonal.
	\item Persona, sexualidad y familia.
	\item Persona, cultura y trabajo.
	\item Las experiencias de finitud, contingencia y trascendencia, nostalgia de infinito.
\end{topics}
\begin{learningoutcomes}
	\item Comprender algunas de las principales manifestaciones del ser personal.[\Usage]
\end{learningoutcomes}
\end{unit}

\begin{unit}{}{Antropocentrismo Teologal}{Figari2002,Figari2000a,CVII2002,Salazar1992,Guardini2002}{9}{C22}
\begin{topics}
 	 \item Qué se entiende por ``antropocentrismo teologal''.
	 \item Necesidad de la Revelación para aclarar el misterio del hombre. 
	 \item Un texto clave: Gaudium et Spes 22.
	 \item Presentación de los principales datos revelados acerca del hombre.
		\subitem Creación.
		\subitem Pecado.
		\subitem Reconciliación en Jesucristo.
\end{topics}
\begin{learningoutcomes}
	\item Reconocer la necesidad de la consideración de los datos provenientes de la revelación cristiana para comprender más plenamente al ser humano. [\Usage]
\end{learningoutcomes}
\end{unit}

\begin{unit}{}{Antropocentrismo Teologal}{Figari2000,Figari2000a,CVII2002,Salazar1992,Guardini2002}{9}{C22}
\begin{topics}
 	 \item Qué se entiende por ``antropocentrismo teologal''.
	 \item Necesidad de la Revelación para aclarar el misterio del hombre. 
		\subitem Un texto clave: Gaudium et Spes 22.
	 \item Presentación de los principales datos revelados acerca del hombre.
		\subitem Creación.
		\subitem Pecado.
		\subitem Reconciliación en Jesucristo.
\end{topics}
\begin{learningoutcomes}
	\item Reconocer la necesidad de la consideración de los datos provenientes de la revelación cristiana para comprender más plenamente al ser humano.[\Usage]
\end{learningoutcomes}
\end{unit}

\begin{unit}{}{La persona y la cultura}{Garcia2010,Garcia2007}{9}{C22}
\begin{topics}
 	 \item Concepto de cultura.
	 \item Técnica.
	 \item Ciencia.
\end{topics}
\begin{learningoutcomes}
	\item Comprensión del ser personal y su despliegue en la cultura.[\Usage]
\end{learningoutcomes}
\end{unit}

\begin{coursebibliography}
\bibfile{GeneralEducation/FG107}
\end{coursebibliography}

\end{syllabus}
