\begin{syllabus}

\course{FG209. Psicología}{Electivos}{FG209}

\begin{justification}
El curso de Psicología corresponde a la Área de Humanidades y la naturaleza del curso es de formación general, en todas las carreras profesionales y es de carácter teórico.
Este curso busca proporcionar una visión del área psicológica nutrida de la dimensión espiritual de manera que, partiendo de una antropología cristrocéntrica, se tome en cuenta a la persona como unidad bio-psico-espiritual. Es esta una de las áreas donde se manifiesta la ruptura de la interioridad del hombre y de su identidad.
Se trata de proporcionar los medios y el espacio adecuado para que el alumno tenga un encuentro consigo mismo. Se orienta al alumno para que desarrolle una mejor comprensión de sí en su forma de pensar, sentir y actuar. Así podrá madurar en su auto-conocimiento, lo cual es un requisito indispensable para realizarse y ser auténtico. Al mismo tiempo, al desplegarse, vive más plenamente la dimensión de encuentro con los otros, con toda la dimensión de comunicación y de poder entender a los demás en su propia realidad.
\end{justification}

\begin{goals}
\item Adquirir e interiorizar una visión cristiana de sí mismo en su dimensión psicológica, que le permita avanzar en su reconciliación personal y descubrir una serie de valores necesarios para ser auténticos. [\Usage]
\end{goals}

\begin{outcomes}
    \item \ShowOutcome{d}{1}
    \item \ShowOutcome{e}{1}
    \item \ShowOutcome{n}{1}
    \item \ShowOutcome{o}{2}
\end{outcomes}

\begin{competences}
    \item \ShowCompetence{C18}{d}
    \item \ShowCompetence{C21}{e}
    \item \ShowCompetence{C22}{n} 
    \item \ShowCompetence{C24}{ñ}
\end{competences}

\begin{unit}{}{Primera Unidad}{Papalia1998,Whittaker1985,Allers1957,Davidorff1994,Verneaux2008,Marietan1996}{21}{C18,C21,C22,C24}
\begin{topics}
	\item Nociones Preliminares.
	      \begin{subtopics}
		\item La psicología y el hombre como unidad bio-psico-espiritual.
		\item La psicología y su importancia.
		\item Métodos de investigación de la psicología.
		\item Aplicación de la antropología en uno mismo y la importancia del conocimiento personal.
	      \end{subtopics}
	\item Funciones Superiores.
	      \begin{subtopics}
		\item Fisiología.
		\item Procesos mentales.
		    \begin{subtopics}
			\item	Inteligencia.
			\item Percepción.
			\item Atención-concentración.
			\item Memoria.
			\item Motivación.
			\item Aprendizaje.
			\item Pensamiento y lenguaje.
		    \end{subtopics}    
	      \end{subtopics}

\end{topics}
\begin{learningoutcomes}
	\item Integrar la visión antropológica del ser humano como unidad bio-psico-espiritual. [\Usage]
	\item Distinguir la supremacía de la dimensión espiritual. [\Familiarity]
	\item Reconocer las relaciones entre las dimensiones.[\Familiarity]
	\item Definir qué es la psicología y sus campos de trabajo. [\Familiarity]
	\item Reconocer la influencia de los dinamismos fundamentales en las diferentes dimensiones de la persona.[\Familiarity]
	\item Explicar en qué consiste los métodos de investigación en la psicología. [\Familiarity]
	\item Reconocer y valorar la utilidad de las herramientas de evaluación en psicología. [\Familiarity]
	\item Reconocer los distintos procesos mentales y su funcionamiento en las funciones psicológicas superiores. [\Familiarity]
\end{learningoutcomes}
\end{unit}

\begin{unit}{}{Segunda Unidad}{Papalia1998,Whittaker1985,Allers1957,Davidorff1994,Verneaux2008,Marietan1996}{12}{C18,C21,C22,C24}
\begin{topics}
	\item Desarollo Humano.
	    \begin{subtopics}
		\item Etapas de Desarrollo Humano.
	    \end{subtopics}
	\item Proceso de Socialización.
	    \begin{subtopics}
		\item Psicología masculina y femenina.
		\item Relaciones Interpersonales.
	    \end{subtopics}
\end{topics}
\begin{learningoutcomes}
	\item Reconocer las importancia de cada una de las etapas del desarrollo de la persona humana. [\Familiarity]
	\item Diferenciar y valorar la psicología masculina y femenina. [\Familiarity]
	\item Conocer e identificar las maneras de socializar con los demás e interiorizar la manera asertiva de relacionarse. [\Familiarity]
\end{learningoutcomes}
\end{unit}

\begin{unit}{}{Tercera Unidad}{Papalia1998,Whittaker1985,Allers1957,Davidorff1994,Verneaux2008,Marietan1996}{12}{C18,C21,C22,C24}
\begin{topics}
  \item Carácter y Personalidad.
    \begin{subtopics}
	\item Herencia y Medio Ambiente.
	\item Voluntad y libertad.
	\item Carácter: concepto y características.
	\item Personalidad: concepto y características.
    \end{subtopics}
  \item Principales Corrientes Psicológicas.
    \begin{subtopics}
	\item Psicología Psicodinámica.
	\item Conductismo.
	\item Psicología Gestalt.
	\item Psicología Humanista.
	\item Psicología Cognitiva.
	\item Psicología Evolutiva.
    \end{subtopics}
\end{topics}
\begin{learningoutcomes}
	\item Identificar las variables externas e internas que influyen en la estructuración de la personalidad. [\Familiarity]
	\item Comprender qué es la personalidad, distinguiéndola del carácter y el temperamento. [\Familiarity]
	\item Conocer e identificar las principales corrientes psicológicas. [\Familiarity]
\end{learningoutcomes}
\end{unit}



\begin{coursebibliography}
\bibfile{GeneralEducation/FG209}
\end{coursebibliography}

\end{syllabus}
