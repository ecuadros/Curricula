\begin{syllabus}

\course{FG3500. Leadership and Performance}{Obligatorio}{FG3500}
% Source file: ../Curricula.in/lang/English/cycle/2019-II/Syllabi/GeneralEducation/FG350.tex

\begin{justification}
En la actualidad las diferentes organizaciones en el mundo exigen a sus integrantes el ejercicio de liderazgo, esto significa asumir los retos asignados con eficacia y afán de servicio, siendo estas exigencias necesarias para la búsqueda de una sociedad más justa y reconciliada. 
Este desafío, pasa por la necesidad de formar a nuestros alumnos con un recto conocimiento de sí mismos, con capacidad de juzgar objetivamente la realidad y de proponer  orientaciones que busquen modificar positivamente el entorno.  \end{justification}

\begin{goals}
\item Desarrollar conocimientos, criterios, capacidades y actitudes para ejercer liderazgo, con el objeto de lograr la eficacia y servicio en los retos asignados, contribuyendo así en la construcción de una mejor sociedad.
\end{goals}

\begin{outcomes}{V1}
    \item \ShowOutcome{d}{2}
    \item \ShowOutcome{f}{2}
    \item \ShowOutcome{o}{2}
\end{outcomes}

\begin{competences}{V1}
    \item \ShowCompetence{C17}{f}
    \item \ShowCompetence{C18}{d}
    \item \ShowCompetence{C24}{ñ}
\end{competences}

\begin{unit}{}{Primera Unidad: Fundamentos del liderazgo}{Cardona, Ferreiro,Dianine,DSouza,Sonnenfeld}{15}{C18,C24}
\begin{topics}
	
	\item Teorías de Liderazgo: 
	\item Definición de Liderazgo.
	\item Fundamentos de Liderazgo.
	\item Visión integral del Ser Humano y Motivos de la acción.
	\item La práctica de la Virtud en el ejercicio de Liderazgo.
\end{topics}
\begin{learningoutcomes}
	\item Analizar y comprender las bases teóricas del ejercicio de Liderazgo.[\Familiarity]
	\item En base a lo comprendido, asumir la actitud correcta para llevarlo a la práctica.[\Familiarity]
	\item Iniciar un proceso de autoconocimiento orientado a descubrir rasgos de liderazgo en sí mismo.[\Familiarity]
\end{learningoutcomes}
\end{unit}

\begin{unit}{}{Segunda Unidad: Formación de virtudes y competencias}{Wilkinson, Huete, Cardona, Chinchilla}{15}{C17,C18,C24}
\begin{topics}
	
	\item Teoría de las Competencias 
	\item Reconocimiento de Competencias
	\item Plan de Desarrollo
	\item Modelos Mentales
	\item Necesidades Emocionales
	\item Perfiles Emocionales
	\item Vicios Motivacionales

\end{topics}
\begin{learningoutcomes}
	\item Conocer y Desarrollar competencias de Liderazgo, centradas en lograr la eficacia, sin dejar de lado el deber de servicio con los demás.[\Familiarity]
	\item Reconocer las tendencias personales y grupales necesarias para el ejercicio de Liderazgo.[\Familiarity]
\end{learningoutcomes}
\end{unit}

\begin{unit}{}{Tercera Unidad: Gestión y dirección de equipos}{Goleman, CardonaP,Hersey, Hunsaker, Hawkins, Ginebra}{18}{C18,C24}
\begin{topics}
	\item La relación personal con el equipo
	\item Liderazgo integral
	\item Acompañamiento y discipulado
	\item Fundamentos de unidad
\end{topics}
\begin{learningoutcomes}
	\item Desarrollar habilidades para el trabajo en equipo[\Familiarity]
\end{learningoutcomes}
\end{unit}

\begin{coursebibliography}
\bibfile{GeneralEducation/FG350}
\end{coursebibliography}

\end{syllabus}
