\begin{syllabus}

\course{FG201. Taller de Arte y Tecnología}{Obligatorio}{FG201}

\begin{justification}
El curso es de naturaleza teórico práctico, tiene como propósito vincular al ser humano con la cultura y sus manifestaciones para apreciarlas y valorarlas.
\end{justification}

\begin{goals}
\item Promover en el estudiante la capacidad de descripción e interpretación crTecnologíatica de la imagen para acrecentar su sensibilidad.
\end{goals}

\begin{outcomes}
    \item \ShowOutcome{ñ}{2}
\end{outcomes}

\begin{competences}
    \item \ShowCompetence{C24}{ñ}
\end{competences}

\begin{unit}{}{Primera Unidad:Arte, elementos y principios del diseño}{Pischel, Bense, Milla, Humberto, Editorial}{9}{C24}
\begin{topics}
	\item El Arte y la Estática: Consideraciones básicas introductorias
	\item Diseño bidimensional: Elementos y principios
	\item Historia del Arte: Consideraciones sobre arte desde la  prehistoria en sus variadas manifestaciones.
\end{topics}
\begin{learningoutcomes}
	\item Exponer los elementos del diseño que intervienen en una obra artTecnologíastica para poder hacer una descripción e interpretación de los mismos. [\Usage].
	\item Exponer las manifestaciones artTecnologíasticas (plásticas) en el transcurso de la historia para vincular estos contenidos con los elementos del diseño. [\Usage].

\end{learningoutcomes}
\end{unit}

\begin{unit}{}{Segunda Unidad: Teoría del color}{Pischel,Milla, Editorial}{9}{C24}
\begin{topics}
	\item El color: TeorTecnologíaas del color y dimensiones del color
	\item Psicología del Color: Consideraciones iniciales
	\item Historia del Arte: Arte en las culturas antiguas.
\end{topics}
\begin{learningoutcomes}
	\item Exponer las teorTecnologíaas del color para una mejor apreciación del mundo que nos rodea. [\Usage].
	\item Desarrollo de la sensibilidad del color. [\Usage].
	\item Exponer las manifestaciones  artTecnologíasticas (plásticas) en el transcurso de la historia para vincular estos contenidos con el conocimiento del color.
 [\Usage].
\end{learningoutcomes}
\end{unit}

\begin{unit}{}{Tercera Unidad: Historia del Arte - Apreciación}{Pischel,Milla, Editorial}{9}{C24}
\begin{topics}
	\item Modelos de análisis y apreciación I: El análisis visual o formal, Consideraciones sobre objeto artTecnologíastico y contexto (influencia).
	\item Historia del Arte: Consideraciones sobre arte hasta el siglo  XVIII.
\end{topics}
\begin{learningoutcomes}
	\item Exponer las manifestaciones  artTecnologíasticas (plásticas) en el transcurso de la historia para distinguir su influencia a través del tiempo.[\Usage].
	\item Formular comentarios, describiendo y analizando, teniendo en cuenta el contexto y los conocimientos adquiridos. [\Usage].
\end{learningoutcomes}
\end{unit}

\begin{unit}{}{Cuarta Unidad: Historia del Arte - Apreciación}{Pischel,Milla, Editorial, Comercio}{9}{C24}
\begin{topics}
	\item Modelos de apreciación II: El análisis visual y su relación con el concepto de tendencia,  El análisis de contexto del objeto artTecnologíastico (variedad)
	\item Historia del Arte: Siglos  XIX , XX, XXI.
\end{topics}

\begin{learningoutcomes}
	\item Exponer las técnicas de comunicación visual, para identificar su aplicación en el lenguaje visual. [\Usage].
	\item Conocer algunas tendencias artTecnologíasticas a través del tiempo en nuestro país, como valioso legado de nuestros antepasados, ubicándolas en su contexto. [\Usage].
\end{learningoutcomes}
\end{unit}

\begin{unit}{}{Quinta Unidad: Arte Peruano - Contexto}{Milla, Comercio}{6}{C24}
\begin{topics}
	\item Visión de la cultura peruana: Las manifestaciones artTecnologíasticas en su contexto.
	\item La apreciación.
\end{topics}
\begin{learningoutcomes}
	\item Relacionar  algunas tendencias artísticas del siglo XIX y XX con las manifestaciones artTecnologíasticas en el Perú, ubicándolas en su contexto. [\Usage].
\end{learningoutcomes}
\end{unit}



\begin{coursebibliography}
\bibfile{GeneralEducation/FG101}
\end{coursebibliography}
\end{syllabus}
