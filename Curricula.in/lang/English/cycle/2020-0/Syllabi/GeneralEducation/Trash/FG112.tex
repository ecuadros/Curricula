\begin{syllabus}

\course{FG112. Matrimonio y Familia}{Electivos}{FG112}

\begin{justification}
El curso de Persona, Matrimonio y Familia corresponde al área de Humanidades, se trata de un curso electivo que abarca todas las carreras profesionales, es de carácter teórico-práctico El propósito fundamental es la comprensión e interiorización  de la familia como  comunidad de personas que tiene su origen en el matrimonio entre un hombre y una mujer; el matrimonio expresa: la comunión de vida que un varón y una mujer establecen entre sí­, libre, públicamente, y para toda la vida, en orden al perfeccionamiento mutuo y a la generación y educación de los hijos , esta entrega recíproca del varón y la mujer crea  un ambiente en el que el niño aprende lo que significa amar y ser amado, y descubre así­ su dignidad personal y la del otro ; esta vida en familia es el fundamento de la sociedad, la persona inicia allí, la vida en sociedad: la autoridad, la estabilidad, y la vida en relación en el seno de la familia, constituyen por otra parte, los fundamentos de la seguridad y fraternidad en el seno de la sociedad.
Las expresiones ``familia reconstruída'', ``familia monoparental'', ``familia disfuncional'' y ``uniones de hecho'' llevan a pensar que vivimos en una cultura que acepta estas situaciones como ``normales'' y hasta ``ideales''. La identidad propia de la familia basada en el matrimonio de un hombre y una mujer se ve afectada y es por ello que urge profundizar en el matrimonio y la familia para descubrir su riqueza y valor y sobre todo su misión como primera línea en la propuesta real de una sociedad de la vida, una comunidad más justa, más reconciliada. Sería errado concluir que la familia es la solución o la medicina a todos los males sociales pero igual de equivocado sería restarle la importancia que se le debe; la sociedad debe comprender su relevancia y ello debe plasmarse en una política pública que la fortalezca y promueva. 
\end{justification}

\begin{goals}
	\item Proponer un conocimiento serio, riguroso e interdisciplinar de la institución familiar de cara a los desafíos que la institución familiar enfrenta.[\Usage]
	\item Comprender que el ser humano ha sido creado por y para el amor, que lo direcciona hacia una unión de las naturalezas (complementariedad) y cuyo fin natural es el matrimonio.[\Familiarity]
	\item Comprender que la familia es el ámbito más íntimo y más propicio para la realización humana y para el desarrollo de los hijos.[\Familiarity]
	\item Reconocer el impacto de la familia en el desarrollo de las sociedades mediante el análisis de investigaciones y estudio bien documentados.[\Familiarity]
\end{goals}

\begin{outcomes}
    \item \ShowOutcome{n}{2}
    \item \ShowOutcome{o}{2}
\end{outcomes}
\begin{competences}
    \item \ShowCompetence{C24}{ñ}
    \item \ShowCompetence{C22}{n}
\end{competences}

\begin{unit}{}{Persona y Matrimonio}{JuanPabloII}{18}{C24}
\begin{topics}
	\item Introducción 
	\item Persona Humana.
		\subitem Esencia y dignidad del hombre.
		\subitem Unidad corporal y espiritual.
		\subitem Estructura de la persona humana .
		\subitem Las facultades superiores.
		\subitem Celibato o Matrimonio.
	\item Matrimonio.
		\subitem Definición.
		\subitem El amor.
		\subitem El amor conyugal y otros.
		\subitem Características del matrimonio .
		\subitem Fines del matrimonio.
\end{topics}

\begin{learningoutcomes}
	\item Comprender que el ser humano ha sido creado por y para el amor, que lo direcciona hacia una unión de las naturalezas (complementariedad) y cuyo fin natural es el matrimonio.[\Familiarity]
\end{learningoutcomes}
\end{unit}

\begin{unit}{}{La Familia: Importancia y Derechos}{Abbott2010,IMHSP1983}{6}{C22,C24}
\begin{topics}
	\item La Familia y sus Derechos.
		\subitem Concepto. 
		\subitem Relaciones familiares. 
		\subitem La familia: Comunidad de personas. 
		\subitem Proceso de cambio social y cultural. 
\end{topics}

\begin{learningoutcomes}
	\item Comprender que la familia es el ámbito más íntimo y más propicio para la realización humana y para el desarrollo de los hijos. [\Familiarity]
\end{learningoutcomes}
\end{unit}

\begin{unit}{}{La Familia: Importancia y Derechos}{Pliego2012,Consejo2004,STI2011,Focus2013,Bradford2009,STI2007}{21}{C22,C24}
\begin{topics}
	\item Familia y Bienestar. 
		\subitem Estudio sobre las diferentes estructuras familiares e indicadores de bienestar.
	\item El dividendo demográfico sostenible.
		\subitem ?`Qué tiene que ver el matrimonio y la familia con la economía? 
	\item Mapa mundial de la familia. 
		\subitem Los cambios en la familia y su impacto en el bienestar de los niños.
	\item El Matrimonio importa. 
		\subitem 26 conclusiones de las ciencias sociales.
	\item Matrimonio y Bien común. 
		\subitem Los 10 principios de Princeton.
	\item Diseños de investigación.
		\subitem Artículos de opinión.
\end{topics}

\begin{learningoutcomes}
	\item Reconocer el impacto de la familia en el desarrollo de las sociedades. A partir del análisis de investigaciones y estudio bien documentados.[\Familiarity]
\end{learningoutcomes}
\end{unit}


\begin{coursebibliography}
\bibfile{GeneralEducation/FG112}
\end{coursebibliography}

\end{syllabus}
