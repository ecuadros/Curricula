\begin{syllabus}

\course{CS410. Pasantía I}{Obligatorio}{CS410}

\begin{justification}
El nivel o evento de pasantia es un espacio de formación práctico-teórico-investigativo, que permitirá al pasante fortalecer y profundizar los conocimientos y operaciones que se desarrollan y se proyectan en la empresa e institución en el campo de la Ciencias de la Computación. El evento se constituye en un aporte cognitivo y práctico de formación profesional. El escenario de actuación del pasante, estará dentro de un marco de responsabilidad académica, coordinado por la carrera profesional y la empresa de convenio.
\end{justification}

\begin{goals}
\item Lograr que los estudiantes fortalezcan sus conocimientos teóricos-prácticos, con la verificación de actividades y acciones de su práctica profesional, en un espacio de observación y apoyo técnico en el desarrollo empresarial e institucional  
\end{goals}

\begin{outcomes}
\ExpandOutcome{f}{3}
\ExpandOutcome{h}{5}
\ExpandOutcome{n}{3}
\end{outcomes}

\begin{unit}{Pasantia I}{Pasantia}{75}{4}
   \begin{topics}
      \item Observación, apoyo y verificación de los procesos prácticos-técnicos que desarrolla la empresa en computación.
   \end{topics}
   
   \begin{learningoutcomes}
      \item Que el alumno tenga una experiencia en el campo laboral que le permita consolidar los conocimientos adquiridos en su carrera.
   \end{learningoutcomes}
\end{unit}

\begin{coursebibliography}
\bibfile{Computing/CS/CS410}
\end{coursebibliography}

\end{syllabus}
