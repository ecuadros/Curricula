\begin{syllabus}

\course{EN0021. Física II}{Obligatorio}{EN0021} % Common.pm

\begin{justification}
Show a high degree of mastery of the laws of wave motion, the nature of fluids, and thermodynamics. Using properly the concepts of wave movement, fluids and thermodynamics in solving problems of daily life. Possess ability and ability in the interpretation of wave, fluid and thermodynamic phenomena, which contribute to the development of efficient and useful solutions in different areas of computer science.
\end{justification}

\begin{goals}
\item  That the student learn and master fluent principles static and moving.
\item  That the student learn and master the principles of MAS, particularly the wave movement.
\item  Thar the student learn and master the principles of thermodynamics.
\item  That the student learn to apply principles of the Physics of fluids, waves and thermodynamics to develop computational models
\end{goals}

\begin{outcomes}{V1}
  \item \ShowOutcome{a}{2}
  \item \ShowOutcome{i}{2}
  \item \ShowOutcome{j}{2}
\end{outcomes}

\begin{competences}{V1}
    \item \ShowCompetence{C1}{a}
    \item \ShowCompetence{C20}{i,j}
\end{competences}

\begin{unit}{FI1. Elasticidad}{}{Sears98,Eisberg98}{4}{C1,C20}
\begin{topics}
         \item  Effort and unit deformation
	 \item  Young's Module
         \item  Poisson Module and Coefficient
	 \item  Stiffness Module
         \item  Module and coefficient of compressibility
   \end{topics}

   \begin{learningoutcomes}
         \item  Understand and characterize the processes of elasticity
         \item  Solve problems
   \end{learningoutcomes}
\end{unit}

\begin{unit}{FI2. Fluidos}{}{Serway98, Tipler98}{8}{C1,C20}
\begin{topics}
         \item  Density and specific gravity
	 \item  Pressure in fluids. Atmospheric pressure and gauge pressure
         \item  Principle of Pascal. Pressure measurement: manometer and barometer
	 \item  Buoyancy and Principle of Archimedes
         \item  Flowing Fluids: Flow and Continuity Equation
	 \item  Bernoulli equation. Applications of the Bernoulli principle: Torricelli's theorem, the ventura tube
         \item  Surface tension and capillarity
   \end{topics}

   \begin{learningoutcomes}
         \item  Explain, analyze and characterize fluid pressure
         \item  Understand, characterize and apply the principle of Archimedes
         \item  Understand, characterize and apply the Bernoulli principle
         \item  Explain, analyze and characterize surface tension and capillarity
   \end{learningoutcomes}
\end{unit}

\begin{unit}{FI3. Movimiento Periódico}{}{Sears98, Serway98}{8}{C1,C20}
\begin{topics}
         \item  Introduction .Elastic modulus of a Spring
	 \item  Simple harmonic motion. Energy in simple harmonic oscillator
         \item  Reference circle: the period and the sinusoidal nature of simple harmonic motion
	 \item  Simple pendulum.
         \item  Cushioned harmonic motion.
         \item  Forced oscillations: resonance.
   \end{topics}

   \begin{learningoutcomes}
         \item  Explain, analyze and characterize the oscillatory movement from the MAS.
         \item  Solve problems.
   \end{learningoutcomes}
\end{unit}

\begin{unit}{FI4. Ondas}{}{Eisberg98,Resnick98,Douglas84}{8}{C1,C20}
\begin{topics}
         \item  Wave motion. Types of waves. One-dimensional traveling waves
	 \item  Wave Overlay and Interference
         \item  Velocity of the waves in a tight rope. Reflection and transmission of waves
	 \item  Sine waves. Energy transmitted by sinusoidal waves in strings
         \item  Stationary waves on a rope. Sound waves. Speed of sound waves
	 \item  Periodic sound waves. Intensity of periodic sound waves
	 \item  Sources of sound: vibratory strings and vibrating air columns
	 \item  Doppler Effect
   \end{topics}

   \begin{learningoutcomes}
         \item  Explain, find and characterized  through problems of the daily life the undulatory movement, as well as, the reflection and transmission of waves in the space.
         \item  Solve problems
   \end{learningoutcomes}
\end{unit}

\begin{unit}{FI5. Temperatura y Teoría Cinética}{}{Eisberg98,Resnick98}{12}{C1,C20}
\begin{topics}
         \item  Atoms. Temperature. Thermometers and temperature scales
	 \item  Thermal expansion of solids and liquids. Coefficients of linear, surface and cubic expansion
         \item  Laws of gases and absolute temperature. The ideal gas law in molecular terms: Avogadro's number
	 \item  Kinetic theory and molecular interpretation of temperature. Distribution of molecular velocities
         \item  Isothermal and adiabatic processes for an ideal gas. The equipartition of energy
	 \item  Termodinámica. Tipos de sistemas que estudia la Termodinámica
         \item  Zero Law of Thermodynamics
	 \item  The constant-volume gas thermometer and the Kelvin scale
         \item  Punto triple del agua
   \end{topics}

   \begin{learningoutcomes}
         \item  Explain, analyze and characterize the concept of Temperature and the thermal expansion of solids and liquids
         \item  Understanding the ideal gas law and the isothermal and adiabatic processes for an ideal gas
         \item  Understand the zero law of thermodynamics
         \item  Solve problems
   \end{learningoutcomes}
\end{unit}

\begin{unit}{FI6. Calor y primera Ley de la Termodinámica}{}{Eisberg98,Resnick98}{8}{C1,C20}
\begin{topics}
         \item  Heat as energy transfer
	 \item  Heat capacity and specific heat
         \item  Internal energy of an ideal gas
	 \item  Specific heat of an ideal gas
         \item  Phase changes. Latent heat of fusion and vaporization
	 \item  Calorimetry. Work and heat in thermodynamic processes
         \item  The first law of thermodynamics
	 \item  Some applications of the first law of thermodynamics
         \item  Transmission of heat by conduction, convection and radiation
   \end{topics}

   \begin{learningoutcomes}
         \item  Understand the concept of heat and internal energy of an ideal gas
         \item  Explain, analyze and characterize the first law of thermodynamics
         \item  Solve problems
   \end{learningoutcomes}
\end{unit}

\begin{unit}{FI7. Máquinas térmicas, entropía y la segunda ley de la Termodinámica}{}{Eisberg98,Resnick98}{8}{C1,C20}
\begin{topics}
         \item  Thermal Machines and the Second Law of Thermodynamics
	 \item  Reversible and irreversible processes. The Carnot Machine
         \item  Absolute temperature range.Chillers 
	 \item  Entropy. Entropy changes in irreversible processes
   \end{topics}

   \begin{learningoutcomes}
         \item  Explain, analyze and characterize the first law of thermodynamics
         \item  Explain, analyze and characterize the Carnot machine
         \item  Solve problems
   \end{learningoutcomes}
\end{unit}

\begin{coursebibliography}
\bibfile{BasicSciences/CF141}
\end{coursebibliography}

\end{syllabus}

%\end{document}
