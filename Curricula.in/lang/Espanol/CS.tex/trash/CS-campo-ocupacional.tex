\section{Campo y mercado ocupacional}\label{sec:campo-ocupacional}
Nuestro egresado podrá prestar sus servicios profesionales en empresas e instituciones públicas y privadas que requieran sus capacidades en función del desarrollo que oferta, entre ellas:

\begin{itemize}
\item Empresas dedicadas a la producción de software con calidad internacional.
\item Empresas, instituciones y organizaciones que requieran software de calidad para mejorar sus actividades y/o servicios ofertados.
\end{itemize}

Nuestro egresado puede desempeñarse en el mercado laboral sin ningún problema ya que, en general, la exigencia del mercado y campo ocupacional está mucho más orientada al uso de herramientas. Sin embargo, es poco común que los propios profesionales de esta carrera se pregunten: ?`que tipo de formación debería tener si yo quisiera crear esas herramientas además de saber usarlas?. Ambos perfiles (usuario y creador) son bastante diferentes pues no sería posible usar algo que todavía no fue creado. En otras palabras, los creadores de tecnología son los que \underline{dan origen a nuevos puestos de trabajo} y abren la posibilidad de que otros puedan usar esa tecnología.

Debido a la formación basada en la investigación, nuestro profesional debe siempre ser un innovador donde trabaje. Esta misma formación permite que el egresado piense también en crear su propia empresa de desarrollo de software. Considerando que países como el nuestro tienen un costo de vida mucho menor que Norte América ó Europa, una posibilidad que se muestra interesante es la exportación de software pero eso requiere que la calidad del producto sea al mismo nivel de lo ofrecido a nivel internacional.

Este perfil profesional también posibilita que nuestros egresados se queden en nuestro país; producir software en nuestro país y venderlo fuera es más rentable que salir al extranjero y comercializarlo allá.

El campo ocupacional de un egresado es amplio y está en continua expansión y cambio. Prácticamente toda empresa u organización hace uso de servicios de computación de algún tipo, y la buena formación básica de nuestros egresados hace que puedan responder a los requerimientos de las mismas exitosamente. Este egresado, no sólo podrá dar soluciones a los problemas existentes sino que deberá proponer innovaciones tecnológicas que impulsen la empresa hacia un progreso constante.

A medida que la informatización básica de las empresas del país avanza, la necesidad de personas capacitadas para resolver los problemas de mayor complejidad aumenta y el plan de estudios que hemos desarrollado tiene como objetivo satisfacer esta demanda considerandola a mediano y largo plazo. El campo para las tareas de investigación y desarrollo de problemas complejos en computación es también muy amplio y está creciendo día a día a nivel mundial.

Debido a la capacidad innovadora de nuestro egresado, existe una mayor la probabilidad de registrar patentes con un alto nivel inventivo lo cual es especialmente importante en nuestros países.
