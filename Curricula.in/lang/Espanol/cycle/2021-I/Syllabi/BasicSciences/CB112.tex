\begin{syllabus}

   \course{CB111. Física Computacional}{Obligatorio}{CB111}
   % Source file: ../Curricula.in/lang/Espanol/cycle/2021-I/Syllabi/BasicSciences/CB111.tex
   
   \begin{justification}
   Física I es un curso que le permitirá al estudiante entender
   las leyes de física de macropartículas y micropartículas considerado desde un
   punto material hasta un sistemas de partículas; debiéndose tener en cuenta que los
   fenómenos aquí estudiados se relacionan a la física clásica: Cinemática, Dinámica, Trabajo y Energía; 
   además se debe asociar que éstos problemas deben ser resueltos con algoritmos computacionales.
   
   Poseer capacidad y habilidad en la interpretación de problemas clásicos
   con condiciones de frontera reales que contribuyen en la elaboración de soluciones eficientes
   y factibles en diferentes áreas de la Ciencia de la Computación.
   \end{justification}
   
   \begin{goals}
   \item Conocer los principios básicos de los fenómenos que gobiernan la física clásica.
   \item Aplicar los principios básicos a situaciones específicas y poder asociarlos con situaciones reales.
   \item Analizar algunos de los fenómenos físicos así como su aplicación a situaciones reales.
   \end{goals}
   
   \begin{outcomes}{V1}
     \item \ShowOutcome{a}{1}
     \item \ShowOutcome{i}{1}
     \item \ShowOutcome{j}{2}
   \end{outcomes}
   
   \begin{competences}{V1}
     \item \ShowCompetence{C1}{a} 
     \item \ShowCompetence{C20}{i}
     \item \ShowCompetence{CS2}{j}
   \end{competences}
   
   \begin{unit}{Vectores}{}{Burbano,ResnikHalliday,SerwayJewett,TriplerMosca}{6}{C1}
   \begin{topics}
         \item Análisis dimensional.
         \item Vectores. Propiedades. Operaciones.
         \item Caso práctico: Estimación de fuerzas.
      \end{topics}
   
      \begin{learningoutcomes}
         \item Entender y trabajar con las magnitudes físicas del SI.[\Usage]
         \item Abstraer de la naturaleza los conceptos físicos rigurosos y representarlos en modelos vectoriales.[\Usage]
         \item Entender y aplicar los conceptos vectoriales a problemas físicos reales.[\Usage]
      \end{learningoutcomes}
   \end{unit}
   
   \begin{unit}{}{Estática}{Burbano,ResnikHalliday,SerwayJewett,TriplerMosca}{6}{C20}
   \begin{topics}
         \item Primera y tercera Ley de Newton.
         \item Diagrama de cuerpo libre.
         \item Primera condición de equilibrio.
         \item Caso práctico: Estimación de la fuerza humana.
         \item Segunda condición de equilibrio.
         \item Torque.
         \item Casos prácticos: Aplicaciones en dispositivos mecánicos.
         \item Fricción.
       \end{topics}
      \begin{learningoutcomes}
         \item Conocer los conceptos que rigen la primera Ley y tercera Ley de Newton.
         \item Conocer y aplicar los conceptos de la primera y segunda condición de equilibro.
         \item Capacidad para resolver problemas de casos prácticos.
         \item Entender el concepto de fricción y resolver problemas.
      \end{learningoutcomes}
   \end{unit}
   
   \begin{unit}{}{Cinemática}{Burbano,ResnikHalliday,SerwayJewett,TriplerMosca}{6}{C24}
   \begin{topics}
         \item Posición, Velocidad, Aceleración.
         \item Gráficas de movimiento.
         \item Casos prácticos: Representación gráfica de movimiento utilizando Excel.
         \item Movimiento circular.
         \item Velocidad angular y velocidad tangencial.
         \item Mecanismos rotativos.
         \item Caso práctico: Operación de la caja de cambios de un automóvil.
      \end{topics}
   
      \begin{learningoutcomes}
         \item Poder determinar la posición, velocidad y aceleración de un cuerpo.
         \item Conocer el concepto de composición de movimientos y saberlo aplicar, en la descripción de un movimiento circular.
         \item Conocer el significado de las componentes tangencial y normal de la aceleración y saberlas calcular en un instante determinado. 
         \item Utilizar excel para el procesamiento de datos experimentales.
      \end{learningoutcomes}
   \end{unit}
   
   \begin{unit}{}{Dinámica}{Burbano,ResnikHalliday,SerwayJewett,TriplerMosca}{6}{C1}
   \begin{topics}
         \item Segunda Ley de Newton.
         \item Fuerza y movimiento.
         \item Momento de inercia.
      \end{topics}
   
      \begin{learningoutcomes}
         \item Aplicar las leyes de Newton en la solución de problemas.
         \item Describir las diversas interacciones por sus correspondientes fuerzas.
         \item Determinar el momento de inercia de un cuerpo usando un método dinámico
      \end{learningoutcomes}
   \end{unit}
   
   \begin{unit}{}{Trabajo mecánico}{Burbano,ResnikHalliday,SerwayJewett,TriplerMosca}{6}{C20}
   \begin{topics}
         \item Trabajo.
         \item Fuerzas constantes.
         \item Fuerzas variables.
         \item Potencia.
         \item Caso práctico: Estimación de la potencia de una planta hidroeléctrica.
     \end{topics}
   
      \begin{learningoutcomes}
         \item Comprender el concepto de Trabajo.
         \item Comprender y aplicar el concepto de Potencia a la resolución de problemas.
         \item Resolver problemas.
      \end{learningoutcomes}
   \end{unit}
   
   \begin{unit}{}{Energía}{Burbano,ResnikHalliday,SerwayJewett,TriplerMosca}{6}{C24}
   \begin{topics}
         \item Tipos de energía.
         \item Conservación de la energía. 
         \item Dinámica de un sistema de partículas.
         \item Colisiones.
     \end{topics}
   
      \begin{learningoutcomes}
         \item Conocer los tipos de energía que existen.
         \item Aplicar el principio de conservación de la energía mecánica a distintas situaciones, diferenciando aquellas en las que la energía total no se mantiene constante. 
         \item Aplicar los principios de conservación del momento lineal y de la energía a un sistema aislado de dos o más partículas interactuantes.
      \end{learningoutcomes}
   \end{unit}
   
   \begin{coursebibliography}
   \bibfile{BasicSciences/CB111}
   \end{coursebibliography}
   
   \end{syllabus}
   