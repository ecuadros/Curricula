\begin{syllabus}

\course{CB111. Física Computacional}{Obligatorio}{CB111}
% Source file: ../Curricula.in/lang/Espanol/cycle/2021-I/Syllabi/BasicSciences/CB111.tex

\begin{justification}
   El curso desarrolla los conocimientos y capacidades para reconocer, evaluar y aplicar los efectos de los fenómenos físicos relacionados a la mecánica en el campo de la ingeniería. En la industria en general, el control de los procesos, el funcionamiento de las máquinas, su mantenimiento, etc., siempre están regidas por algún tipo de manifestación física. Debido a eso, es importante para el estudiante entender los fundamentos de los fenómenos físicos, las leyes que los rigen, su manifestación y la forma de detectarlos. El presente curso permitirá al estudiante comprender e identificar los fenómenos físicos relacionados a la mecánica con el fin de que puedan controlar sus efectos sobre algún proceso técnico. 
\end{justification}

\begin{goals}
\item Capacidad de aplicar los conocimientos de ciencias.
\item Capacidad de diseñar y llevar a cabo experimentos.
\item Capacidad de aplicar conocimientos de computación y de matemáticas.
\item Capacidad de desarrollar principios de investigación con nivel internacional.
\end{goals}

\begin{outcomes}{V1}
  \item \ShowOutcome{a}{1}
  \item \ShowOutcome{i}{1}
  \item \ShowOutcome{j}{2}
\end{outcomes}

\begin{competences}{V1}
  \item \ShowCompetence{C1}{a} 
  \item \ShowCompetence{C20}{i}
  \item \ShowCompetence{CS2}{j}
\end{competences}

\begin{unit}{Trabajo, Energía y Potencia}{}{YoFreed,Hewitt}{6}{C1}
\begin{topics}
      \item Definición de trabajo y la relación entre trabajo neto y energía cinética.
      \item Potencia y Eficiencia
   \end{topics}

   \begin{learningoutcomes}
      \item Determinar las variables que afectan la oposición a la traslación y la oposición a la rotación (momento de inercia) y calcular la energía cinética de traslación y rotación.
      \item Calcular el trabajo de una fuerza, aplicar el Teorema de trabajo neto y energía sobre un sistema de la vida real, y determinar la potencia y eficiencia.
   \end{learningoutcomes}
\end{unit}

\begin{unit}{}{Cinemática}{YoFreed,Hewitt}{6}{C20}
\begin{topics}
      \item Sistemas de referencia espacial y temporal.
      \item Velocidad media, aceleración media, lineal y angular.
      \item Vectores de posición, velocidad y aceleración, lineal y angular.
      \item Relación entre cinemática lineal y angular.
    \end{topics}
   \begin{learningoutcomes}
      \item Entender los conceptos de cinemática sistema de referencia espacial y temporal, y trayectoria y determinar la posición, velocidad y aceleración lineal y angular, según un contexto físico o de gráficos.
      \item Descomponer la aceleración lineal, en función a un sistema de coordenadas, para poder describir la posición y en aceleración radial y tangencial.
      \item Determina la posición, velocidad y aceleración, usando el cálculo diferencial e integral. 
   \end{learningoutcomes}
\end{unit}

\begin{unit}{Las tres leyes de Newton}{Las tres leyes de Newton}{YoFreed,Hewitt}{6}{C24}
\begin{topics}
      \item Torque de una fuerza.
      \item Las 3 leyes de Newton y su aplicación en partículas y sólidos rígidos.
   \end{topics}

   \begin{learningoutcomes}
      \item Plantear las ecuaciones de rotación y traslación para un sólido y aplicar las leyes de Newton. 
      \item Analizar las características de la fuerza de fricción. Calcular la fuerza radial neta y la fuerza centrípeta neta.
      \item Calcular el centro de masa y analizar la relación entre las variables de fuerza neta, tiempo y cambio de velocidad. 
   \end{learningoutcomes}
\end{unit}

\begin{coursebibliography}
\bibfile{BasicSciences/CB111}
\end{coursebibliography}

\end{syllabus}
