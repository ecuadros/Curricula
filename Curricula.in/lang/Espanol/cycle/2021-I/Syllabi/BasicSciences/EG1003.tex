\begin{syllabus}

\course{EG0003. Matemática I}{Obligatorio}{EG0003}
% Source file: ../Curricula.in/lang/Espanol/cycle/2020-I/Syllabi/BasicSciences/EG1003.tex

\begin{justification}
El curso tiene como objetivo desarrollar en los estudiantes las habilidades para manejar modelos en ciencia e ingeniería relacionados con habilidades de cálculo diferencial simple. En el curso se estudian y aplican conceptos relacionados con el cálculo de Límites, derivados e integrales de funciones reales y vectoriales de variables reales únicas que se utilizarán como base y
apoyo al estudio de nuevos contenidos y materias. También busca lograr capacidades de razonamiento y aplicabilidad para interactuar con problemas del mundo real proporcionando una base matemática para actividades de desarrollo.
\end{justification}

\begin{goals}
\item Aplicar los conceptos de números complejos y funciones para resolver problemas relacionados con la ciencia.
\item Aplicar conceptos matemáticos y técnicas de cálculo diferencial de una variable para resolver situaciones problemáticas de la ciencia
\item Calcular las expresiones matemáticas de las integrales indefinidas con exactitud, orden y claridad en el tratamiento de los datos.
\end{goals}

\begin{outcomes}{V1}
    \item \ShowOutcome{a}{3}
    \item \ShowOutcome{j}{3}
\end{outcomes}

\begin{competences}{V1}
    \item \ShowCompetence{C1}{a}
    \item \ShowCompetence{C20}{j}
    \item \ShowCompetence{C24}{j}
\end{competences}

\begin{unit}{Números complejos}{}{Stewart,RonLarson}{20}{C1}
   \begin{topics}
    \item Operaciones con números complejos
    \item Teorema de Moivre 
   \end{topics}

   \begin{learningoutcomes}
      \item  Definir y operar con números complejos, calculando su forma polar y exponencial [\Assessment].
      \item  Utilizar el teorema de Moivre para simplificar los cálculos de complejos[\Assessment].
      \end{learningoutcomes}
\end{unit}

\begin{unit}{Funciones de una sola variable}{}{Stewart,RonLarson}{10}{C20}

  
  \begin{topics}
    \item Dominio y rango.
    \item Tipos de funciones.
    \item Gráfico de exponenciales y funciones logarítmicas.
    \item Funciones trigonométricas.
    \item Aplicar reglas para transformar funciones.
    \item  Problemas de aplicaciones usando Excel, modelando crecimiento bacteriano, escala logarítmica, etc.
   \end{topics}

   \begin{learningoutcomes}
      \item Definir una función de una sola variable y entender y ser capaz de determinar su dominio y rango. [\Assessment].
      \item Reconocer diferentes tipos específicos de funciones y crear diagramas de dispersión y seleccionar un modelo apropiado. [\Assessment].
      \item Comprender cómo un cambio en la base afecta a la gráfica de exponenciales y funciones logarítmicas. [\Assessment].
      \item Reconoce y construye funciones trigonométricas. [\Assessment].
      \item Aplicar reglas para transformar funciones. [\Assessment].
      \item Ser capaz de resolver problemas de aplicaciones simples como regresión y ajuste de curvas. [\Assessment].
    \end{learningoutcomes}
\end{unit}

\begin{unit}{Límites y derivadas}{}{Stewart,RonLarson}{20}{C1}
   \begin{topics}
      \item Límites
      \item Derivadas
      \item Conceptos sobre Derivadas y calcular errores relativos.
      \item El Teorema de L'Hospital
      \item Problemas de aplicaciones tales como velocidad, crecimiento exponencial y decaimiento, acumulación de grava creciente, optimización de una lata, etc.
   \end{topics}

   \begin{learningoutcomes}
      \item Entender el concepto de límites y calcular los límites de la gráfica de una función. [\Assessment].
      \item Encontrar límites usando las leyes de límites y la simplificación algebraica. [\Assessment].
      \item Encontrar asíntotas verticales y horizontales. [\Assessment].
      \item Calcular y estimar derivados. [\Assessment].
      \item Interpretar la derivada como una tasa de cambio. [\Assessment].
      \item Encontrar los derivados de la función básica y compuesta [\Assessment].
      \item Aproximación de  funciones usando conceptos de  derivadas y calculo de  errores relativos [\Assessment].
      \item Encontrar los números críticos , los valores máximos y mínimos absolutos y locales para la función continua. [\Assessment].
      \item Aplicar Teorema de L'Hospital para calcular algunos límites. [\Assessment].
      \item Entender los problemas de optimización, encontrar la función a ser optimizada y resolver[\Assessment].
      \item Ser capaz de resolver problemas de aplicaciones simples.. [\Assessment].
        \end{learningoutcomes}
\end{unit}

\begin{unit}{Integrales}{}{Stewart,RonLarson}{22}{C20}
   \begin{topics}
    \item Estrategia para la integración.
    \item Tecnica para integrar funciones.
    \item Herramientas adicionales para encontrar integrales
    \item Problemas de aplicaciones.

   \end{topics}

   \begin{learningoutcomes}
      \item Resolver correctamente el área de estimación usando los rectángulos izquierdo y derecho del punto final y del punto medio.[\Assessment].
      \item Utilizar el teorema fundamental para encontrar derivados de funciones de evaluar integrales definidas e indefinidas mediante sustitución.[\Assessment].
      \item Utilizar diferentes técnicas para integrar funciones. [\Assessment].
      \item Aplicar integrales a las áreas encontradas.[\Assessment].
      \item Calcular volúmenes de sólidos obtenidos girando una región limitada alrededor del eje x o del eje y. [\Assessment].
      \item Calcular el volumen de sólidos obtenidos al girar una región limitada alrededor del eje x o del eje y, considerando cascarones cilíndricos.[\Assessment].
      \item Calcula el valor promedio de una función. [\Assessment].
      \item Calcular el trabajo realizado por una fuerza y calcule el centro de masa para una placa plana en el plano.[\Assessment].
      \item Definir curvas paramétricas y funciones vectoriales encontrando relaciones entre ellas. [\Assessment].
      \item Aplicar integrales para calcular la longitud de las curvas descritas por las funciones vectoriales.[\Assessment].
      \item Ser capaz de resolver problemas de aplicaciones simples tales como tráfico en un servicio de Internet, consumo de combustible, tomografía: volumen del cerebro, bomba de agua, masa en espesante, superformula, volumen en máquina de Wankel, longitud de hélice de molécula de ADN, etc.[\Assessment].
    \end{learningoutcomes}
\end{unit}

\begin{coursebibliography}
\bibfile{BasicSciences/EG1003}
\end{coursebibliography}

\end{syllabus}
