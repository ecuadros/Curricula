\begin{syllabus}

\course{. }{}{} % Common.pm

\begin{justification}
El egresado de la Universidad San Pablo, no sólo deberá ser un excelente profesional, conocedor de la más avanzada tecnología, sino también, un ser humano sensible y de amplia cultura. En esta perspectiva, el curso proporciona los instrumentos conceptuales básicos para una óptima comprensión de las obras musicales como producto cultural y artístico creado por el hombre.
\end{justification}

\begin{goals}
\item Analizar de manera crítica las diferentes manifestaciones artísticas a través de la historia identificando su naturaleza expresiva, compositiva y características estéticas así como las nuevas tendencias artísticas identificando su relación directa con los actuales indicadores socioculturales. Demostrar conducta sensible, crítica, creativa y asertiva, y conductas valorativas como indicadores de un elevado desarrollo personal.
\end{goals}

\begin{outcomes}
    \item \ShowOutcome{n}{2}
    \item \ShowOutcome{o}{2}
\end{outcomes}
\begin{competences}
    \item \ShowCompetence{C24}{n,ñ}
\end{competences}

\begin{unit}{}{Primera Unidad}{Aopland,Salvat,Hamel}{9}{C24}
\begin{topics}
	\item La música en la vida del hombre. Concepto. El Sonido: cualidades.
	\item Los elementos de la música. Actividades y audiciones.
\end{topics}
\begin{learningoutcomes}
	\item Dotar al alumno de un lenguaje musical básico, que le permita apreciar y emitir un juicio con propiedad [\Usage].
\end{learningoutcomes}
\end{unit}

\begin{unit}{}{Segunda Unidad}{Salvat,Hamel}{9}{C24}
\begin{topics}
	\item La voz, el canto y sus intérpretes. Práctica de canto.
	\item Los instrumentos musicales. El conjunto instrumental.
	\item El estilo, género y las formas musicales. Actividades y audiciones.
\end{topics}
\begin{learningoutcomes}
	\item Que el alumno conozca, discrimine y aprecie los elementos que integran la obra de arte musical [\Usage].
\end{learningoutcomes}
\end{unit}

\begin{unit}{}{Tercera Unidad}{Donald,Hamel}{15}{C24}
\begin{topics}
	\item El origen de la música - fuentes. La música en la antigüedad.
	\item La música medieval: Música religiosa.  Canto Gregoriano. Música profana.
	\item El Renacimiento: Música instrumental y música vocal.
	\item El Barroco y sus representantes. Nuevos instrumentos, nuevas formas.
	\item El Clasicismo. Las formas clásicas y sus más destacados representantes.
	\item El Romanticismo y el Nacionalismo, características generales instrumentos y formas. Las escuelas nacionalistas europeas.
	\item La música contemporánea: Impresionismo, Postromanticismo, Expresionismo y las nuevas corrientes de vanguardia.
\end{topics}
\begin{learningoutcomes}
	\item Que el alumno conozca y distinga con precisión los diferentes momentos del desarrollo musical [\Usage].
	\item Dotar al alumno de un repertorio mínimo que le permita poner en práctica lo aprendido antes de emitir una apreciación crítica de ella [\Usage].
\end{learningoutcomes}
\end{unit}

\begin{unit}{}{Cuarta Unidad}{Donald,Bellenger,Carpio,Aretz}{12}{C24}
\begin{topics}
	\item Principales corrientes musicales del Siglo XX.
	\item La música peruana: Autóctona, Mestiza, Manifestaciones musicales actuales.
	\item Música arequipeña, principales expresiones.
	\item Música latinoamericana y sus principales manifestaciones.
\end{topics}
\begin{learningoutcomes}
	\item Que el alumno conozca e identifique las diferentes manifestaciones populares actuales [\Usage].
	\item Que el alumno Se identifique con sus raíces musicales [\Usage].
\end{learningoutcomes}
\end{unit}



\begin{coursebibliography}
\bibfile{GeneralEducation/FG105}
\end{coursebibliography}

\end{syllabus}
