\begin{syllabus}

\course{GH1012. Perú,temas de la sociedad contemporánea}{Obligatorio}{GH1012}
% Source file: ../Curricula.in/lang/Espanol/cycle/2021-I/Syllabi/GeneralEducation/GH1012.tex

\begin{justification}
Se propone discutir y analizar, desde una perspectiva interdisciplinaria (geográfica, histórica y antropológica), la realidad social peruana y el proceso histórico que devino en esta, desarrollando en el estudiante un sentido de contexto histórico-social, conciencia social y pensamiento crítico. 
\end{justification}

\begin{goals}
\item Con. 
\end{goals}

\begin{outcomes}{V1}
     \item \ShowOutcome{e}{2}
     \item \ShowOutcome{f}{2}
     \item \ShowOutcome{i}{2}
     \item \ShowOutcome{n}{2}
\end{outcomes}

\begin{competences}{V1}
    \item \ShowCompetence{C17}{f,h,n}
    \item \ShowCompetence{C20}{f,n}
    \item \ShowCompetence{C24}{f,h}
\end{competences}

\begin{unit}{Laboratorio de Comunicación I}{}{Cassany93}{12}{4}
   \begin{topics}
      \item Características 
   \end{topics}
   \begin{learningoutcomes}
      \item Auto-evaluación: el estudiante es capaz de reconocer sus propias fortalezas y deficiencias al formular críticas constructivas sobre su propio trabajo.
   \end{learningoutcomes}
\end{unit}

\begin{coursebibliography}
\bibfile{GeneralEducation/GH1012}
\end{coursebibliography}

\end{syllabus}
