\begin{syllabus}

\course{FG2018. Dibujo, comic e ilustración digital}{Electivo}{FG2018}
% Source file: ../Curricula.in/lang/Espanol/cycle/2020-I/Syllabi/GeneralEducation/FG2018.tex

\begin{justification}
Este curso es de carácter práctico. En él se revisarán las técnicas de ilustración e ilustración digital más influyente en las últimas décadas. Se indagarán en los trabajos de ilustradores de cómic y manga más reconocidos del medio, poniendo énfasis en los motivos y técnicas.  Al final, los estudiantes desarrollarán un producto creativo ilustrado en técnica mixta.

\end{justification}

\begin{goals}
\item Capacidad de interpretar información.
\end{goals}

\begin{outcomes}{V1}
    \item \ShowOutcome{d}{2}
    \item \ShowOutcome{e}{2}
    \item \ShowOutcome{n}{2}
    
\end{outcomes}

\begin{competences}{V1}
    \item \ShowCompetence{C10}{d,n}
    \item \ShowCompetence{C17}{d}
    \item \ShowCompetence{C18}{n}
    \item \ShowCompetence{C21}{e}
\end{competences}

\begin{unit}{Culturas de Gobernanza y Distribución de Poder}{}{Lessig15}{12}{4}
   \begin{topics}
      \item ?`Cómo se relaciona la economía con la política?.
      \item El rol de las Instituciones.
      \item Análisis de casos.
   \end{topics}
   \begin{learningoutcomes}
      \item Desarrollo del innterés por conocer sobre temas actuales en la sociedad peruana y el mundo.
   \end{learningoutcomes}
\end{unit}

\begin{coursebibliography}
\bibfile{GeneralEducation/FG2018}
\end{coursebibliography}

\end{syllabus}
