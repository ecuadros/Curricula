\begin{syllabus}

\begin{justification}
Provee una serie de herramientas para resolver problemas que son difíciles de solucionar con los métodos algorítmicos tradicionales. Incluyendo heurísticas, planeamiento, formalismos en la representación del conocimiento y del razonamiento, técnicas de aprendizaje en máquinas, técnicas aplicables a los problemas de acción y reacción: asi como el aprendizaje de lenguaje natural, visión artificial y robótica entre otros. 
\end{justification}

\begin{goals}
\item Realizar algún curso avanzado de Inteligencia Artificial sugerido por el curriculo de la ACM/IEEE.
\end{goals}

--COMMON-CONTENT--

\begin{unit}{}{Levantamiento del estado del arte}{Russell03,Haykin99,Goldberg89}{60}{a,h}
\begin{topics}
  \item Sistemas Inteligentes.
  \item Razonamiento automatizado.
  \item Sistemas Basados en Conocimiento.
  \item Aprendizaje de Maquina. \cite{Russell03},\cite{Haykin99}
  \item Sistemas de Planeamiento.
  \item Procesamiento de Lenguaje Natural.
  \item Agentes.
  \item Robótica.
  \item Computación Simbólica.
  \item Algoritmos Genéticos. \cite{Goldberg89}
\end{topics}
\begin{learningoutcomes}
  \item Profundizar en diversas técnicas relacionadas a la Inteligencia Artificial. [\Usage]
\end{learningoutcomes}
\end{unit}

\begin{coursebibliography}
\bibfile{Computing/CS/CS361}
\end{coursebibliography}

\end{syllabus}
