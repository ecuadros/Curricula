\begin{syllabus}

	\course{CS272. Bases de Datos II}{Obligatorio}{CS272}
	% Source file: ../Curricula.in/lang/Espanol/cycle/2021-I/Syllabi/Computing/CS/CS272.tex
	
	\begin{justification}
	La Gestión de la Información (IM-\textit{Information Management}) juega un rol principal en casi todas las áreas donde los computadores son usados. Esta área incluye la captura, digitalización, representación, organización, transformación y presentación de información; algorítmos para mejorar la eficiencia y efectividad del acceso y actualización de información almacenada, modelamiento de datos y abstracción, y técnicas de almacenamiento de archivos físicos.
	
	Este también abarca la seguridad de la información, privacidad, integridad y protección en un ambiente compartido. Los estudiantes necesitan ser capaces de desarrollar modelos de datos conceptuales y físicos, determinar que métodos de IM y técnicas son apropiados para un problema dado, y ser capaces de seleccionar e implementar una apropiada solución de IM que refleje todas las restricciones aplicables, incluyendo escalabilidad y usabilidad.
	\end{justification}
	
	\begin{goals}
	\item Hacer que el alumno entienda las diferentes aplicaciones que tienen las bases de datos, en las diversas áreas de conocimiento.
	\item Mostrar las formas adecuadas de almacenamiento de información basada en sus diversos enfoques y su posterior recuperación de información.
	\end{goals}
	
	--COMMON-CONTENT--
	
	\begin{unit}{\IMPhysicalDatabaseDesign}{}{burleson04,celko05}{10}{b,j}
	\begin{topics}%
		\item ...
		\item ...
		\item ...
	\end{topics}
	\begin{learningoutcomes}
		\item ... [\Usage]
		\item ... [\Usage]
		\item ... [\Usage]
	\end{learningoutcomes}
	\end{unit}
	
	\begin{coursebibliography}
	\bibfile{Computing/CS/CS272}
	\end{coursebibliography}
	
	\end{syllabus}
	
	%\end{document}
	