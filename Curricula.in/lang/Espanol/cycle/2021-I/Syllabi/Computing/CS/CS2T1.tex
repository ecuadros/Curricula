\begin{syllabus}

	\course{CB309. Bioinformática}{Obligatorio}{CB309}
	% Source file: ../Curricula.in/lang/Espanol/cycle/2021-I/Syllabi/BasicSciences/CB309.tex
	
	\begin{justification}
	El uso de métodos computacionales en las ciencias biológicas se ha convertido en una de las herramientas claves para el campo de la biología molecular, siendo parte fundamental en las investigaciones de esta área. 
	
	En Biología Molecular, existen diversas aplicaciones que involucran tanto al ADN, al análisis de proteínas o al secuenciamiento del genoma humano, que dependen de métodos computacionales. Muchos de estos problemas son realmente complejos y tratan con grandes conjuntos de datos. 
	
	Este curso puede ser aprovechado para ver casos de uso concretos de varias áreas de conocimiento de Ciencia de la Computacion como: Lenguajes de Programación (PL), Algoritmos y Complejidad (AL), Probabilidades y Estadística, Manejo de Información (IM), Sistemas Inteligentes (IS).
	\end{justification}
	
	\begin{goals}
	\item Que el alumno tenga un conocimiento sólido de los problemas biológicos moleculares que desafían a la computación.
	\item Que el alumno sea capaz de abstraer la esencia de los diversos problemas biológicos para plantear soluciones usando sus conocimientos de Ciencia de la Computación
	\end{goals}
	
	--COMMON-CONTENT--
	
	\begin{unit}{Introducción a la Biología Molecular}{}{Clote2000,Setubal1997}{4}{CS1}
	\begin{topics}
			\item ... 
			\item ...
			\item ...
	\end{topics}
	\begin{learningoutcomes}
			\item ... [\Familiarity]
			\item ... [\Assessment]
	\end{learningoutcomes}
	\end{unit}
	
	\begin{coursebibliography}
	\bibfile{BasicSciences/CB309}
	\end{coursebibliography}
	
	\end{syllabus}
	