\begin{syllabus}

\course{ID101. Inglés I}{Obligatorio}{ID101}
% Source file: ../Curricula.in/lang/Espanol/cycle/2021-I/Syllabi/ForeignLanguages/ID101.tex

\begin{justification}
El aprendizaje del idioma ingles en los estudiantes de nivel superior se ha 
convertido en elemento necesario para su desarrollo académico, personal y 
profesional. 

La mayor cantidad de literatura académica de los diversos campos del 
saber es redactada en inglés, lo cual garantiza que las personas que cuenten 
con el dominio del idioma puedan siempre estar actualizados. 

Asimismo, el conocimiento de este idioma permite tener perspectivas 
sociales y culturales más amplias. En ese sentido, para un efectivo 
aprendizaje de la lengua es necesario el desarrollo de 
las cuatro (4) habilidades; escuchar, hablar, leer y escribir 
considerando los lineamientos del 
Marco Común Europeo de Referencia de Lengua - MCERL.
\end{justification}

\begin{goals}
\item Comunicarse e intercambiar información de manera limitada en situaciones predecibles de forma general.
\item Alcanzar el nivel B1 según el MCERL.
\item Manejar terminología propia del área de estudios.
\end{goals}

--COMMON-CONTENT--

\begin{unit}{Hello everybody}{}{de2002marco}{12}{C25}
   \begin{topics}
      \item Greetings and farewells.
      \item To be (affirmative, question and negatives).
      \item Connector (addition – contrast).
   \end{topics}

   \begin{learningoutcomes}
      \item Greeting people
      \item Introduce oneself and others
   \end{learningoutcomes}
\end{unit}

\begin{unit}{Meeting people}{}{de2002marco}{12}{C25}
   \begin{topics}
      \item To be (affirmative, question and negatives)
      \item Questions (what / where / when– to be)
      \item Possessive adjectives
      \item Connector (addition – contrast)
   \end{topics}

   \begin{learningoutcomes}
      \item Exchange basic personal information  
   \end{learningoutcomes}

\end{unit}

\begin{unit}{What are you doing?}{}{de2002marco}{12}{C25}
   \begin{topics}
      \item Present continuous (affirmative, question and negatives).
      \item Questions (what / where / when).
      \item Connector (addition – contrast).
   \end{topics}

   \begin{learningoutcomes}
      \item Talk about ongoing activities.
   \end{learningoutcomes}

\end{unit}

\begin{unit}{My clothes}{}{de2002marco}{12}{C25}
   \begin{topics}
      \item Personal pronouns.
      \item Possessive pronouns.
      \item Common adjectives.
      \item Demostrative adjectives.
      \item Connector (addition – contrast).
   \end{topics}

   \begin{learningoutcomes}
      \item Describe clothes
      \item Ask and answer about possession
   \end{learningoutcomes}

\end{unit}

\begin{unit}{Smaller or Bigger?}{}{de2002marco}{12}{C25}
   \begin{topics}
      \item Comparatives.
      \item Intensifiers - very basic.
      \item Superlatives.
      \item Connector (addition – contrast).
   \end{topics}

   \begin{learningoutcomes}
      \item Compare clothing items
   \end{learningoutcomes}
\end{unit}

\begin{unit}{Can you speak English?}{}{de2002marco}{15}{C25}
   \begin{topics}
      \item Design by the teacher
      \item Connector (addition – contrast).
   \end{topics}

   \begin{learningoutcomes}
      \item Get familiar with technical expressions of the field.
   \end{learningoutcomes}
\end{unit}

\begin{coursebibliography}
\bibfile{ForeignLanguages/ID101}
\end{coursebibliography}
\end{syllabus}
%\end{document}
