\begin{syllabus}

\course{ID101. Inglés I}{Obligatorio}{ID101}
% Source file: ../Curricula.in/lang/Espanol/cycle/2021-I/Syllabi/ForeignLanguages/ID101.tex

\begin{justification}
Parte fundamental de la formación integral de un profesional es la habilidad de comunicarse en un idioma extranjero además del propio idioma nativo. No solamente amplía su horizonte cultural sino que permite una visión más humana y comprensiva de la vida de las personas. En el caso de los idiomas extranjeros, indudablemente el Inglés es el más prátcico porque es hablado alrededor de todo el mundo. No hay país alguno donde éste no sea hablado. En las carreras relacionadas con los servicios al turista el Inglés es tal vez la herramienta práctica más importante que el alumno debe dominar desde el primer momento, como parte de su formación integral.
\end{justification}

\begin{goals}
\item Conocer el idioma Inglés y su estructura gramatical.
\item Identificar situaciones y emplear diálogos relacionados a ellas.
\end{goals}

\begin{outcomes}{V1}
\item \ShowOutcome{f}{2}
\end{outcomes}

\begin{competences}{V1}
    \item \ShowCompetence{C25}{f}
\end{competences}

\begin{unit}{Hello everybody}{}{Soars022S,Cambridge06,MacGrew99}{0}{C25}
   \begin{topics}
      \item Verbo To Be.
      \item Oraciones Afirmativas, Negativas y Preguntas.
      \item Expresiones Numéricas.
      \item Objetos y Paí­ses.
      \item Expresiones para saludar y hacer presentaciones.
   \end{topics}

   \begin{learningoutcomes}
      \item Al terminar la primera unidad, cada uno de los alumnos, comprendiendo la gramática del tiempo presente es capaz de expresar una mayor cantidad de expresiones de tiempo y además usar oraciones con el verbo To Be para expresar situación y estado. 
      \item Que el alumno sea capaz de analizar y expresar ideas acerca de fechas y números en orden. 
   \end{learningoutcomes}
\end{unit}

\begin{unit}{Meeting people}{}{Soars022S,Cambridge06,MacGrew99}{0}{C25}
   \begin{topics}
      \item Adjetivos Posesivos.
      \item Expresiones para averiguar precios.
      \item Expresiones de Posesión.
      \item Vocabulario de Familia, Comidas y Bebidas.
      \item Pedidos formales.
      \item Cartas informales.
   \end{topics}

   \begin{learningoutcomes}
      \item Al terminar la segunda unidad, los alumnos habiendo identificado la forma de expresar pedidos y hacer ofrecimientos en restaurantes los utilizan en situaciones varias. Explica y aplica vocabulario de comidas y bebidas.  
   \end{learningoutcomes}

\end{unit}

\begin{unit}{The world of work}{}{Soars022S,Cambridge06,MacGrew99}{0}{C25}
   \begin{topics}
      \item Tiempo Presente Simple. Auxiliares.
      \item Oraciones Afirmativas, Negativas y Preguntas.
      \item Verbos comunes y Ocupaciones.
      \item Indicaciones para expresar la hora.
   \end{topics}

   \begin{learningoutcomes}
      \item Al terminar la tercera unidad, los alumnos habiendo reconocido las caracterTecnologíasticas  del presente simple, lo utiliza para hacer descripciones de diversos tipos. Describen personas y lugares y dan indicaciones de dirección. Expresa la hora. 
   \end{learningoutcomes}

\end{unit}

\begin{unit}{Take it easy}{}{Soars022S,Cambridge06,MacGrew99}{0}{C25}
   \begin{topics}
      \item Presente Simple 2.
      \item Oraciones Afirmativas, Negativas y Preguntas.
      \item Uso de Verbos de entretenimiento.
      \item Tiempo Libre.
      \item Las estaciones del año.
      \item Expresiones de actividades sociales.
   \end{topics}

   \begin{learningoutcomes}
      \item Al terminar la cuarta unidad, los alumnos habiendo identificado la idea de expresar ideas de acciones de tiempo libre en Presente Simple y Continuo. Expresan ideas de estaciones y actividades relacionadas.
   \end{learningoutcomes}

\end{unit}

\begin{unit}{Where do you live ?}{}{Soars022S,Cambridge06,MacGrew99}{0}{C25}
   \begin{topics}
      \item Uso There is/There are.
      \item Oraciones con Preposiciones.
      \item Expresiones de Cantidad.
      \item Vocabulario de aviones y lugares.
      \item Expresiones de indicaciones de dirección.
   \end{topics}

   \begin{learningoutcomes}
      \item Al finalizar la quinta unidad, los alumnos, a partir de la comprensión del tiempo presente continuo, elaborararán oraciones utilizando ideas de ubicación y de lugar. Asimilarán además la necesidad de expresar objetos de uso común. Adquirirán vocabulario para describir las partes de una casa usan expresiones para pedir indicaciones de dirección.
   \end{learningoutcomes}
\end{unit}

\begin{unit}{Can you speak English?}{}{Soars022S,Cambridge06,MacGrew99}{0}{C25}
   \begin{topics}
      \item Can/cant.
      \item Pasado del verbo To Be. Uso de Could.
      \item Vocabulario de países e idiomas.
      \item Expresiones para el uso del teléfono.
      \item Redacción de cartas formales.
      \item Lecturas.
   \end{topics}

   \begin{learningoutcomes}
      \item Al finalizar la sexta unidad, los alumnos habiendo conocido los fundamentos del uso de auxiliares de modo, crearan oraciones aplicadas al contexto adecuado. Enfatizan la diferencia entre idiomas y nacionalidades. Describen sentimientos. Utilizan expresiones en el teléfono.
   \end{learningoutcomes}
\end{unit}

\begin{unit}{Then and now}{}{Soars022S,Cambridge06,MacGrew99}{0}{C25}
   \begin{topics}
      \item Pasado Simple.
      \item Expresiones de tiempo pasado.
      \item Vocabulario verbos regulares e irregulares.
      \item Expresiones para describir el clima. 
      \item Redacción de párrafos descriptivos.
      \item Ocasiones Especiales.
   \end{topics}

   \begin{learningoutcomes}
      \item Al finalizar la sétima unidad, los alumnos habiendo conocido los fundamentos de la estructuración del Pasado Simple experimentan la necesidad de poder expresar este tipo de tiempo en acciones. Realizarán prácticas en contextos adecuados. Enfatizan la diferencia entre verbos irregulares y regulares. Describen acciones con verbos varios. Utilizan expresiones para describir el clima.
   \end{learningoutcomes}
\end{unit}

\begin{coursebibliography}
\bibfile{ForeignLanguages/ID101}
\end{coursebibliography}
\end{syllabus}
%\end{document}
