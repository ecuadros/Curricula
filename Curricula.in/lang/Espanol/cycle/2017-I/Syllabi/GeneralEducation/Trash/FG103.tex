\begin{syllabus}

\course{FG103. Introducción a la Vida Universitaria}{Obligatorio}{FG103}

\begin{justification}
El ingreso a la universidad es un momento de nuevos desafíos y decisiones en la vida de una persona. En ese sentido, la Universidad Católica San Pablo busca, mediante el presente espacio, escuchar y acoger al joven ingresante con sus inquietudes y anhelos personales, presentar la identidad y misión de la universidad como su alma mater”, señalando los principales desafíos que el futuro profesional enfrentará en el mundo actual  y orientando a nuestros jóvenes estudiantes, a través de diversos principios, medios y otros recursos, con el fin de que puedan formarse integralmente y desplegarse plenamente en la fascinante aventura de la vida universitaria.  Su realización como buen profesional depende de una buena formación personal y cultural que le brinde horizontes amplios, que sustenten y proyecten su conocimiento y quehacer técnicos e intelectuales y que le permitan contribuir siendo agentes de cambio cultural y social.
El curso corresponde al área de Humanidades, tiene carácter analítico, teórico y práctico, pues propone al alumno descubrir  la verdad de sí mismo y de su entorno para así responder a los retos de la nueva etapa que emprende en la universidad. Para ello ha de tener también un panorama claro de lo que es la universidad, de sus fines y objetivos para insertar su proyecto de vida en ella. Abarca los siguientes aspectos:   descubrimiento de su experiencia existencial y el sentido de su vida, reconocimiento de la realidad del mundo y así, definir su misión como persona y  su rol como universitario.
\end{justification}

\begin{goals}
\item Que el alumno canalice sus inquietudes y anhelos a través del encuentro y descubrimiento de sí mismo, que le brinden espacios de análisis y reflexión personales para asumir posturas bien fundamentadas hacia los valores e ideales de su entorno. Mediante su inserción en la vida universitaria, logrará una disposición de apertura a su propio mundo interior y a su misión en el mundo, cuestionando su cosmovisión y a sí mismo para obtener un conocimiento y crecimiento personales que permitan su despliegue integral y profesional. [\Familiarity]
\end{goals}

\begin{outcomes}
    \item \ShowOutcome{n}{2}
    \item \ShowOutcome{o}{2}
\end{outcomes}
\begin{competences}
    \item \ShowCompetence{C22}{n}
    \item \ShowCompetence{C24}{ñ}
\end{competences}

\begin{unit}{}{La experiencia existencial y el descubrimiento del sentido de la vida}{Sanz2008,Frankl2004,Rilke1941,Marias1995}{24}{C22,C24}
\begin{topics}
	\item Introducción al curso: presentación y dinámicas.
	\item Sentido de la Vida, búsqueda de propósito y vocación profesional.
	\item Obstáculos para el autoconocimiento: el ruido, la falta de comunicación, la mentira existencial, máscaras.
	\item Ofertas Intramundanas.
		\subitem Hedonismo.
		\subitem Relativismo.
		\subitem Consumismo.
		\subitem Individualismo.
		\subitem Inmanentismo.
	\item Las consecuencias.
		\subitem La falta de interioridad.
		\subitem Masificación y el desarraigo.
		\subitem Soledad. 
	\item Los vicios capitales como plasmación en lo personal
\end{topics}
\begin{learningoutcomes}
	\item Identificar el anhelo de sentido que pervive en toda persona humana.[\Familiarity]
	\item Identificar y caracterizar la propia cosmovisión y los criterios personales predominantes en sí mismos acerca del propósito y sentido de la vida y la felicidad.[\Familiarity]
\end{learningoutcomes}
\end{unit}

\begin{unit}{}{La Visión sobre la Persona Humana}{Guardini1994,Fromm1959,Vergara1991,Figari2002,Pieper2007}{15}{C22,C24}
\begin{topics}
	\item ?`Quién soy? Las preguntas fundamentales.
	\item El hombre como unidad.
	\item El hombre: nostalgia de infinito.
	\item La libertad como elemento fundamental en las elecciones personales: la experiencia del mal.
	\item Análisis del Amor y la Amistad. 
	\item Aceptación y Reconciliación personal.
	\item Llamados a ser personas: la vivencia de la virtud según un modelo concreto.
\end{topics}
\begin{learningoutcomes}
	\item Reconocer la importancia de iniciar un proceso de autoconocimiento.[\Familiarity]
	\item Identificar las manifestaciones que evidencian la unidad de la persona humana y su anhelo de trascendencia.[\Familiarity]
	\item Contrastar los modelos de amor y libertad ofertados por la cultura actual con los propuestos en el curso.[\Familiarity]
	\item Distinguir los criterios que conducen a una recta valoración personal.[\Familiarity]
\end{learningoutcomes}
\end{unit}

\begin{unit}{}{Vida universitaria y horizontes de misión}{PabloII2012,Garcia2000,Guardini2012}{9}{C22,C24}
\begin{topics}
	\item Origen y propósito de la Universidad: breve reseña histórica.
	\item La identidad católica de la UCSP.
		\subitem Comunidad académica.
		\subitem Búsqueda de la verdad.
		\subitem La formación integral.
		\subitem Evangelización de la cultura.
	\item Proyecto Final
\end{topics}
\begin{learningoutcomes}
	\item Conocer e identificar  a la UCSP dentro del contexto histórico de las universidades.[\Familiarity]
	\item Reconocer a su universidad como un ámbito de despliegue y espacio para crear cultura.[\Familiarity]
	\item Afirmar, desde su vocación profesional, la necesidad de transformar el mundo que le toca vivir.[\Familiarity]
\end{learningoutcomes}
\end{unit}

\begin{coursebibliography}
\bibfile{GeneralEducation/FG103}
\end{coursebibliography}

\end{syllabus}
