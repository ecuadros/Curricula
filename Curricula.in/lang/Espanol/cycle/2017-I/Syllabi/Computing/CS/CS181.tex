\begin{syllabus}

\curso{CS181. Historia de la Computación}{Obligatorio}{CS181}

\begin{justification}
Este curso está orientado aque el alumno conozca los origenes de la tecnología y lo vincule al origen del área de la Ciencia de la Computación. Conocer el origen permitirá planear el futuro de esta área con mayor claridad.
\end{justification}

\begin{goals}
\item Que el alumno conozca los origenes y la historia de la tecnología.
\item Mencionar los eventos históricos más importantes que llevaron a la invención de la computadora.
\item Mencionar algunos de los mecanismos antiguos de la computación y sus inventores.
\item Definir el término computadora y esbozar algunas carácteristicas.
\end{goals}

\begin{unit}{Hitos de la Historia de la Computación}{Engines84,TheDreamMachine91,IEEEJournal}{0}
\begin{topics}
      \item 350 Million Years BC The first tetrapods leave the oceans
	\item 30,000 BC to 20,000 BC Carving notches into bones
	\item 8500 BC Bone carved with prime numbers discovered
	\item 1900 BC to 1800 BC The first place-value number system
	\item 1000 BC to 500 BC The invention of the abacus
	\item 383 BC to 322 BC Aristotle and the Tree of Porphyry
	\item 300 BC to 600 AD The first use of zero and negative numbers
	\item 1274 AD Ramon Lull's Ars Magna
	\item 1285 AD to 1349 AD William of Ockham's logical transformations
	\item 1434 AD The first self-striking water clock
	\item 1500 AD Leonardo da Vinci's mechanical calculator
	\item 1600 AD John Napier and Napier's Bones
	\item 1621 AD The invention of the slide rule
	\item 1625 AD Wilhelm Schickard's mechanical calculator
	\item 1640 AD Blaise Pascal's Arithmetic Machine
	\item 1658 AD Pascal creates a scandle
	\item 1670 AD Gottfried von Leibniz's Step Reckoner
	\item 1714 AD The first English typewriter patent
	\item 1726 AD Jonathan Swift writes Gulliver's Travels
	\item 1735 AD The Billiard cue is invented
	\item 1761 AD Leonhard Euler's geometric system for problems in class logic
	\item 1800 AD Jacquard's punched cards
	\item Circa 1800 AD Charles Stanhope invents the Stanhope Demonstrator
	\item 1822 AD Charles Babbage's Difference Engine
	\item 1829 AD Sir Charles Wheatstone invents the accordion
	\item 1829 AD The first American typewriter patent
	\item 1830 AD Charles Babbage's Analytical Engine
	\item 1834 AD Georg and Edward Scheutz's Difference Engine
	\item 1834 AD Tally sticks: The hidden dangers
	\item 1837 AD Samuel Morse invents the electric telegraph
	\item 1847 AD to 1854 AD George Boole invents Boolean Algebra
	\item 1857 AD Sir Charles Wheatstone uses paper tape to store data
	\item 1860 AD Sir Joseph Wilson Swan's first experimental light bulb
	\item 1865 AD Lewis Carroll writes Alice's Adventures in Wonderland
	\item 1867 AD The first commercial typewriter
	\item 1868 AD First skeletons of Cro-Magnon man discovered
	\item 1869 AD William Stanley Jevons invents the Jevons' Logic Machine
	\item 1872 AD Lewis Carroll writes Through the Looking-Glass
	\item Circa 1874 AD The Sholes keyboard
	\item 1876 AD Lewis Carroll writes The Hunting of the Snark
	\item 1876 AD George Barnard Grant's Difference Engine
	\item 1878 AD The first true incandescent light bulb
	\item 1878 AD The first shift-key typewriter
	\item 1879 AD Robert Harley publishes article on the Stanhope Demonstrator
	\item 1880 AD The invention of the Baudot Code
	\item 1881 AD Allan Marquand's rectangular logic diagrams
	\item 1881 AD Allan Marquand invents the Marquand Logic Machine
	\item 1883 AD Thomas Alva Edison discovers the Edison Effect
	\item 1886 AD Lewis Carroll writes The game of Logic
	\item 1886 AD Charles Pierce links Boolean algebra to circuits based on switches
	\item 1890 AD John Venn invents Venn Diagrams
	\item 1890 AD Herman Hollerith's tabulating machines
	\item Circa 1900 AD John Ambrose Fleming invents the vacuum tube
	\item 1902 AD The first teleprinters
	\item 1906 AD Lee de Forest invents the Triode
	\item 1921 AD Karel Capek's R.U.R. (Rossum's Universal Robots)
	\item 1926 AD First patent for a semiconductor transistor
	\item 1927 AD Vannevar Bush's Differential Analyser
	\item Circa 1936 AD The Dvorak keyboard
	\item 1936 AD Benjamin Burack constructs the first electrical logic machine
	\item 1937 AD George Robert Stibitz's Complex Number Calculator
	\item 1937 AD Alan Turing invents the Turing Machine
	\item 1938 AD Claude Shannon's master's Thesis
	\item 1939 AD John Vincent Atanasoff's special-purpose electronic digital computer
	\item 1939 AD to 1944 AD Howard Aiken's Harvard Mark I (the IBM ASCC)
	\item 1940 AD The first example of remote computing
	\item 1941 AD Konrad Zuse and his Z1, Z3, and Z4
	\item 1943 AD Alan Turing and COLOSSUS
	\item 1943 AD to 1946 AD The first general-purpose electronic computer -- ENIAC
	\item 1944 AD to 1952 AD The first stored program computer -- EDVAC
	\item 1945 AD The "first" computer bug
	\item 1945 AD Johann (John) Von Neumann writes the "First Draft"
	\item 1947 AD First point-contact transistor
	\item 1948 AD to 1951 AD The first commercial computer -- UNIVAC
	\item 1949 AD EDSAC performs it's first calculation
	\item 1949 AD The first assembler -- "Initial Orders"
	\item Circa 1950 AD Maurice Karnaugh invents Karnaugh Maps
	\item 1950 AD First bipolar junction transistor
	\item 1952 AD G.W.A. Dummer conceives integrated circuits
	\item 1953 AD First TV Dinner
	\item 1957 AD IBM 610 Auto-Point Computer
	\item 1958 AD First integrated circuit
	\item 1962 AD First field-effect transistor
	\item 1962 AD The "worst" computer bug
	\item 1963 AD MIT's LINC Computer
	\item 1970 AD First static and dynamic RAMs
	\item 1971 AD CTC's Datapoint 2200 Computer
	\item 1971 AD The Kenbak-1 Computer
	\item 1971 AD The first microprocessor: the 4004
	\item 1972 AD The 8008 microprocessor
	\item 1973 AD The Xerox Alto Computer
	\item 1973 AD The Micral microcomputer
	\item 1973 AD The Scelbi-8H microcomputer
	\item 1974 AD The 8080 microprocessor
	\item 1974 AD The 6800 microprocessor
	\item 1974 AD The Mark-8 microcomputer
	\item 1975 AD The 6502 microprocessor
	\item 1975 AD The Altair 8800 microcomputer
	\item 1975 AD Bill Gates and Paul Allen found Microsoft
	\item 1975 AD The KIM-1 microcomputer
	\item 1975 AD The Sphere 1 microcomputer
	\item 1976 AD The Z80 microprocessor
	\item 1976 AD The Apple I and Apple II microcomputers
	\item 1977 AD The Commodore PET microcomputer
	\item 1977 AD The TRS-80 microcomputer
	\item 1979 AD The VisiCalc spreadsheet program
	\item 1979 AD ADA programming language is named after Ada Lovelace
	\item 1980 AD Danny Cohen writes "On Holy Wars and a Plea for Peace"
	\item 1981 AD The first IBM PC
	\item 1997 AD The first Beboputer Virtual Computer
	\item El futuro de la computación
   \end{topics}

   \begin{learningoutcomes}
     	\item Que el alumno conozca la evolución histórica que dio origen a los computadores y a la Ciencia de la Computación como la conocemos hoy en dia.
	\item Que el alumno conozca de forma analítica los hitos que marcaron el rumbo de esta área del conocimiento
	\item Que, a partir del conocimiento de la historia, el alumno se vea como un engranaje de cambio para la computación del futuro.
   \end{learningoutcomes}
\end{unit}

\begin{coursebibliography}
\bibfile{Computing/CS/CS181}
\end{coursebibliography}
\end{syllabus}

