\begin{syllabus}

\course{ME0020. Ciencia de Materiales}{Obligatorio}{ME0020} % Common.pm

\begin{justification}
La introducción y la innovación de este curso empieza con la presentación selecta de los fundamentos generales sobre Ciencia de los materiales e Ingeniería .
Luego, se enfoca en seminarios sobre la familia de materiales: metales y aleaciones, cerámicos y vidrios, polímeros y copolímeros, y compuestos y nanomateriales.
Las aplicaciones abarcan materiales tradicionales y de vanguardia. EL estudido de estas aplicaciones cubre el papel desempeñado por los materiales, 
las mismas aplicaciones y su relevancia. Casos avanzados sobre materiales e innovadores aplicaciones de relevancia potencial sobre el contexto peruano son cubiertos.

\end{justification}

\begin{goals}
\item Capacidad de trabajo en equipo.
\item Capacidad para identificar problemas de ingeniería.
\item Capacidad para comunicarse oralmente.
\item Capacidad para comunicarse por escrito.
\end{goals}

\begin{outcomes}{V1}
    \item \ShowOutcome{d}{2}
    \item \ShowOutcome{f}{2}
\end{outcomes}

\begin{competences}
    \item \ShowCompetence{C20}{d,f}
\end{competences}

\begin{unit}{Comprensión aplicada de los materiales}{}{MSaE2014}{0}{C20}
\begin{topics}
      \item Presentación y organización del curso.
      \item Importancia de los materiales para las Ciencias de la Ingeniería .
      \item Clasificación general de los materiales.
      \item Funciones deseables para materiales.
	  \begin{subtopics}
	      \item Propiedades mecánicas (por ejemplo materiales estructurales).
	      \item Conductividad eléctrica y térmica (por ejemplo, circuitos, células, sensores).
	      \item Resistencia química (por ejemplo compatibilidad química, corrosión).
	      \item Compatibilidad ambiental y biológica.
	  \end{subtopics}
      \item Fundamentos generales
	  \begin{subtopics}
	      \item Enlace químico y su impacto sobre la maleabilidad y la ductilidad
	      \item Aleaciones y diagramas de fases
	      \item Cristales crecimiento y defectos
	      \item Reactividad química (defectos, límites de grano)
	      \item Pares galvánicos
	      \item Diagramas de Pourbaix
	      \item Teoría de banda ,calor y conducción eléctrica
	      \item Conductores, semiconductores.
	  \end{subtopics}      
\end{topics}
   \begin{learningoutcomes}
    \item Comprender los fundamentos generales y las funciones deseables para los materiales.
    \item Reconociendo la importancia de adquirir una comprensión básica de los materiales para avanzar de forma autónoma en el área.
   \end{learningoutcomes}
\end{unit}

\begin{unit}{Manejo de Metales y Aleaciones}{}{MSaE2014}{0}{C20}
\begin{topics}
      \item Otros fundamentos específicos necesarios.
      \item Propiedades y aplicaciones correlacionadas.
      \item Estudio de metales y aleaciones - aplicaciones tradicionales
      \item Estudio de cerámica y vidrios - aplicaciones de vanguardia
\end{topics}
   \begin{learningoutcomes}
      \item Reconocer el propósito, requisitos y carácteristicas generales de Metales y Aleaciones.
   \end{learningoutcomes}
\end{unit}


\begin{unit}{Tratamiento con Cerámica y Vidrios }{}{MSaE2014}{0}{C20}
\begin{topics}
      \item Otros fundamentos específicos necesarios
      \item Propiedades y aplicaciones correlacionadas
      \item Estudio de metales y aleaciones - aplicaciones tradicionales
      \item Estudio de cerámica y vidrios - aplicaciones de vanguardia
\end{topics}
   \begin{learningoutcomes}

      \item Reconocer el propósito, los requisitos y las carácteristicas generales de Cerámica y Vasos.
   \end{learningoutcomes}
\end{unit}

\begin{unit}{Tratamiento con Polímeros y Copolímeros}{}{MSaE2014}{0}{C20}
\begin{topics}
      \item Otros fundamentos específicos necesarios
      \item Propiedades y aplicaciones correlacionadas
      \item Estudio de polímeros y copolímeros - aplicaciones tradicionales
      \item Estudio de polímeros y copolímeros - aplicaciones de vanguardia
\end{topics}
   \begin{learningoutcomes}
      \item Reconocer el propósito, requisitos y carácteristicas generales de Polímeros y Copolímeros.
   \end{learningoutcomes}
\end{unit}

\begin{unit}{Tratamiento de compuestos y con nanomateriales}{}{MSaE2014}{0}{C20}
\begin{topics}
      \item Otros fundamentos específicos necesarios.
      \item Propiedades y aplicaciones correlacionadas.
      \item Estudio de compuestos - aplicaciones tradicionales y de vanguardia
      \item Encuesta de nanomateriales: aplicaciones tradicionales y de vanguardia
\end{topics}
   \begin{learningoutcomes}
      \item Reconocer el propósito, los requisitos y las carácteristicas generales de los compuestos y nanomateriales.
   \end{learningoutcomes}
\end{unit}

\begin{unit}{Búsqueda de nuevos materiales y desarrollo de aplicaciones}{}{MSaE2014}{0}{C20}
\begin{topics}
      \item Par innovador  "material - aplicación", por ejemplo:
	  \begin{subtopics}
	    \item  Arte y conservación / restauración arqueológica
	    \item  Ambiente
	    \item  Nanomateriales
	    \item  Bioingeniería
	    \item  Impresión 3d
	    \item  Materiales funcionales
	    \item  Embalaje
	  \end{subtopics}  
\end{topics}
   \begin{learningoutcomes}
      \item Capacidad para integrar la comprensión de los nuevos materiales a las aplicaciones en desarrollo.
   \end{learningoutcomes}
\end{unit}














\begin{coursebibliography}
  \bibfile{BasicSciences/CB320}
\end{coursebibliography}
\end{syllabus}
