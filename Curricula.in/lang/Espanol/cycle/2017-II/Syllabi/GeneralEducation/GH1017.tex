\begin{syllabus}

\course{GH0017. Introducción al Quechua}{Electivo}{GH0017} % Common.pm

\begin{justification}
El curso de Quechua comunicativo permite acercar a los estudiantes al uso práctico de la lengua andina en su variedad chanca. Esta es una de las variedades de mayor difusión y modelo para abordar otras variedades del denominado quechua sureño o quechua II. Además, se presenta sencilla en su aprendizaje por compartir sonidos con el castellano. Asimismo, el curso busca familiarizar al alumno con las estructuras básicas de esta lengua, así como con la traducción y producción de textos. El objetivo último es proporcionar las herramientas básicas de aprendizaje de modo que el estudiante pueda expresarse en ella a un nivel básico y funcional, así como conducir y desarrollar su propio aprendizaje de la lengua.
Consideramos que hablar quechua en ciertas situaciones donde los ingenieros UTEC tienen que desarrollarse es una ventaja importantísima: los hablantes nativos de quechua practican un trato diferenciado con las personas que lo hablan por sentir que se está respetando su tradición y, a la vez, se está haciendo un esfuerzo por entablar un diálogo en su propia lengua. Esto representa ventajas operativas muy puntuales en el trato y el acuerdo de intereses. 
\end{justification}

\begin{goals}
\item Otorgar herramientas básicas para presentarse y conversar en la lengua quechua, en la variedad chanca.
\item Acercar al estudiante a las estructuras básicas de la lengua con el fin de dirigir su estudio y auto aprendizaje.
\item Entrenar al alumno en la traducción y producción de textos en la lengua nativa.
\item Proporcionar herramientas para que el alumno desarrolle el conocimiento de esta lengua de manera individual.
\item Dar herramientas para reconocer la procedencia del quechua al cual se enfrentan a través de elementos de análisis lingüístico
\end{goals}

\begin{outcomes}
    \item \ShowOutcome{d}{2} % Multidiscip teams
    \item \ShowOutcome{e}{2} % ethical, legal, security and social implications
    \item \ShowOutcome{f}{2} % communicate effectively
    \item \ShowOutcome{n}{2} % Apply knowledge of the humanities
    \item \ShowOutcome{p}{2} % TASDSH
\end{outcomes}

\begin{competences}
    \item \ShowCompetence{C10}{d,n,o}
    \item \ShowCompetence{C17}{f}
    \item \ShowCompetence{C18}{f}
    \item \ShowCompetence{C21}{e}
\end{competences}

\begin{unit}{Introducción al Quechua}{}{Cerron76,Press82}{12}{4}
   \begin{topics}
      \item Dialectología general del quechua.
      \item Sistema fonológico del quechua chanca: fonemas, sílaba, acento, pronunciación.
      \item Presentación, preguntas básicas, pedidos básicos.
      \item Frase nominal: pronombres, persona posesora, plural, casos gramaticales y pronombres interrogativos.
      \item Frase verbal: tiempos, personas verbales.
      \item Derivación deverbativa y denominativa.
      \item Temas oracionales: sintaxis.
      \item Partículas discursivas: validadores, reportativos, etc.
                
   \end{topics}

   \begin{learningoutcomes}
      \item Empleo de recursos comunicativos básicos en lengua quechua.
   \end{learningoutcomes}
\end{unit}



\begin{coursebibliography}
\bibfile{GeneralEducation/GH1017}
\end{coursebibliography}

\end{syllabus}
