\begin{syllabus}

\course{CF152. Laboratorio de FTecnologíasica 2}{Obligatorio}{CF152} % Common.pm

\begin{justification}
El propósito del curso es desarrollar habilidades en el uso de instrumental de laboratorio en los temas de electricidad y magnetismo, mediante técnicas, métodos y procedimientos para el análisis de datos experimentales usando un software para asTecnología verificar los fenómenos fTecnologíasicos que se presentan en la naturaleza y en el ámbito laboral. Desarrolla actitudes y habilidades para el logro de las siguientes competencias: comprensión de fenómenos fTecnologíasicos, análisis y sTecnologíantesis, aplicación de conocimientos teóricos en la práctica, comunicación oral y escrita, trabajo en equipo, habilidades de investigación, compromiso ético, motivación al logro de objetivos. Abarca experiencias de laboratorio de electricidad y magnetismo. 
\end{justification}

\begin{goals}
\item Al finalizar el semestre, el estudiante será capaz de relacionar su teoría de fTecnologíasica 3 a problemas y experiencias de la vida real en temas de la electricidad y magnetismo con la finalidad de reforzar su teoría y estar preparado en la resolución de problemas concretos a fines del curso.
\end{goals}

\begin{outcomes}
\item \ShowOutcome{a}{2}
\item \ShowOutcome{i}{2}
\end{outcomes}

\begin{competences}
    \item \ShowCompetence{C1,C20}{a,i}
\end{competences}

\begin{unit}{SESIóN 1. Campo eléctrico y superficies equipotenciales}{}{GuiaLab2018,Sears2013,Serway2008}{20}{C1,C20}
\begin{topics}
      \item Campo eléctrico y superficies equipotenciales de un arreglo de cargas puntuales y de uno de cargas no puntuales.
      \item Obtención de las lTecnologíaneas de campo eléctrico a partir de las curvas equipotenciales..
\end{topics}
\end{unit}

\begin{unit}{SESIóN 2. Capacitancia}{}{GuiaLab2018,Sears2013,Serway2008}{8}{C1,C20}
\begin{topics}
      \item Condensadores en serie y en paralelo.
      \item Carga y descarga de un condensador.
      \item Uso del galvanómetro. 
\end{topics}
\end{unit}

\begin{unit}{SESIóN 3. Ley de Ohm y circuitos con corriente continua}{}{GuiaLab2018,Sears2013,Serway2008}{8}{C1,C20}
\begin{topics}
      \item Uso del voltTecnologíametro y del amperTecnologíametro.
      \item Resistencias en serie.
      \item Resistencias en paralelo.
      \item Comportamiento óhmico y no óhmico de los materiales.
\end{topics}
\end{unit}

\begin{unit}{SESIóN 4. Campo magnético}{}{GuiaLab2018,Sears2013,Serway2008}{8}{C1,C20}
\begin{topics}
      \item Fuerza magnética.
      \item Campo magnético de un imán permanente.
      \item Determinación de la intensidad del campo magnético de un imán permanente.
   \end{topics}
\end{unit}

\begin{unit}{SESIóN 5. Inducción electromagnética}{}{GuiaLab2018,Sears2013,Serway2008}{8}{C1,C20}
\begin{topics}
	\item Ley de Faraday.
	\item Ley de Lenz.
	\item Fuerza electromotriz inducida.
	\item Transformadores de corriente alterna.
   \end{topics}
\end{unit}

\begin{unit}{Sesión 6. Corriente alterna}{}{GuiaLab2018,Sears2013,Serway2008}{4}{C1,C20}
   \begin{topics}
	\item Circuitos RL y RC en serie.
	\item Análisis de las ondas sinusoidales mostradas por un osciloscopio.
   \end{topics}
\end{unit}



\begin{coursebibliography}
\bibfile{BasicSciences/CF153}
\end{coursebibliography}

\end{syllabus}
