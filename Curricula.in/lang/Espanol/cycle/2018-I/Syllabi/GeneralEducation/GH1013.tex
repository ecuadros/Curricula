\begin{syllabus}

\course{GH0013. Crítica de la Modernidad}{Obligatorio}{GH0013} % Common.pm

\begin{justification}
El desarrollo del curso obedece a tres objetivos.En primer lugar, comprender la modernidad desde tres aspectos: científico , filosófico y artístico.
En segundo lugar,revisar, apartir de los aspectos anteriores, los conceptos fundamentales que constituyeron el imaginario moderno.Por último,discutir y 
dialogar entorno a las actuales críticas a la modernidad y sus presupuestos desde las nuevas teorias filosóficas y artisticas de los siglo XX y XXI.Estos
tres onjetivos se organizan alrededor de los dos horizontes que conforman la intención principal de nuestro curso:el horizonte cientifico moderno y el horizonte
artítico .Desde este doble horizonte expondremos y dialogaremos sobre la reconstucción del imaginario y de la mentalidad moderna , cuyos aportes y desarrollos 
múltiples constituyen e impactan aún en nuestro imaginario contemporáneo. El curso no solo se limita a la revisión historica, sino que implica el análisis de los
límites y alcances de la modernidad. Desde esta revisión entenderemos mejor las consecuencias contemporáneas del proyecto moderno.

\end{justification}

\begin{goals}
\item Capacidad de interpretar información.
\item Capacidad de comprender textos escritos.
\item Capacidad de comunicación escrita
\end{goals}

\begin{outcomes}{V1}
    \item \ShowOutcome{d}{2} % Multidiscip teams
    \item \ShowOutcome{e}{2} % ethical, legal, security and social implications
    \item \ShowOutcome{f}{2} % communicate effectively
    \item \ShowOutcome{n}{2} % Apply knowledge of the humanities
\end{outcomes}

\begin{competences}{V1}
    \item \ShowCompetence{C10}{d,n}
    \item \ShowCompetence{C17}{f}
    \item \ShowCompetence{C18}{f}
    \item \ShowCompetence{C21}{e}
\end{competences}

\begin{unit}{Qué es ser Moderno?.}{}{Echeverría08}{12}{4}
   \begin{topics}
      \item Introducción al curso.
   \end{topics}
   \begin{learningoutcomes}
      \item Desarrollar .
   \end{learningoutcomes}
\end{unit}

\begin{unit}{La modernidad Ciéntifica.}{}{Meca12}{12}{4}
   \begin{topics}
      \item Introducción al curso.
   \end{topics}
   \begin{learningoutcomes}
      \item Desarrollar .
   \end{learningoutcomes}
\end{unit}

\begin{unit}{Crítica de la Modernidad  .}{}{Alain14}{12}{4}
   \begin{topics}
      \item Introducción al curso.
   \end{topics}
   \begin{learningoutcomes}
      \item Desarrollar .
   \end{learningoutcomes}
\end{unit}






\begin{coursebibliography}
\bibfile{GeneralEducation/GH2015}
\end{coursebibliography}

\end{syllabus}
