\begin{syllabus}

\course{EG0004. Desafíos Globales}{Obligatorio}{EG0004} % Common.pm

\begin{justification}
Durante las sesiones plenarias, se realizarán clases magistrales relacionadas a la metodología de Design Thinking así como su uso e importancia en los procesos de creación . Así mismo, durante estas sesiones tendremos ponencias sobre emprendimientos y startups relacionados a la ingeniería o tecnología.
Durante las sesiones de laboratorio, los alumnos forman equipos que mantienen durante el ciclo. Con la guía del profesor y a través de la metodología del Design Thinking desarrollada en las plenarias, los alumnos deberán plantear soluciones innovadoras a problemas reales inspirados en los Global Challenges de las Naciones Unidas.
Los alumnos contarán con una Bitácora Digital que será revisada constantemente por los docentes a cargo. En ella se encontrarán los avances, procesos y referentes del proyecto grupal. El curso culmina con las presentaciones de las propuestas planteadas por los grupos.
\end{justification}

\begin{goals}
\item Capacidad de diseñar y llevar a cabo experimentos 
\item Capacidad de analizar información
\item Capacidad para diseñar un sistema, un componente o un proceso para satisfacer las necesidades deseadas dentro de restricciones realistas (Nivel 1)
\item Capacidad de trabajo en equipo
\item Capacidad de liderar un equipo
\item Capacidad de comunicación oral (Nivel 1)
\item Capacidad de comunicación escrita (Nivel 1)
\item Comprende el impacto de las soluciones de la ingeniería en un contexto global, económico, ambiental y de la sociedad.

\end{goals}

\begin{outcomes}{V1}
    \item \ShowOutcome{n}{2}
    \item \ShowOutcome{o}{2}
\end{outcomes}

\begin{competences}{V1}
    \item \ShowCompetence{C20}{n,ñ}
\end{competences}

\begin{unit}{Desafíos Globales}{}{Curedale12,Upton15}{12}{4}
   \begin{topics}
      \item Pasos de DT.
      \item Técnica y usos del Brainstorm.
      \item Conocimiento del usuario, empatía y uso de arquetipos.
      \item Tipos de Investigación, diferencias y usos.
      \item Estrategias de recolección de Insights.
      \item Métodos de Ideación.
      \item Introducción al Prototipado.
      \item Introducción a la Experiencia de Usuario.
      \item Estrategias de Testeo e Iteración
      \item Usos del Storytelling
   \end{topics}
   \begin{learningoutcomes}
      \item Flexibilidad y Adaptabilidad: Los alumnos aprenden a trabajar en equipo, en un ambiente flexible, variable y de constantes retos.
   \end{learningoutcomes}
\end{unit}

\begin{coursebibliography}
\bibfile{GeneralEducation/FG101D}
\end{coursebibliography}

\end{syllabus}
