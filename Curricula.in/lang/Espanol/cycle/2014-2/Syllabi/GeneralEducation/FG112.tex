\begin{syllabus}

\course{FG112. Matrimonio y Familia}{Electivo}{FG112}

\begin{justification}
La familia es una comunidad de personas que tiene su origen en el matrimonio entre un hombre y una mujer. "Intima comunidad de vida y amor'', que forman los padres con sus hijos y potencialmente otros miembros.   El matrimonio expresa: la comunión de vida que un varón y una mujer establecen entre si, libre, públicamente, y para toda la vida, en orden al perfeccionamiento mutuo y a la generación y educación de los hijos. 
Para la persona: Es la "escuela de humanidad más completa y rica'' : la entrega reciproca del varón y la mujer unidos en matrimonio, crea  un ambiente en el que el niño aprende lo que significa amar y ser amado, y descubre así su dignidad personal y la del otro. 
Para la sociedad: La vida en familia es el fundamento de la sociedad. La persona inicia allí, la vida en sociedad. La autoridad, la estabilidad, y la vida en relación en el seno de la familia, constituyen por otra parte, los fundamentos de la seguridad y fraternidad en el seno de la sociedad. 
\end{justification}

\begin{goals}
	\item Comprender que la familia es una comunión de vida y amor, fundado en el matrimonio entre un hombre y una mujer, para toda la vida en orden al perfeccionamiento mutuo y a la procreación y educación de los hijos.
	\item Que el alumno entienda los criterios fundamentales sobre los que descansa una recta comprensión de la persona, el matrimonio y la familia
	\item Que el alumno tenga elementos para comprender la vida afectiva como un llamado a la vida matrimonial y familiar
	\item Comprender la importancia de la familia para la persona y para la sociedad entera.
\end{goals}

\begin{outcomes}
\ExpandOutcome{HU}{2}
\ExpandOutcome{FH}{2}
\end{outcomes}

\begin{unit}{Persona y Matrimonio}{JuanPabloII}{21}{2}
\begin{topics}
	\item Introducción 
	\item Persona Humana: Unidad. Niveles de acción. Las emociones. Integración. 
	\item Matrimonio: Algunas definiciones. El diálogo: una forma de amar. Bienes del matrimonio. Características del matrimonio. El amor conyugal y la apertura a la vida. Castidad  y sexualidad.El matrimonio como institución  
\end{topics}

\begin{learningoutcomes}
	\item Que el alumno se entienda como un ser creado para el encuentro e invitado a participar del amor, y como el matrimonio responde a su naturaleza más profunda.  
\end{learningoutcomes}
\end{unit}

\begin{unit}{La Familia: Importancia y Derechos}{Concilio,Pontificio,SantaSede}{12}{2}
\begin{topics}
	\item La Familia: Concepto. Relaciones familiares. La familia: Comunidad de personas. Proceso de cambio social y cultural. Nuevo enfoque: Perspectiva de Familia.
	\item Los Derechos de la familia.	
\end{topics}

\begin{learningoutcomes}
	\item Que el alumno se entienda como un ser creado para el encuentro e invitado a participar del amor, y como el matrimonio responde a su naturaleza más profunda.  
\end{learningoutcomes}
\end{unit}

\begin{unit}{Situación de la Familia en el Mundo Actual}{Biblia}{18}{2}
\begin{topics}
	\item Ideología de Género.
	\item Divorcio.
	\item Convivencia y relaciones libres.
	\item Homosexualidad.
	\item Anticoncepción y mito poblacional.
	\item El Futuro de la humanidad.

\end{topics}

\begin{learningoutcomes}
	\item Comprender la importancia de la familia como célula fundamental de la sociedad y corazón de la civilización.
\end{learningoutcomes}
\end{unit}



\begin{coursebibliography}
\bibfile{GeneralEducation/FG112}
\end{coursebibliography}

\end{syllabus}
