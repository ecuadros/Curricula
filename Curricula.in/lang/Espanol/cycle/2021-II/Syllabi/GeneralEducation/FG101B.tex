\begin{syllabus}

\course{. }{}{} % Common.pm

\begin{justification}
Para lograr una eficaz comunicación en el ámbito personal y profesional, es prioritario el manejo adecuado de la Lengua en forma oral y escrita. Se justifica, por lo tanto, que los alumnos  conozcan, comprendan y apliquen los aspectos conceptuales y operativos de su idioma, para el desarrollo de sus habilidades comunicativas fundamentales: Escuchar, hablar, leer y escribir.
En consecuencia el ejercicio permanente y el aporte de los fundamentos contribuyen grandemente en la formación académica y, en el futuro, en el desempeño de su profesión
\end{justification}

\begin{goals}
\item Desarrollar capacidades comunicativas a través de la teoría y práctica del lenguaje que ayuden al estudiante a superar las exigencias académicas del pregrado y contribuyan a su formación humanística y como persona humana.
\end{goals}

\begin{outcomes}
   \item \ShowOutcome{f}{2}
   \item \ShowOutcome{h}{2}
   \item \ShowOutcome{n}{2}
\end{outcomes}

\begin{competences}
    \item \ShowCompetence{C17}{f,h,n}
    \item \ShowCompetence{C20}{f,n}
    \item \ShowCompetence{C24}{f,h}
\end{competences}

\begin{unit}{Presentación del Curso}{}{Real}{16}{C17,C20}
  \begin{topics}
      \item Comunicación Oral y Escrita II.
      \item Características del texto argumentativo.
  \end{topics}

  \begin{learningoutcomes}
   \item Que el alummno reafirme las diferencias entre un texto Argumentativo apartir de ejemplos.
  \end{learningoutcomes}
\end{unit}

\begin{unit}{?`Qué es un informe de laboratorio?}{}{Real}{16}{C17,C20}
  \begin{topics}
      \item Características y partes.
      \item Presentación de un modelo y análisis.
      \item Presentación de la metodología 
  \end{topics}

  \begin{learningoutcomes}
   \item .%ToDo
  \end{learningoutcomes}
\end{unit}

\begin{unit}{Informe de Laboratorio :Resultados de laboratorio y aplicaciones}{}{Real}{16}{C17,C20}
  \begin{topics}
      \item Presentación de las carácteristicas de estas partes.
      \item Ejercicio de redacción.
  \end{topics}

  \begin{learningoutcomes}
   \item .%ToDo
  \end{learningoutcomes}
\end{unit}

\begin{unit}{Informe de laboratorio: Introducción y conclusiones}{}{Real}{16}{C17,C20}
  \begin{topics}
      \item Presentación de las carácteristicas de estas partesl.
      \item Ejercicio de redacción.
  \end{topics}

  \begin{learningoutcomes}
   \item .%ToDo
  \end{learningoutcomes}
\end{unit}

\begin{unit}{Citado, referencias parentéticas y construcción de bibliografía}{}{Real}{16}{C17,C20}
  \begin{topics}
      \item Formato APA.
  \end{topics}

  \begin{learningoutcomes}
   \item .%ToDo
  \end{learningoutcomes}
\end{unit}

\begin{unit}{Características de oralidad}{}{Real}{16}{C17,C20}
  \begin{topics}
      \item Análisis de un ted talk .
      \item Características de exposición oral.
  \end{topics}

  \begin{learningoutcomes}
   \item .%ToDo
  \end{learningoutcomes}
\end{unit}

\begin{unit}{Preparación para la exposición oral}{}{Real}{16}{C17,C20}
  \begin{topics}
      \item Rúbrica de evaluación de pares
  \end{topics}

  \begin{learningoutcomes}
   \item .%ToDo
  \end{learningoutcomes}
\end{unit}

\begin{unit}{Presentación de un texto argumentativo y carácteristicas de la argumentación}{}{Real}{16}{C17,C20}
  \begin{topics}
      \item Pautas para delimitar tema.
      \item Esquema y Resumen.
      \item Elaboración de esquemas argumentativos.
      \item Redacción de argumentos breves.
  \end{topics}

  \begin{learningoutcomes}
   \item .%ToDo
  \end{learningoutcomes}
\end{unit}

\begin{unit}{?`Cómo se construye un argumento?}{}{Real}{16}{C17,C20}
  \begin{topics}
      \item El argumento pragmático
      \item Ejercicios de identificación 
      \item Ejercicio de redacción de argumento pragmático.
  \end{topics}

  \begin{learningoutcomes}
   \item .%ToDo
  \end{learningoutcomes}
\end{unit}

\begin{unit}{Contraargumentación}{}{Real}{16}{C17,C20}
  \begin{topics}
      \item Qué es la contraargumentación.
      \item Modelos de textos contraargumentativos.
      \item Elaboración de un esquema contraargumentativo.
      \item Redacción de la contraargumentación.
  \end{topics}

  \begin{learningoutcomes}
   \item .%ToDo
  \end{learningoutcomes}
\end{unit}

\begin{unit}{Solidez argumentativa de la contraargumentación}{}{Real}{16}{C17,C20}
  \begin{topics}
      \item Entrega de la contraargumentación
  \end{topics}

  \begin{learningoutcomes}
   \item .%ToDo
  \end{learningoutcomes}
\end{unit}

\begin{unit}{Debates}{}{Real}{16}{C17,C20}
  \begin{topics}
      \item Debates.
  \end{topics}

  \begin{learningoutcomes} 
   \item .%ToDo
  \end{learningoutcomes}
\end{unit}



\begin{coursebibliography}
\bibfile{GeneralEducation/FG101B}
\end{coursebibliography}

\end{syllabus}
