\begin{syllabus}

\course{GH0019. Emprendedores en Acción}{Electivo}{GH0019}
% Source file: ../Curricula.in/lang/Espanol/cycle/2020-I/Syllabi/GeneralEducation/GH1019.tex

\begin{justification}
El propósito de este curso es que los estudiantes adquieran las herramientas y conocimientos específicos para realizar un análisis de mercado que se refleje en: (i) un plan de negocios; Y (ii) el desarrollo de habilidades de liderazgo, trabajo en equipo y comunicación efectiva.
Esto se logrará trabajando junto con un empresario, llevando al alumno a los problemas cotidianos que surgen en las empresas.
Este curso es prácticamente práctico, donde lo que se aprende en el aula se utilizará para analizar el mercado junto con el empresario, siguiendo la estructura de un plan de negocios. Por lo tanto, el estudiante aplicará este conocimiento y adquirirá durante su carrera, siempre guiado por el profesor y los ayudantes.
Por un lado, el alumno estará conectado con un caso real de emprendimiento, para que aprenda mediante la técnica "aprender haciendo". Por otra parte, se intentará reducir la tasa de fracaso de los empresarios (según Small Business Administration [http://www.sba.gov]), el 95 porciento de los empresarios fallan antes del quinto año, debido principalmente a la falta De diferenciación Con competencia y falta de una estrategia de marketing efectiva).
Los empresarios que serán asesorados en el curso de Emprendimiento Aplicado pertenecen a la Fundación Independizate (www.fundacionindependizate.cl), y son personas con un nivel técnico o profesional que saben mucho sobre su producto pero que tienen fallas en análisis de mercado y estrategias. Ventas Y comercialización.
\end{justification}

\begin{goals}
\item Analizar las partes que componen un plan de negocios, tales como segmentación, estrategias de marketing y flujos de efectivo.
\item Analizar el mercado y las oportunidades que existen para abrir un nuevo negocio, donde se hará hincapié en la identificación de estas oportunidades y la propuesta de valor.
\item Comprender el funcionamiento actual de la empresa, sus debilidades y fortalezas, y luego hacer una propuesta con valor para el empresario.
\item Entender cómo hacer avanzar un proyecto, liberándolo del "valle de la muerte", donde los empresarios a menudo se quedan atascados.
\item Aplicar los conocimientos adquiridos por el alumno a lo largo de su carrera a través del trabajo práctico con emprendedores, que representa el eje principal de este curso.
\item Desarrollar liderazgo en investigación y desarrollo de metodologías de evaluación de pequeñas empresas.
\end{goals}

\begin{outcomes}{V1}
    \item \ShowOutcome{n}{2}
    \item \ShowOutcome{o}{2}
\end{outcomes}

\begin{competences}{V1}
    \item \ShowCompetence{C24}{n,ñ}
\end{competences}

\begin{unit}{Modelos de Negocios}{}{Kotler08}{12}{4}
   \begin{topics}
      \item .
   \end{topics}

   \begin{learningoutcomes}
      \item Que el estudiante entienda cuáles son las diferentes formas en que una empresa puede generar ingresos. Muchas veces los empresarios están seguros de que es sólo a través de un solo camino, sin darse cuenta de que tienen múltiples oportunidades.
   \end{learningoutcomes}

\end{unit}

\begin{unit}{Segmentando al Mercado}{}{Kotler08}{24}{3}
   \begin{topics}
      \item .
   \end{topics}

   \begin{learningoutcomes}
      \item Entregar herramientas a los estudiantes para llevar a los empresarios a lograr una buena segmentación del mercado. Se llevarán a cabo herramientas prácticas para llevar a cabo un estudio de mercado y se analizarán diferentes formas de segmentación.
   \end{learningoutcomes}

\end{unit}

\begin{unit}{Estudiando a la Competencia}{}{Kotler08}{24}{3}
   \begin{topics}
      \item .
   \end{topics}

   \begin{learningoutcomes}
      \item Que el estudiante pueda transmitir al emprendedor los beneficios de conocer la competencia en profundidad, y la importancia de lograr la diferenciación.
   \end{learningoutcomes}

\end{unit}

\begin{unit}{Estrategias de Marketing}{}{Wiley07}{30}{3}
   \begin{topics}
      \item . 
   \end{topics}

   \begin{learningoutcomes}
      \item Que el estudiante domine las tácticas de marketing eficientes para los empresarios con bajo presupuesto.
   \end{learningoutcomes}
\end{unit}

\begin{unit}{Estrategias de Venta}{}{Wiley07}{30}{3}
   \begin{topics}
      \item .
   \end{topics}

   \begin{learningoutcomes}
      \item Que el estudiante desarrolle las herramientas para llevar a cabo una venta, profundizando en la introducción de productos en los puntos de venta, así como en la venta de servicios a terceros.
   \end{learningoutcomes}
\end{unit}

\begin{unit}{Implementación/ Operaciones}{}{Kotler08}{30}{3}
   \begin{topics}
      \item . 
   \end{topics}

   \begin{learningoutcomes}
      \item Que el alumno domine los temas relacionados con la organización, planificación y gestión del control en las pequeñas empresas.
   \end{learningoutcomes}

\end{unit}

\begin{unit}{Proyecciones Financieras}{}{Wiley07}{30}{3}
   \begin{topics}
      \item Que el estudiante pueda hacer proyecciones financieras, profundizando el flujo de caja.
   \end{topics}

   \begin{learningoutcomes}
      \item .
   \end{learningoutcomes}
\end{unit}

\begin{coursebibliography}
\bibfile{GeneralEducation/GH1019}
\end{coursebibliography}

\end{syllabus}
