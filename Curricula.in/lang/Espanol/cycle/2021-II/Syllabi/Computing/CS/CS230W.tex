\begin{syllabus}

\course{IT303. Comunicación de Datos y Redes}{Obligatorio}{IT303}
% Source file: ../Curricula.in/lang/Espanol/cycle/2020-I/Syllabi/Computing/CS/CS230W.tex

\begin{justification}
   Con el desarrollo de las tecnologías de comunicación   y la información
    hace que exista una tendencia creciente a  establecer  más redes de
    computadores,  con el objetivo de realizar una mejor gestión de la
    información.  Ello implica, los temas de sistemas de comunición de  datos,
     seguridad, redes de area extensa  y redes  locales, etc.  . Que permitan
     interpretar la  evolución, divisar  el desarrollo  futuro de las
     nuevas  tecnologías en redes de datos.
   \end{justification}
   
   \begin{goals}
   \item Permitir al alumno gestionar y programar la configuración de una red LAN y de una red WAN.
   \item Dotar al alumno de conceptos de seguridad y de tecnologías futuras de redes de datos.
   \item Desarrollar la habilidad para analizar y diseñar nuevos protocoles de red para casos específicos.
   \end{goals}
   
   \begin{outcomes}{V1}
      \item \ShowOutcome{b}{3}
      \item \ShowOutcome{c}{3}
      \item \ShowOutcome{e}{2}
      \item \ShowOutcome{g}{2}
      \item \ShowOutcome{i}{2}
      \item \ShowOutcome{j}{4}
   \end{outcomes}
   
   \begin{unit}{\NCIntroduction}{}{Stalling05,Tanenbaum05,Cisco04,Wasserman08,McNab08,Gast08}{12}{b}
      \NCIntroductionAllTopics
      \NCIntroductionAllLearningOutcomes
   \end{unit}
   
   \begin{unit}{\NCNetworkedApplications}{}{Stalling05,Stevens05,Tanenbaum05,Gast08}{12}{b,c}
      \NCNetworkedApplicationsAllTopics
      \NCNetworkedApplicationsAllLearningOutcomes
   \end{unit}
   
   \begin{unit}{NC/Tecnologias de Redes Locales}{}{Cisco04,Stevens06,Wasserman08,Gast08,McNab08}{16}{b,c}
      \begin{topics}
         \item Evaluación de las  Redes  Locales.
         \item Protocolo CSMA. CD Ethernet.
         \item Diseño y  análisis de  tráfico para intranets.
      \end{topics}
   
      \begin{learningoutcomes}
         \item Estudiar la tecnologias  ethernet  para redes lan, protocolo MAC, protocolo LLC.
         \item Usar las  herramientas  adecuadas para realizar un diganóstico del rendimiento de una Intranet.
      \end{learningoutcomes}
   \end{unit}
   
   \begin{coursebibliography}
   \bibfile{Computing/CS/CS230W}
   \end{coursebibliography}
   
   \end{syllabus}
   