\begin{syllabus}

\begin{justification}
Parte fundamental de la formación integral de un profesional es la habilidad de 
comunicarse en un idioma extranjero además del propio idioma nativo. No solamente 
amplía su horizonte cultural sino que permite una visión más humana y comprensiva de 
la vida. En el caso de los idiomas extranjeros, indudablemente el Inglés es el más 
práctico porque es hablado alrededor de todo el mundo. No hay país alguno donde éste 
no sea hablado. En las carreras relacionadas con los servicios al turista el inglés 
es tal vez la herramienta práctica más importante que el alumno debe dominar desde 
el primer momento como parte de su formación integral.
\end{justification}

\begin{outcomes}
\ExpandOutcome{l}{1}
\ExpandOutcome{HU}{1}
\end{outcomes}

\begin{unit}{The world of work!}{Soars023S,Soars023W,Soars023T, Cambridge06, MacGrew99}{0}{1}
   \begin{topics}
      \item Presente Perfecto vs Pasado Simple
      \item Voz Pasiva en Presente Perfecto 
      \item Frases Verbales
      \item Expresiones de sentido literal e idiomático
      \item Expresiones a utilizar al dejar mensajes por teléfono
      \item Ocupaciones 
   \end{topics}

   \begin{unitgoals}
      \item Al terminar la octava unidad, cada uno de los alumnos, comprendiendo la gramática del tiempo presente perfecto es capaz de expresar ideas literales y otras idiomáticas y además usar adjetivos de todo tipo, adverbios de modificación y sustantivos compuestos. Además es capaz de analizar y expresar ideas acerca de estar de acuerdo y en desacuerdo. Inglés coloquial, pedir disculpas, hacer sugerencias precisas y aclarar ventajas y desventajas.
   \end{unitgoals}

\end{unit}

\begin{unit}{Just imagine!}{Soars023S,Soars023W,Soars023T, Cambridge06, MacGrew99}{0}{1}
   \begin{topics}
      \item Condicionales
      \item Primer Condicional
      \item Segundo Condicional
      \item Cláusulas de Tiempo
      \item Tipos de Adjetivos
      \item Adverbios como modificadores
      \item Expresiones para hacer sugerencias
   \end{topics}

   \begin{unitgoals}
      \item Al terminar la novena unidad, los alumnos habiendo identificado la forma de expresar condicionales y cláusulas de tiempo los utilizan en situaciones varias. Expresar situaciones y estados relacionados con significados literales y coloquiales. Explica y aplica vocabulario de mensajes telefónicos.
   \end{unitgoals}

\end{unit}

\begin{unit}{Relationships!}{Soars023S,Soars023W,Soars023T, Cambridge06, MacGrew99}{0}{1}
   \begin{topics}
      \item Auxiliares de Modo 2
      \item Modales en Presente
      \item Modales en Pasado
      \item Modales en Negativo
      \item Adjetivos de carácter
      \item Expresiones para estar de acuerdo y en desacuerdo
      \item Vocabulario de familias
   \end{topics}

   \begin{unitgoals}
      \item Al terminar la décima unidad, los alumnos habiendo reconocido las características  de auxiliares de modo en tiempo pasado. Describen personas según su carácter. Utilizarán expresiones para estar de acuerdo y en desacuerdo. Discuten asuntos familiares. 
   \end{unitgoals}

\end{unit}

\begin{unit}{Obsessions!}{Soars023S,Soars023W,Soars023T, Cambridge06, MacGrew99}{0}{1}
   \begin{topics}
      \item Presente Perfecto Continuo y Simple
      \item Oraciones Afirmativas, Negativas y Preguntas
      \item Sustantivos Compuestos 
      \item Expresiones de Cantidad
      \item Expresiones de tiempo variadas
      \item Coleccionistas famosos
   \end{topics}

   \begin{unitgoals}
      \item Al terminar la décimo primera unidad, los alumnos habiendo identificado la idea de expresar ideas de acciones que suceden en un momento especí­fico del pasado y que se relacionan a cualquier tiempo estructuran oraciones en Presente Perfecto Progresivo. Expresan ideas de sustantivos compuestos y expresiones de cantidad con respecto a eventos de la vida diaria y a coleccionistas.
   \end{unitgoals}

\end{unit}

\begin{unit}{Tell me about it!}{Soars023S,Soars023W,Soars023T, Cambridge06, MacGrew99}{0}{1}
   \begin{topics}
      \item Preguntas Indirectas
      \item Question Tags 
      \item Verbos y sustantivos que van juntos
      \item Verbos de acción
      \item Expresiones idiomáticas
      \item Inglés Informal
      \item Ventajas y desventajas
   \end{topics}

   \begin{unitgoals}
      \item Al finalizar la décimo segunda unidad, los alumnos, a partir de la comprensión de diferentes tipos de preguntas, como las indirectas y las tag questions, elaborarán oraciones utilizando los elementos necesarios. Asimilarán además la necesidad de expresar expresiones idiomáticas y verbos y sustantivos que van unidos. Adquirirán vocabulario para describir historias cortas y creativas. Se presentará expresiones para hacer altos en el trabajo diario.
   \end{unitgoals}

\end{unit}

\begin{unit}{Life's great events!}{Soars023S,Soars023W,Soars023T, Cambridge06, MacGrew99}{0}{1}
   \begin{topics}
      \item Oraciones de Reporte de Ideas
      \item Reporte de afirmaciones
      \item Reporte de preguntas
      \item Reporte de comandos y órdenes
      \item Expresiones para nacimientos, matrimonios y fallecimientos
      \item Pedir disculpas
      \item Corrección de errores
   \end{topics}

   \begin{unitgoals}
      \item Al finalizar la décimo tercera unidad, los alumnos habiendo conocido los fundamentos de las oraciones de reporte de mensajes las usarán en la estructuración de ideas y preguntas diversas. Realizarán trabajos aplicativos en contextos adecuados. Enfatizan la diferencia entre nacimiento, matrimonio y muertes. Describen la necesidad de pedir disculpas. Utilizan expresiones para describir costumbres. Asumen la idea corregir errores.
   \end{unitgoals}
\end{unit}

\begin{coursebibliography}
\bibfile{ForeignLanguages/ID101}
\end{coursebibliography}
\end{syllabus}
%\end{document}

