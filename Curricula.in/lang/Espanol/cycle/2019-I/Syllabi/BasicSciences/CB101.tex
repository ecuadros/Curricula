\begin{syllabus}

\course{CB101. Álgebra y Geometría}{Obligatorio}{CB101}

\begin{justification}
Curso introductorio, soporte de los posteriores cursos de Análisis Matemático, 
estudia el plano y el espacio, haciendo énfasis es su aspecto vectorial y su interpretación geométrica, 
lo que permite visualizar conceptos que posteriormente se verán en forma abstracta.
\end{justification}

\begin{goals}
\item Familiarizarse y manejar las matrices, determinantes y sus relaciones con los sistemas de ecuaciones y aplicaciones.
\item Establecer relaciones lineales y cuadráticas en el plano y en el espacio.
\item Relacionar el álgebra con la geometría, de modo que visualice problemas que de otro modo serían abstractos.
\end{goals}

\begin{outcomes}
\ExpandOutcome{a}{3}
\ExpandOutcome{i}{2}
\ExpandOutcome{j}{4}
\end{outcomes}

\begin{unit}{Sistemas de coordenadas. La recta.}{Lehmann05}{12}{4}
   \begin{topics}
      \item El plano cartesiano
      \item La Recta, Ecuaciones de la recta
   \end{topics}
   \begin{learningoutcomes}
      \item Identificar, graficar una recta y manejarla en sus diferentes formas.
   \end{learningoutcomes}
\end{unit}

\begin{unit}{Cónicas y Coordenadas polares}{Lehmann05}{24}{3}
   \begin{topics}
      \item Cónicas
      \item Coordenadas Polares
   \end{topics}

   \begin{learningoutcomes}
      \item Reconocer las ecuaciones de las cónicas. Trazar la gráfica de una cónica descrita en su forma canónica y viceversa.
      \item Manejar el cambio de coordenadas polares a cartesianas y viceversa
      \item Trazar la gráfica de una curva en coordenadas polares
      \end{learningoutcomes}
\end{unit}

\begin{unit}{Sistemas de ecuaciones. Matrices y determinantes}{Strang03,Grossman96}{24}{3}
   \begin{topics}
      \item Sistemas de ecuaciones lineales
      \item Matrices
      \item Determinantes
      \end{topics}

   \begin{learningoutcomes}
      \item Resolver sistemas de ecuaciones lineales utilizando los métodos de eliminación
      \item Determinar la consistencia e inconsistencia de un sistema
      \item Identificar y manipular los diferentes tipos de matrices, así como el álgebra de matrices
      \item Relacionar las matrices con los sistemas de ecuaciones lineales
      \item Calcular determinantes e inversas de matrices
   \end{learningoutcomes}
\end{unit}

\begin{unit}{Vectores en $R^2$ y vectores en $R^3$}{Grossman96}{30}{3}
   \begin{topics}
      \item Vectores en $R^2$
      \item Vectores en $R^3$
   \end{topics}

   \begin{learningoutcomes}
      \item Manipular las operaciones con vectores. Interpretarlos geométricamente.
      \item Aplicar los vectores a la resolución de problemas geométricos.
      \item Formular y analizar la ecuación vectorial de la recta y el plano. Manipular ecuaciones de planos
   \end{learningoutcomes}
\end{unit}



\begin{coursebibliography}
\bibfile{BasicSciences/CB101}
\end{coursebibliography}

\end{syllabus}
