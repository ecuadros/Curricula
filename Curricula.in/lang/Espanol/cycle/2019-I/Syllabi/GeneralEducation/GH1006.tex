\begin{syllabus}

\course{GH0006. Laboratorio de Comunicación II}{Obligatorio}{GH0006}
% Source file: ../Curricula.in/lang/Espanol/cycle/2019-I/Syllabi/GeneralEducation/GH1006.tex

\begin{justification}
Este laboratorio está orientado a consolidar las habilidades comunicativas del estudiante, tanto a nivel oral como escrito en el marco de la disciplina que se estudia. En particular, el estudiante fortalecerá sus capacidades expositivas al ejercitarse en toda la primera parte del curso en la escritura de un tipo de texto que
desarrollará a lo largo de su carrera como ingeniero: los informes de laboratorio. Reflexionará sobre la situación retórica que enfrenta al escribir este tipo de texto: quién será su lector, cuál es la intención comunicativa de ese texto y el tema sobre el que está escribiendo.
En una segunda parte, el curso se presenta como un espacio de discusión sobre el discurso argumentativo y de lectura crítica de textos argumentativos, para que el alumno reflexione, conozca y emplee las herramientas comunicativas para producir textos argumentativos formales. En este sentido, el curso se orienta hacia la producción
permanente de textos escritos y orales, por lo que el alumno participará no solo en foros de discusión sino que se espera que sea capaz de debatir con sus compañeros sobre un tema propuesto por el profesor. En suma, el curso busca consolidar las competencias de lectura, análisis y elaboración de textos escritos y orales, tanto expositivos como argumentativos.
\end{justification}

\begin{goals}
\item Desarrollar habilidades que les permitan a los estudiantes mejorar sus capacidades comunicativas, tanto orales como escritas.
\item Comprender y producir textos expositivos en los que informen sobre la aplicación del conocimiento teórico en un experimento o contexto diferente.
\item Comprender y producir textos argumentativos orales y escritos.
\item Se capaz de debatir empleando argumentos sólidos.
\item Emplear adecuadamente y reflexivamente la información obtenida en diferentes fuentes.
\item Mostrar apertura y respeto para escuchar la diversidad de opiniones o puntos de vista de los compañeros de clase.
\end{goals}

\begin{outcomes}{V1}
   \item \ShowOutcome{i}{2}
   \item \ShowOutcome{f}{2}
\end{outcomes}

\begin{competences}{V1}
    \item \ShowCompetence{C17}{i}
    \item \ShowCompetence{C20}{f}
    \item \ShowCompetence{C24}{f}
\end{competences}

\begin{unit}{Laboratorio de Comunicación II}{}{Cassany08}{12}{C17}
   \begin{topics}
      \item ?`Qué es un informe de laboratorio ?
      \item Desarrollo del Laboratorio y aplicaciones.
      \item Resultados de Laboratorio y aplicaciones.
      \item Introducción y conclusiones.
      \item Citado,referencias parentéticas y contrucción de bibliografiaTecnologia.
      \item Preparación para la exposición oral.
      \item Presentación de un texto Argumentativo: textos formales y  no formales.
      \item Citado,referencias (formato APA)
      
   \end{topics}
\begin{learningoutcomes}
      \item Manejar adecuadamente el sistema citado y de referencias bibliográficas,y reconocer la importacia de su uso.
   \end{learningoutcomes}
\end{unit}

\begin{coursebibliography}
\bibfile{GeneralEducation/GH1006}
\end{coursebibliography}

\end{syllabus}
