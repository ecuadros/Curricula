\begin{syllabus}

\course{GH1102. Inglés II}{Obligatorio}{GH1102} % Common.pm

\begin{justification}
Parte fundamental de la formación integral de un profesional es la habilidad de 
comunicarse en un idioma extranjero además del propio idioma nativo. No solamente 
amplía su horizonte cultural sino que permite una visión más humana y comprensiva 
de la vida de las personas. En el caso de los idiomas extranjeros, indudablemente 
el Inglés es el más práctico porque es hablado alrededor de todo el mundo. No hay 
país alguno donde éste no sea hablado. En las carreras relacionadas con los 
servicios al turista el Inglés es tal vez la herramienta práctica más importante 
que el alumno debe dominar desde el primer momento, como parte de su formación 
integral.
\end{justification}

\begin{goals}
    \item Desarrollar la capacidad de hablar fluídamente el idioma.
    \item Incrementar el vocabulario y el manejo de expresiones simples.
\end{goals}

\begin{outcomes}{V1}
    \item \ShowOutcome{f}{2}
\end{outcomes}

\begin{competences}{V1}
    \item \ShowCompetence{C25}{f}
\end{competences}

\begin{unit}{How long ago?}{}{Soars021S,Cambridge06, MacGrew99}{0}{2}
   \begin{topics}
      \item Pasado Simple
      \item Oraciones Negativas con ago.
      \item Conjunciones
      \item Expresiones de Tiempo en pasado
      \item Relaciones y sí­mbolos fonéticos
      \item Expresiones para dar la fecha
   \end{topics}

   \begin{learningoutcomes}
      \item Al terminar la octava unidad, cada uno de los alumnos, comprendiendo la gramática del tiempo pasado es capaz de expresar una mayor cantidad de expresiones de tiempo y además usar preposiciones para describir lugares y tiempos variados. Además es capaz de analizar y expresar ideas acerca de fechas y números en orden. 
   \end{learningoutcomes}

\end{unit}
\begin{unit}{Food you like!}{}{Soars021S,Cambridge06, MacGrew99}{0}{2}
   \begin{topics}
      \item Sustantivos Contables y No Contables
      \item Expresiones con Would like y I'd like
      \item Cuantificadores
      \item Comidas alrededor del mundo
      \item Pedidos formales
      \item Cartas formales
   \end{topics}

   \begin{learningoutcomes}
      \item Al terminar la novena unidad, los alumnos habiendo identificado la forma de expresar pedidos y hacer ofrecimientos, los utilizan en situaciones varias. Expresar situaciones y estados relacionados con cantidades. Explica y aplica vocabulario de comidas y bebidas.
   \end{learningoutcomes}
\end{unit}

\begin{unit}{The world of work}{}{Soars021S,Cambridge06, MacGrew99}{0}{2}
   \begin{topics}
      \item Adjetivos
      \item Oraciones con Adjetivos Comparativos.
      \item Oraciones con Adjetivos Superlativos
      \item Ciudades y el campo
      \item Indicaciones de dirección
   \end{topics}

   \begin{learningoutcomes}
      \item Al terminar la décima unidad, los alumnos habiendo reconocido las características de los adjetivos, utilizan éstos para hacer comparaciones de diversos tipos. Describen personas y lugares y dan indicaciones de dirección. Utilizarán conjunciones para unir ideas tipo. 
   \end{learningoutcomes}

\end{unit}

\begin{unit}{Looking good!}{}{Soars021S,Cambridge06, MacGrew99}{0}{2}
   \begin{topics}
      \item Presente Continuo
      \item Oraciones Afirmativas, Negativas y Preguntas
      \item Uso de Whose
      \item Pronombres Posesivos
      \item Ropa y colores
      \item Expresiones a usar en tiendas de ropa
      \item Símbolos fonéticos.
   \end{topics}

   \begin{learningoutcomes}
      \item Al terminar la décimo primera unidad, los alumnos habiendo identificado la idea de expresar ideas de acciones que suceden en el momento o que se relacionan a cualquier tiempo estructuran oraciones en Presente Progresivo. Expresan ideas de posesión con respecto a la ropa y los colores.
   \end{learningoutcomes}

\end{unit}

\begin{unit}{Life is an adventure!}{}{Soars021S,Cambridge06, MacGrew99}{0}{2}
   \begin{topics}
      \item Uso de going to
      \item Oraciones en Tiempo Futuro
      \item Expresiones de Cantidad.
      \item Verbos de acción
      \item Vocabulario del clima
      \item Expresiones de Sugerencia
      \item Escribir una postal
   \end{topics}
   \begin{learningoutcomes}
      \item Al finalizar la décimo segunda unidad, los alumnos, a partir de la comprensión del tiempo futuro, elaborarán oraciones utilizando los elementos necesarios. Asimilarán además la necesidad de expresar infinitivos de propósito. Adquirirán vocabulario para describir el clima. Se presentará expresiones para hacer y pedir sugerencias.
   \end{learningoutcomes}
\end{unit}

\begin{unit}{You`re pretty smart!}{}{Soars021S,Cambridge06,MacGrew99}{0}{2}
\begin{topics}
      \item Formas de Preguntas
      \item Adverbios y Adjetivos
      \item Vocabulario descripción de sentimientos
      \item Expresiones para viajes en tren 
      \item Redacción de historias cortas
      \item Lecturas.
   \end{topics}

   \begin{learningoutcomes}
      \item Al finalizar la décimo tercera unidad, los alumnos habiendo conocido los fundamentos de la estructuración de preguntas diversas, realizarán trabajos aplicativos en contextos adecuados. Enfatizan la diferencia entre adjetivos y adverbios. Describen sentimientos. Utilizan expresiones para coger un tren. Asumen la idea se sufijos y prefijos.
   \end{learningoutcomes}

\end{unit}

\begin{unit}{Have you ever?}{}{Soars021S,Cambridge06, MacGrew99}{0}{2}
   \begin{topics}
      \item Presente Perfecto
      \item Expresiones con never, ever y yet
      \item Vocabulario verbos en Participio pasado
      \item Expresiones para viajes en avión 
      \item Redacción de cartas de agradecimiento
      \item Lecturas
   \end{topics}

   \begin{learningoutcomes}
      \item Al finalizar la décimo cuarta unidad, los alumnos habiendo conocido los fundamentos de la estructuración del Presente Perfecto experimentan la necesidad de poder expresar este tipo de tiempo en acciones. Realizarán prácticas en contextos adecuados. Enfatizan la diferencia entre pasado simple y presente perfecto. Describen acciones con never, ever y yet. Utilizan expresiones para utilizar en un aeropuerto.
   \end{learningoutcomes}

\end{unit}

\begin{coursebibliography}
\bibfile{ForeignLanguages/ID101}
\end{coursebibliography}

\end{syllabus}
%\end{document}
