\begin{syllabus}

\curso{CM142. Cálculo Vectorial II}{Obligatorio}{CM142}

\begin{justification}
Se estudia los fundamentos de matrices para luego tratar determinantes, rango y sistemas lineales. Asimismo, se considera el estudio de valores y vectores propios con la finalidad de diagonalizar una matriz. Por otra parte, se trata espacios vectoriales, transformaciones lineales y formas cuadráticas para tener aplicación a las secciones cónicas.
\end{justification}

\begin{goals}
\item Al finalizar el curso el alumno comprenderá los fundamentos de Matrices y Determinantes, el cálculo de valores y vectores propios, espacios vectoriales en $R^n$ y las cuádricas.
\end{goals}

\begin{outcomes}
\ExpandOutcome{a}
\ExpandOutcome{i}
\ExpandOutcome{j}
\end{outcomes}

\begin{unit}{1. Matrices y Determinantes}{Hasser97,Burgos08}{12}
\begin{topics}
      \item Definición de Matrices. Notación. Igualdad de matrices. Transpuesta de una matriz.
      \item Matrices Especiales: Matriz cuadrada, nula. Matriz diagonal. Matriz  Escalar. Matriz Identidad. Matriz simétrica. Matriz antisimétrica. Matriz  triangular superior. Matriz triangular inferior.
      \item Operaciones con Matrices; Suma de matrices. Multiplicación de una matriz para escalar. Multiplicación de matrices.
	\item Potencia de una matriz. Inversa de una Matriz
	\item Determinantes: Definición para una matriz de orden 2. Definción para una matriz de orden 3
	\item Determinante de una matriz de orden nxn por menores y cofactores
	\item Propiedades de las determinantes
	\item Obtención de la Inversa de una Matriz
	\item Rango de una matriz. Definición
	\item Operaciones elementales por filas y por columnas. Matrices equivalentes
	\item Matriz Escalonada
	\item Determinación del rango de una matriz por operaciones elementales
	\item Determinación de la Inversa de una matriz por operaciones elementales
	\item Sistemas de Ecuaciones Lineales. Definición. Sistemas equivalentes
	\item Rango asociado a un sistema de Ecuaciones lineales. Condición necesaria y suficiente para su consistencia.
	\item Soluciones Independientes de un sistema de Ecuaciones lineales
	\item Regla de Cramer.
	\item Sistemas de Ecuaciones Lineales, homogéneas. Propiedades.
   \end{topics}

   \begin{unitgoals}
      \item Conocer las caracterTecnologíasticas de las matrices y el cálculo de determinantes
	\item Resolver problemas
   \end{unitgoals}
\end{unit}

\begin{unit}{2. Espacios Vectoriales}{Hasser97,Burgos08}{8}
\begin{topics}
	\item Espacio euclideano n-dimensional. n-ada ordenada. Operaciones: Suma, multiplicación por un escalar. Producto. Interior. Longitud de un vector.
	\item Espacio Vectorial tridimensional. Definición. Subespacio. Combinación Lineal. Generación de sub espacios.
      \item Independencia Lineal. Base y dimensión. Espacio fila y espacio columna de una matriz. Rango. Método para hallar bases
      \item Espacios con producto Interno. Desigualdad de Cauchy-Schwarz. Longitud de un vector. Ángulo entre vectores
      \item Bases ortonormales. Proceso de Gram-Schmidt. Coordenadas. Cambio de base. Rotación de los ejes coordenados en el plano, en el espacio. Matriz ortogonal.
    \end{topics}
   \begin{unitgoals}
      \item Describir matemáticamente el concepto de Espacio Vectorial
      \item Conocer y aplicar conceptos de espacios y bases
	\item Resolver problemas
   \end{unitgoals}
\end{unit}

\begin{unit}{3. Transformaciones Lineales}{Anton,Burgos08}{6}
\begin{topics}
      \item Definición. Algunos tipos de transformación lineal: Identidad, dilatación, contracción, rotación. Proyección. Núcleo e imagen.
      \item Transformaciones lineales de $R^n$ a $R^n$, de $R^n$ a $R^m$; geometría de la transformación lineal de $R^2$ a $R^2$.
      \item Matrices asociadas a las transformaciones lineales. Equivalencia. Semejanza.
\end{topics}

   \begin{unitgoals}
      \item Describir matemáticamente los transformaciones lineales
	\item Resolver problemas
   \end{unitgoals}
\end{unit}

\begin{unit}{4. Valores Propios y Vectores Propios}{Anton,Burgos08}{12}
\begin{topics}
      \item Valor propio y Vector propio. Definiciones. Polinomio caracterTecnologíastico. Propiedades.
      \item Diagonalización de una matriz. Definición. Condición, necesaria y suficiente para la diagonalización de una matriz.
      \item Diagonalización ortogonal. Matrices Simétricas.
	\item Formas Cuadráticas. Definición. Aplicación a las secciones cónicas en el plano y en el espacio.
	\end{topics}

   \begin{unitgoals}
      \item Conocer y aplicar conceptos de valores y vectores propios
	\item Resolver problemas
   \end{unitgoals}
\end{unit}

\begin{unit}{CINCO Cuádricas}{Granero85,Saal84}{12}
\begin{topics}
      \item Coordenadas Homogéneas en el Espacio.
	\item Ecuación paramétrica de la recta que pasa por dos puntos.
	\item Cuádrica. Su Ecuación general.
	\item Cuádricas degeneradas.
	\item Reducción de la ecuación general de las cuádricas a forma canónica.
	\item Estudio particular de cuádricas en forma canónica.
	\item Estudio particular de cuádricas en forma canónica. Superficie esférica. Elipsoide de revolución. Hiperboloides. Paraboloides.
	\item Intersección de las cuádricas con el plano X4 = 0
	\item Clasificación de las Cuádricas.
   \end{topics}

   \begin{unitgoals}
      \item Conocer y aplicar conceptos de ecuaciones cuádricas
	\item Resolver problemas
   \end{unitgoals}
\end{unit}

\begin{coursebibliography}
\bibfile{BasicSciences/CM142}
\end{coursebibliography}
\end{syllabus}
