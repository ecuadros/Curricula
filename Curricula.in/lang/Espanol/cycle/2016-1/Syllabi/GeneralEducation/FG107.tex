\begin{syllabus}

\course{FG107. Antropología Filosófica y Teológica}{Obligatorio}{FG107}

\begin{justification}
  Todos los hombres desean saber (Aristóteles, MetafTecnologíasica, I, 1). La aspiración natural de todo hombre por alcanzar la verdad y la sabiduría se encuentra desde los orTecnologíagenes mismos de la humanidad. Este saber se dirige de manera especial hacia el hombre mismo, porque la pregunta acerca de la verdad del hombre afecta a lo más Tecnologíantimo de la felicidad y destino humano". Con éstas palabras comienza José Angel García Cuadrado su obra Antropología Filosófica. Una introducción a la Filosofía del Hombre, y resume la fundamentación de este curso que, aunque apretado en cuanto al vasto conocimiento acerca del hombre, intenta proporcionar una sTecnologíantesis significativa de conocimientos y razonamientos que sirvan de base para responder a la pregunta sobre el ser humano.
\end{justification}

\begin{goals}
\item Ser capaz de comprender la naturaleza humana (es decir, las facultades y las finalidades de cada facultad, su jerarquización y posible dominio); la condición de persona humana y su dignidad; las consecuencias existenciales de dicha naturaleza y condición de persona humana (manifestaciones del ser persona: libertad, sociabilidad, sexualidad, cultura); y los datos relevantes de la antropología teológica que dan explican el misterio de la existencia humana y su fin trascendente (pecado original, redención, encarnación del Verbo, vida después de la muerte, resurrección, naturaleza y sobre-naturaleza).
\end{goals}

\begin{outcomes}
    \item \ShowOutcome{e}{3}
    \item \ShowOutcome{g}{3}
    \item \ShowOutcome{ñ}{3}
    \item \ShowOutcome{o}{3}
\end{outcomes}
\begin{competences}
    \item \ShowCompetence{C10}{g,ñ,o}
    \item \ShowCompetence{C20}{e,g,ñ,o}
    \item \ShowCompetence{C21}{e}
\end{competences}

\begin{unit}{}{Primera Unidad: Estatuto científico de la antropología filosófica}{GarciaCuadrado}{5}{C20}
\begin{topics}
	\item Presentación del curso
	\item La pregunta sobre el hombre y la importancia de la AntropologTecnologíaa
	\item Antropología filosófica y antropologTecnologíaas positivas. Delimitando el objeto: 
	\begin{subtopics}
		\item Antropología fTecnologíasica o natural (etnografía  paleo antropologTecnologíaa)
		\item Antropología cultural o social (etnologTecnologíaa)
		\item Psicología Moderna
		\item Limitaciones de la ciencia moderna y el mito del progreso
		\item Antropología filosófica: Objeto y definición
		\item Antropología teológica: Objeto y definición
		\item Antropología teológica: Objeto y definición	
	\end{subtopics}
	\item Planos metodológicos del estudio de la antropología filosófica. 
\end{topics}

\begin{learningoutcomes}
	\item Comprender la importancia del curso para la formación universitaria (personal y profesional).
Delimitar y definir la Antropología Filosófica y Teológica y sus respectivos objetos de estudio.
Comprender el método de estudio según los diversos planteamientos. [\Usage].

\end{learningoutcomes}
\end{unit}

\begin{unit}{}{Segunda Unidad: El hombre en el mundo natural}{GarciaCuadrado}{15}{C20}
\begin{topics}
	\item La jerarquía del mundo natural 
		\begin{subtopics} 
			\item Noción de vida: el alma
			\item CaracterTecnologíasticas de la vida
			\item Tipos de alma. 
		\end{subtopics}
	\item Semejanzas y diferencias con los vivientes: 
		\begin{subtopics} 
			\item Las operaciones básicas vitales
			\item La sensibilidad interna y externa
			\item Las tendencias sensibles: deseos (apetito concupiscible) e impulsos (apetito irascible).
		\end{subtopics}
	\item Afectividad humana:
		\begin{subtopics}
			 \item Pasiones humanas
			\item Educación de la afectividad.
		\end{subtopics}
	\item Diferencias especTecnologíaficas: 
		\begin{subtopics} 
			\item Inteligencia
			\item Voluntad
			\item La inmortalidad del alma humana.
		\end{subtopics}
\end{topics}
\begin{learningoutcomes}
	\item Analizar y comprender la naturaleza humana utilizando una metodología ascendente: Comprendiendo el fenómeno de la vida, las semejanzas y diferencias que tenemos con los vivientes (facultades) y las facultades superiores que nos otorgan nuestra diferencia especTecnologíafica en el mundo natural. [\Usage]
\end{learningoutcomes}
\end{unit}

\begin{unit}{}{Tercera Unidad: La persona humana: fundamento, dignidad y manifestaciones}{GarciaCuadrado,Melendo}{15}{C10, C20}
\begin{topics}
	\item Origen de la noción de persona.
	\item La fundamentación metafTecnologíasica de la persona humana.
	\item Otras aproximaciones a la fundamentación de persona.
	\item Dignidad de la persona humana.
		\begin{subtopics}
			\item Otras aproximaciones.
		\end{subtopics}
	\item El cuerpo humano.
	\item Manifestaciones de la persona humana.
	\item Manifestaciones persona humana (perspectiva dinámico-existencial de la naturaleza humana):
		\begin{subtopics}
			\item Persona y libertad:  La libertad: ser libre (ontológico) y la operatividad de la libertad (3 tipos de operatividad).
			\item Las relaciones interpersonales: El ser humano es social por naturaleza el amor y la amistad.
			\item Persona y sexualidad: Ser varón y ser mujer: Sexualidad y matrimonio, La cuestión homosexual.
		\end{subtopics}
\end{topics}
\begin{learningoutcomes}
	\item Analizar y comprender qué es significa ser «persona humana», desde los orTecnologíagenes de la noción hasta el aporte definitivo del cristianismo; comprender la fundamentación de la dignidad de la persona humana desde una perspectiva metafTecnologíasica y cristiana; comprender las manifestaciones de la persona humana a través de su naturaleza en un plano dinámico-existencial. [\Usage].
\end{learningoutcomes}
\end{unit}

\begin{unit}{}{Cuarta Unidad: La persona humana: aproximación existencial y fenomenológica}{GarciaQuesada,Frankl}{3}{C20}
\begin{topics}
	\item Aproximación existencial y fenomenológica a la persona humana.
	\item El sentido de la vida: fines objetivos y fines subjetivos.
	\item La necesidad de una misión particular.
	\item Quién soy? La perenne pregunta.
	\item Visión tripartita - fenomenológica de la persona humana: unidad biológica, psicológica y espiritual.
	\item Diferencia entre el yo psicológico y el yo personal.
		\begin{subtopics}
			\item Personalidad, mismidad e identidad.Persona y sexualidad: Ser varón y ser mujer.
			\item Sexualidad y matrimonio.
			\item La cuestión homosexual.
		\end{subtopics}
	\item Complementariedad entre las visiones dualista y tripartita de la persona.
		\begin{subtopics}
			\item Teocentrismo y Antropocentrismo 
		\end{subtopics}
	\item Dinamismos de la persona: permanencia y despliegue  necesidad de seguridad y significación.
\end{topics}
\begin{learningoutcomes}
	\item Presentar la antropología desde una perspectiva fenomenológica existencial. 
Analizar el aporte de dicho enfoque frente a los vistos anteriormente y frente el hombre moderno. [\Usage].
\end{learningoutcomes}
\end{unit}

\begin{unit}{}{Quinta Unidad: Destino de la Persona Humana - Antropología Teológica}{GarciaCuadrado}{3}{C20,C21}
\begin{topics}
 	 \item Finitud y trascendencia de la persona humana.
 	 \item El deseo de eternidad.
 	 \item Aproximación metafTecnologíasica desde la inmortalidad del alma humana.
 	 \item Aproximación existencial y fenomenológica desde la experiencia de finitud:
		\begin{subtopics}
			\item Nostalgia de infinito.
		\end{subtopics}	
 	 \item Antropología Teológica:
 	 	\begin{subtopics}
			\item El hombre como imagen de Dios.
			\item El pecado original y la redención.
			\item Resurrección.	
			\item Vida en Cristo (vida sobrenatural). 
		\end{subtopics}
\end{topics}
\begin{learningoutcomes}
	\item Comprensión ontológica del ser personal desde sus dinamismos fundamentales y otros dinamismos [\Usage].
	\item Comprensión de la fe, enraizada en el ser personal [\Usage].
\end{learningoutcomes}
\end{unit}

\begin{unit}{}{La persona: ser en relación}{GarciaQuesada,GarciaCuadrado,Congregacion,Edith}{6}{C22}
\begin{topics}
 	 \item Relacionalidad, encuentro y comunión en la persona.
 	 \item La persona como ser sexuado y como ser social.
 	 \item La familia.
 	 \item La sociedad.
\end{topics}
\begin{learningoutcomes}
	\item Comprensión de la persona desde su dimensión ontológica relacional con Dios, consigo mismo, con lo demás y con la Creación [\Usage].
\end{learningoutcomes}
\end{unit}



\begin{coursebibliography}
\bibfile{GeneralEducation/FG101}
\end{coursebibliography}

\end{syllabus}
