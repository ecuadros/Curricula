\begin{syllabus}

\course{CS331. Cloud Computing}{Obligatorio}{CS331}

\begin{justification}
La capacidad de procesamiento de una sola máquina es limitada y la Ley de Moore se ha encontrado 
con barreras antes de lo previsto, a pesar de esto la necesidad de mayor poder computacional es cresciente. 

El uso de las computadoras como elementos conectados entre sí­ es cada vez más común y cada vez en mayor escala, 
la capacidad de comunicación entre dispositivos (computadoras, celulares, pdas, etc.), abre las puertas 
a la existencia de una única plataforma donde la información de los usuarios
esté disponible siempre, sin importar el medio de acceso a esta (\textit{Cloud computing}).

La computación en la nube de internet o un grupo de computadores 
permite conseguir ambos objetivos, traspasando la barrera de una sola máquina para poder
integrar las capacidades de distintos dispositivos y permitirles interactuar en un entorno que
el usuario perciba como unificado; además, al conectarlos, el tope de desempeño
del sistema ya no es la capacidad de un sólo elemento (e.g. CPU) sino la cantidad de participantes en este,
por lo cual existe una escalabilidad del poder computacional muchísimo mayor.
\end{justification}

\begin{goals}
\item Comprender los conceptos básicos de la computación en nube, incluyendo definiciones, historia, pros y cons de la misma, comparaciones con tecnologías relacionadas, tales como grid computing, o utility computing.
\item Conocer la tecnología que soporta a la computación en nube.
\item Comprender la relación entre data-intensive applications y cloud computing, y
\item Evaluar el nuevo modelo de computación para conocer las tendencias de esta área emergente.
\end{goals}

\begin{outcomes}{V1}
\ExpandOutcome{a}{3}
\ExpandOutcome{b}{4}
\ExpandOutcome{c}{4}
\ExpandOutcome{d}{3}
\ExpandOutcome{i}{3}
\ExpandOutcome{j}{4}
\ExpandOutcome{k}{4}
\end{outcomes}

\begin{unit}{Introducción a cloud computing}{Armbrust:EECS-2009-28,nistcc}{7}{2}
   \begin{topics}
        \item \ARDistributedArchitecturesTopicNetwork%
        \item \SESpecializedSystemsTopicClient%
        \item \SESpecializedSystemsTopicDistributed%
        \item \SESpecializedSystemsTopicParallel%
        \item \SESpecializedSystemsTopicWeb%
   \end{topics}

   \begin{learningoutcomes}
        \item \ARDistributedArchitecturesObjSIX%
        \item \SESpecializedSystemsObjONE%
        \item \SESpecializedSystemsObjFIVE%
        \item Comprender como aparecio el paradigma de computación en nube.
   \end{learningoutcomes}
\end{unit}

\begin{unit}{Temas de investigación en cloud computing}{vaquero09,lmei2008}{8}{2}
   \begin{topics}
        \item Data Center Network Architecture
        \item Network Management
        \item Resource and Performance Management
        \item Data management
   \end{topics}

   \begin{learningoutcomes}
        \item Entender la relación entre los diferentes tipos de investigación que procedieron a la computación en nube.
        \item Conocer distintas líneas de investigación de computación en nube.
   \end{learningoutcomes}
\end{unit}

\begin{unit}{Cloud data management}{stonebrakersharednothing86,stonebrakerendofera07,claremont09}{10}{3}
   \begin{topics}
      \item \IMInformationModelsTopicInformationStorage%
      \item \IMInformationModelsTopicSearch%
      \item \IMInformationModelsTopicScalability%
      \item \IMDatabaseSystemsTopicDatabase%
      \item \IMDistributedDatabasesAllTopics%
      \item Big Data.
      \item Large small data.
      \item Bases de datos \textit{NoSQL}.
   \end{topics}

    \begin{learningoutcomes}
      \item \IMInformationModelsObjTWO%
      \item \IMInformationModelsObjSEVEN%
      \item \IMInformationModelsObjEIGHT%
      \item \IMInformationModelsObjNINE%
      \item \IMDistributedDatabasesObjTWO%
      \item Conocer diferentes casos de objetos distribuidas.
   \end{learningoutcomes}
\end{unit}


\begin{unit}{Data-intensive applications}{heystewart09,bryantdisc07,dean08}{10}{3}
    \begin{topics}
      \item Modelo de programación MapReduce.
      \item Ejemplos de aplicaciones en la academia y en la industria.
      \item Aplicaciones usando MapReduce.
      \item Otros lenguajes de programación para Cloud Computing.
   \end{topics}

   \begin{learningoutcomes}
      \item Entender el modelo de programación MapReduce.
      \item Conocer diferentes modos de uso de MapReduce. 
      \item \IMInformationModelsObjEIGHT%
      \item \IMInformationModelsObjNINE%
      \item \IMDistributedDatabasesObjTWO%
   \end{learningoutcomes}
\end{unit}


\begin{unit}{Programando para Cloud Computing}{dean08,Nurmi:2009:EOC:1577849.1577895,aws}{10}{3}
   \begin{topics}
      \item Usando Amazon Web Services.
      \item MapReduce en Amazon Web Services.
      \item Proveedores de Cloud Computing.
      \item Frameworks para crear servicios de Cloud Computing.
   \end{topics}

   \begin{learningoutcomes}
      \item Conocer los diferentes services de Amazon Web Services.
      \item Aplicar conocimientos de Cloud Computing para crear aplicaciones que usen otros servicios de Cloud Computing.
      \item Conocer los diferentes proveedores de servicios de Cloud Computing.
      \item Entender las similitudes y diferencias, ventajas y desventajas de los diferentes frameworks para crear \textit{private clouds}.
   \end{learningoutcomes}
\end{unit}



\begin{coursebibliography}
\bibfile{Computing/CS/CS331}
\end{coursebibliography}

\end{syllabus}
