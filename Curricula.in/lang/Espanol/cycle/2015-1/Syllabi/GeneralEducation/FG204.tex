\begin{syllabus}

\course{FG204. Teología I}{Obligatorio}{FG204}

\begin{justification}
La Universidad Católica San Pablo busca ofrecer una visión de la persona humana y del mundo iluminada por el Evangelio y, consiguientemente, por la fe en Jesucristo, centro de la creación y de la historia. El estudio de la teología es fundamental para dicha comprensión de Dios, del hombre y del mundo.

El curso de Teología I, o Teología Fundamental, introduce al estudiante a la experiencia de la comprensión sistemática y cientTecnologíafica de los fundamentos de la fe cristiana, tanto desde la Fe como desde las razones en las cuales se apoya el acto de creer, que le permitirán adentrarse en la comprensión de los contenidos y consecuencias del Dogma cristiano (Teología II).

La Teología permite la comprensión de la cosmovisión que ha forjado la cultura occidental en la cual el alumno ha nacido, vive y desarrolla su vida. Al creyente en Cristo, le permite conocer y comprender mejor su fe. Al no creyente, abrirse al conocimiento de Dios desde Jesucristo y su Iglesia.
\end{justification}

\begin{goals}
\item Estudiar y fundamentar el cristianismo en cuanto religión revelada desde las razones en las que se apoya mostrando su credibilidad. 
\end{goals}

\begin{outcomes}
\ShowOutcome{FH}{2}
\end{outcomes}

\begin{unit}{Homo Capax Dei. El hombre es capaz de Dios}{HomoCapax,Swinburne,DeLubac}{9}{4}
\begin{topics}
	\item El hombre: un ser inquieto en búsqueda.
	\item	La vía ascendente del hombre a Dios
		\subitem	La razón y el conocimiento de Dios.
		\subitem	La experiencia existencial.
		\subitem	La búsqueda religiosa.
	\item	Expresiones del espTecnologíaritu religioso.
	\item	La negación de Dios.
\end{topics}
\begin{learningoutcomes}
	\item Mostrar la \"hipótesis Dios\" como algo connatural al espTecnologíaritu humano y las consecuencias que de ello se derivan.
\end{learningoutcomes}
\end{unit}

\begin{unit}{Dios sale al encuentro del Hombre}{DeiVerbum,Catecismo,Latourelle}{9}{4}
\begin{topics}
      \item Dios habla al hombre
      \item Jesucristo: Plenitud de la revelación.
      \item Las Sagradas Escrituras
      \item La Tradición
      \item La sucesión apostólica
\end{topics}

\begin{learningoutcomes}
      \item Presentar al alumno la doctrina católica de la Revelación,  entendida como el camino descendente de Dios al hombre.
Mostrar las implicaciones que de dicha doctrina se derivan.
\end{learningoutcomes}
\end{unit}

\begin{unit}{Credo ut Intelligam. La Fe y la razón}{Catecismo,BenedictoRatisbona,JuanPabloA,SanAnselmo}{6}{4}
\begin{topics}
      \item El acto de creer como acto razonable
      \item Fe natural 
      \item Fides qua creditur: La Fe sobrenatural
      \item Fides quae creditur: El contenido de la Fe
      \item Fe y razón

\end{topics}

\begin{learningoutcomes}
      \item 
\end{learningoutcomes}
\end{unit}

\begin{unit}{Jesús de Nazaret}{JuanPabloII1998,Catecismo,AdamKarl,Benedicto2007,BiffiGiacomo}{15}{4}
\begin{topics}
      \item Quién es Jesús?
      \begin{inparaenum}
      \item Historicidad de Jesús de Nazaret.
      \item Jesús el MesTecnologíaas
      \item Jesús el Hijo del Hombre
      \item Jesús el Hijo de Dios
\end{inparaenum}
      \item Qué hizo Jesús?
      \begin{inparaenum}
      \item Testigo de la Verdad: El mensaje de Jesús
      \item Pasó haciendo el bien: Los milagros de Jesús
      \item La Resurrección
\end{inparaenum}
      \item La Fe de la Iglesia en Cristo
      \begin{inparaenum}
      \item Verdadero Dios: Logos
      \item Verdadero Hombre: La Encarnación
      \item Dios y hombre Verdadero: La unión Hipostática
      \item El Reconciliador
      \item El Señor
	\end{inparaenum}

\end{topics}

\begin{learningoutcomes}
      \item Presentar a Jesús de Nazaret como el Cristo, Plenitud de la revelación de Dios a los hombres.
\end{learningoutcomes}
\end{unit}

\begin{unit}{La Iglesia de Cristo}{JuanPabloII1998,VaticanoIILumen,Catecismo,RatzingerJoseph1992}{6}{2}
\begin{topics}
      \item Objeciones contra la Iglesia
      \item La Iglesia de Cristo
      \begin{inparaenum}
      \item Cristo funda la Iglesia. 
      \item La Iglesia Cuerpo de Cristo
      \item La Iglesia prolonga en la historia la presencia de Cristo.
      \item Sacramento Universal de Salvación
\end{inparaenum}
      \item Las notas de la Iglesia.
      \begin{inparaenum}
      \item Una
      \item Santa
      \item Católica 
      \item Apostólica
\end{inparaenum}     

\end{topics}

\begin{learningoutcomes}
      \item Presentar la naturaleza y misión de la Iglesia y su inseparable relación con Jesucristo.
\end{learningoutcomes}
\end{unit}



\begin{coursebibliography}
\bibfile{GeneralEducation/FG101}
\end{coursebibliography}

\end{syllabus}
