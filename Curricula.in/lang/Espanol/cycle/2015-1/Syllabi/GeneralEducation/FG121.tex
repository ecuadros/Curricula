\begin{syllabus}

\curso{HU121. Economía}{Obligatorio}{HU121}

\begin{justification}
La sociedad y sus componentes interactúan en un sistema económico que se caracteriza por ser de libre mercado, el mismo que ha demostrado ser el más eficiente en cuanto a la asignación de los recursos e incentivos para la inversión por parte de las empresas privadas. En tal sentido se hace necesario que los profesionales estudien y conozcan los principios fundamentales inherentes al funcionamiento de los mercados de tal manera de desarrollar una capacidad analítica que permita entender los fenómenos que se presentan diariamente dentro de su entorno. Asimismo el estudio de la economía permitiría entender la política económica adoptada por los diferentes gobiernos a lo largo de nuestra historia económica de tal manera de poder evaluarlas sobre la base de los resultados obtenidos.
\end{justification}

\begin{goals}
\item \OutcomeHU
\end{goals}

\begin{outcomes}
\ExpandOutcome{HU}
\end{outcomes}

\begin{unit}{Introduccion}{Case93}{16}
\begin{topics}
	\item Conceptos generales
	\item Metodología de la ciencia económica
	\item Descripción general de la micro y macroeconomía
\end{topics}

\begin{learningoutcomes}
      \item Desarrollar la capacidad de analizar situaciones propias de la gestión con un enfoque económico y de eficiencia
   \end{learningoutcomes}
\end{unit}

\begin{unit}{La teoria de la demanda}{Fernandez99}{16}
\begin{topics}
	\item  Utilidad, presupuesto y consumo óptimo del consumidor.
	\item Bienes normales, inferiores, sustitutos y complementarios.
 	\item Caso: Indice de Precios del Consumidor
\end{topics}

\begin{learningoutcomes}
      \item Aprender a utilizar las herramientas básicas de la ciencia económica en las diferentes situaciones profesionales que serán afrontadas por el alumno
   \end{learningoutcomes}
\end{unit}


\begin{unit}{Teoría de la producción y de los costos económicos de corto plazo}{Blanchard00}{16}
\begin{topics}
	\item Función de producción de corto plazo.
	\item Rendimientos marginales decrecientes.
	\item Costos de corto plazo
	\item Relación entre la producción y los costosConceptos generales
\end{topics}

\begin{learningoutcomes}
      \item Desarrollar la capacidad de entendimiento de las políticas económicas que normalmente adoptan los gobiernos de turno
   \end{learningoutcomes}
\end{unit}

\begin{coursebibliography}
\bibfile{GeneralEducation/FG121}
\end{coursebibliography}
\end{syllabus}





