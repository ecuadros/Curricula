\begin{syllabus}

\course{CS107. Álgebra Abstracta}{Obligatorio}{CS107}

\begin{justification}
El álgebra abstracta tiene un lado práctico que explotaremos para
comprender en profundidad temas de computación como criptografía y
álgebra relacional.
\end{justification}

\begin{goals}
\item Conocer las técnicas y métodos de encriptación de datos.
\end{goals}

\begin{outcomes}
\ExpandOutcome{a}{4}
\ExpandOutcome{b}{3}
\ExpandOutcome{j}{3}
\end{outcomes}

\begin{unit}{\ALCryptographicAlgorithmsDef}{Grimaldi97, Scheinerman01}{20}{3}
    \ALCryptographicAlgorithmsAllTopics
    \ALCryptographicAlgorithmsAllObjectives
\end{unit}

%GLA Unidad eliminada a pedido del docente
%begin{unit}{\IMRelationalDatabasesDef}{Grassmann97}{20}{3}
%   \begin{topics}
         %\item \IMRelationalDatabasesTopicMapeo
%         \item \IMRelationalDatabasesTopicEntity
%         \item \IMRelationalDatabasesTopicRelational
%   \end{topics}

%   \begin{learningoutcomes}
         %\item \IMRelationalDatabasesObjONE
         %\item \IMRelationalDatabasesObjTWO
%         \item \IMRelationalDatabasesObjTHREE
%         \item \IMRelationalDatabasesObjFOUR
%         \item \IMRelationalDatabasesObjFIVE
%   \end{learningoutcomes}
%\end{unit}

\begin{unit}{Teoría de Números}{Grimaldi97, Scheinerman01}{20}{3}
   \begin{topics}
      \item Teoría de los números
      \item Aritmética  Modular
      \item Teorema del Residuo Chino
      \item Factorización
      \item Grupos, teoría de la codificación y método de enumeración de Polya
      \item Cuerpos finitos y diseños combinatorios
   \end{topics}

   \begin{learningoutcomes}
      \item Establecer la importancia de la teoría de números en la criptografía
      \item Utilizar las propiedades de las estructuras algebraicas en el estudio de la teoría algebraica de códigos
   \end{learningoutcomes}
\end{unit}



\begin{coursebibliography}
\bibfile{Computing/CS/CS105}
\end{coursebibliography}

\end{syllabus}

%\end{document}
