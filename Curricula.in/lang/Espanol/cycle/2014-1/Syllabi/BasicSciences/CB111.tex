\begin{syllabus}

\course{CB111. Física Computacional}{Obligatorio}{CB111}

\begin{justification}
Física Computacional es un curso que le permitirá al estudiante entender
las leyes de física de macropartículas y micropartículas considerado desde un
punto material hasta un sistemas de partículas; debiéndose tener en cuenta que los
fenómenos aquí estudiados van desde la mecánica clásica hasta la mecánica cuántica;Cinemática, Dinámica, Trabajo y Energía, Termodinámica, Fluidos, Oscilaciones, Electrodinámica y Física Cuánticas; 
además se debe asociar que éstos problemas deben ser resueltos con algoritmos computacionales.

Poseer capacidad y habilidad en la interpretación de problemas clásicos y cuánticos 
con condiciones de frontera reales que contribuyen en la elaboración de soluciones eficientes
y factibles en diferentes áreas de la Ciencia de la Computación.
\end{justification}

\begin{goals}
\item Identificar los principios que rigen la materia.
\item Utilizar las leyes físicas para la solución de problemas.
\item Aplicar la simulación a sistemas físicos.
\end{goals}

\begin{outcomes}
\ExpandOutcome{a}{3}
\ExpandOutcome{i}{3}
\ExpandOutcome{j}{4}
\end{outcomes}

\begin{unit}{FI1 Fundamentos de Física y Algebra vectorial}{ComputationalPhysics2007,ComputationalPhysics1999,Serway2009,Alonso94}{6}{3}
\begin{topics}
      \item Introducción.
      \item Naturaleza de la Física.
      \item Relación de la física con las ciencias básicas y aplicadas.
      \item Modelo idealizado.
      \item Magnitudes físicas  elementales.
      \item Propiedades de los vectores.
      \item Componentes de un vector y vectores unitarios.
      \item Producto de vectores.
      \item Ejercicios y problemas.
   \end{topics}

   \begin{learningoutcomes}
      \item Entender y trabajar con las magnitudes físicas del SI.
      \item Abstraer de la naturaleza los conceptos físicos rigurosos y
      representarlos en modelos vectoriales.
      \item Entender y aplicar los conceptos vectoriales a problemas físicos reales.
   \end{learningoutcomes}
\end{unit}

\begin{unit}{FI2 Cinemática}{ComputationalPhysics2007,ComputationalPhysics1999,OptimizationAlgorithmsinPhysics2002,Physicsforcomputersciencestudents1998,Sears2007,Serway2009}{6}{3}
\begin{topics}
      \item Velocidad y Aceleración Instantánea.
      \item Interpretación algebraico y geométrico
      \item Caída Libre.
      \item Movimiento Compuesto.
      \item Movimiento Circular.
      \item Aplicación con POO
      \item Ejercicios y problemas.
    \end{topics}
   \begin{learningoutcomes}
      \item Describir matemáticamente el movimiento mecánico de una partícula unidimensional como un cuerpo de dimensiones despreciables.
      \item Conocer y aplicar conceptos de magnitudes cinemáticas.
      \item Describir el comportamiento de movimiento de partículas, teórica y graficamente.
      \item Conocer representaciones vectoriales de estos movimientos unidimensionales.
      \item Resolver problemas.
   \end{learningoutcomes}
\end{unit}

\begin{unit}{FI3. Dinámica}{ComputationalPhysics2007,ComputationalPhysics1999,OptimizationAlgorithmsinPhysics2002,Physicsforcomputersciencestudents1998,Sears2007,Serway2009}{6}{3}
\begin{topics}
      \item Fuerzas e interacciones.
      \item Masa inercial.
      \item Peso.
      \item Condiciones de Equilibrio.
      \item Leyes de Newton
      \item Dinámica del movimiento compuesto.
      \item Aplicación de las leyes de Newton.
      \item Aplicación con POO.
      \item Ejercicios y problemas.
   \end{topics}

   \begin{learningoutcomes}
      \item Conocer los conceptos de fuerza.
      \item Conocer las interacciones de la materia a través de la inercia.
      \item Conocer los conceptos de equilibrio.
      \item Conocer y aplicar las leyes de Newton.      
      \item Conocer y aplicar las leyes de la dinámica lineal y circular.
      \item Resolver problemas.
   \end{learningoutcomes}
\end{unit}

\begin{unit}{FI4 Trabajo y Energia}{ComputationalPhysics2007,ComputationalPhysics1999,Physicsforcomputersciencestudents1998,Sears2007,Serway2009}{6}{3}
\begin{topics}
      \item Trabajo realizado por una fuerza constante.
      \item Trabajo realizado por fuerzas variables.
      \item Trabajo y energía cinética.
      \item Potencia.
      \item Energía potencial gravitatoria.
      \item Energía potencial elástica.
      \item Fuerzas conservativas y no conservativas.
      \item Principios de conservación de la energía.
      \item Ejercicios y problemas.
   \end{topics}

   \begin{learningoutcomes}
      \item Establecer los conceptos de trabajo y energía.
      \item Conocer tipos de energía.
      \item Establecer la relación energía convencional y no convencional.
      \item Conocer y aplicar los conceptos de conservación de energía.
      \item Resolver problemas.
   \end{learningoutcomes}
\end{unit}

\begin{unit}{FI5 Momento lineal}{ComputationalPhysics2007,ComputationalPhysics1999,Physicsforcomputersciencestudents1998,Sears2007,Serway2009}{6}{3}
\begin{topics}
      \item Momento lineal.
      \item Conservación del momento lineal.
      \item Centro de masa y de gravedad.
      \item Movimiento de un sistema de partículas.
      \item Ejercicios y problemas.
  \end{topics}

   \begin{learningoutcomes}
      \item Establecer los conceptos de momento lineal.
      \item Conocer los conceptos de conservación del momento lineal.
      \item Conocer el momento de un sistema de partículas.
      \item Resolver problemas.
   \end{learningoutcomes}
\end{unit}

\begin{unit}{FI6 Fluidos y Transferencia de Calor}{ComputationalPhysics2007,SolvingPDEsinC++2006,AnIntroductiontoComputerSimulationMethods:ApplicationstoPhysicalSystems2006,Sears2007,Serway2009}{6}{3}
\begin{topics}
      \item Estática de Fluidos.
      \item Dinámica de fluidos. 
      \item Viscosidad.
      \item Ejercicios y problemas.
  \end{topics}

   \begin{learningoutcomes}
      \item Conocer los conceptos y principios que rigen a los fluidos.
      \item Conocer el movimiento de fluidos 
      \item Resolver problemas.
   \end{learningoutcomes}
\end{unit}

\begin{unit}{FI7 Termodinámica}{ComputationalPhysics2007,AnIntroductiontoComputationalPhysics2006,ComputationalContinuumMechanics2008,Physicsforcomputersciencestudents1998,Sears2007,Serway2009}{6}{3}
\begin{topics}
      \item Calor y Temperatura.
      \item Leyes de la Termodinámica.
      \item Transferencia de calor.
      \item Ecuación del Calor.
      \item Ejercicios y problemas.
  \end{topics}

   \begin{learningoutcomes}
      \item Establecer los conceptos de temperatura.
      \item Comprender las leyes de la termodinámica.
      \item Conocer los conceptos de transferencia de calor.
      \item Resolver problemas.
   \end{learningoutcomes}
\end{unit}

\begin{unit}{FI8 Movimiento Oscilatorio y Ondulatorio}{ComputationalPhysics2007,ASurveyofComputationalPhysics:IntroductoryComputationalScience2008,Sears2007}{8}{3}
\begin{topics}
      \item Movimiento armónico simple
      \item Sistema masa - resorte.
      \item El péndulo.
      \item Movimiento amortiguado
      \item Resonancia
      \item Ondas mecánicas.
      \item Resolver problemas.
  \end{topics}

   \begin{learningoutcomes}
      \item Establecer los conceptos de oscilación.
      \item Conocer los sistemas amortiguados.
      \item Conocer fenómenos de resonancia.
      \item Analizar las diferentes magnitudes que intervienen en el movimiento ondulatorio para su aplicación a variados casos
      \item Resolver problemas.
   \end{learningoutcomes}
\end{unit}



\begin{coursebibliography}
\bibfile{BasicSciences/CB111}
\end{coursebibliography}

\end{syllabus}
