\begin{syllabus}

\course{FG108. Antropología Cultural}{Obligatorio}{FG108}

\begin{justification}    
El presente curso busca introducir al estudiante en la metodología y los temas fundamentales del 
objeto de estudio de la Antropología Cultural. Aprender a conocer al hombre a través de sus 
creaciones culturales, comprender su dimensión cultural, su actitud de apertura y 
tolerancia hacia otra culturas.
\end{justification}

\begin{goals}
\item Formar en el estudiante la capacidad de observación y análisis del comportamiento humano a lo largo de la historia  
\item Comprender las lógicas de funcionamiento de los elementos culturales propios de las sociedades y cosmovisiones de los pueblos
\end{goals}

\begin{outcomes}
\ExpandOutcome{FH}{2}
\end{outcomes}

\begin{unit}{Antropología y Cultura}{Herskovitz84}{12}{2}
   \begin{topics}
      \item La Antropología como disciplina
	\item Teoría y Método de la Antropología Cultural
	\item El concepto de cultura. La Cultura Ideal y la Cultura Real
	\item El contexto cultural del comportamiento humano. Exclusión e Inclusión. Justicia y Violencia.
	\item Manifestaciones de la Violencia
   \end{topics}

   \begin{learningoutcomes}
      \item Definir el objeto de estudio y comprender la importancia de la Antropología Cultural como una disciplina indispensable para poder obtener conocimiento y capacidad de análisis del comportamiento humano
   \end{learningoutcomes}
\end{unit}

\begin{unit}{Cultura, Sociedad e Individuo}{Nanda82}{10}{2}
   \begin{topics}
      \item El hombre y la Cultura
	\item Aprendizaje de la Cultura. Transculturación, aculturación y endoculturación
	\item Rango y estratificación social
	\item Matrimonio, Familia y grupos domésticos. Concepción Machista. Feminismo en el siglo XX
	\item Parentesco asociación. Las Relaciones con la Familia
   \end{topics}

   \begin{learningoutcomes}
      \item Explicar la importancia de la cultura en la comprensión del hombre como ser social
   \end{learningoutcomes}
\end{unit}

\begin{unit}{Expresión Simbólica}{Marvin97}{10}{2}
   \begin{topics}
      \item Cultura y Visión del mundo. Cultura de Paz
	\item Expresiones Culturales: Lenguaje y Arte
	\item Religión, Fenómeno Religioso, Mito, Ritual
	\item La Medicina Tradicional y la enfermedad
	\item Cultura, Religión y Salud
   \end{topics}

   \begin{learningoutcomes}
      \item Comprender y analizar los distintos tipos de expresiones culturales como aspecto importante e inseparable del ser humano
   \end{learningoutcomes}
\end{unit}

\begin{coursebibliography}
\bibfile{GeneralEducation/FG108}
\end{coursebibliography}
\end{syllabus}
