\begin{syllabus}

\course{FG101. Comunicación}{Obligatorio}{FG101}

\begin{justification}
La institución en su Proyecto Educativo señala la importancia de la Formación Humana de sus alumnos, que mejor vehículo para contribuir a este objetivo que el curso de Comunicación, que contribuye al desarrollo y perfeccionamiento de las capacidades comunicativas, del alumno a partir de la construcción de significados. Estos aprendizajes se constituyen en base fundamental para introducir a los educandos en su realidad cultural y profesional.
\end{justification}

\begin{goals}
\item Desarrollar su capacidad de comunicación y su sentido crítico y reflexivo
\item Reconocer estructuras básicas en comunicación, capacitando al alumno en el uso correcto de la gramática castellana, ortografía y lexicología.
\item Identificar la realidad del alumno, facilitando habilidades y destrezas en la expresión oral y escrita.
\item Desarrollar su sensibilidad para apreciar la función estética del lenguaje.
\end{goals}

\begin{outcomes}
\ExpandOutcome{d}{3}
\ExpandOutcome{f}{3}
\end{outcomes}

\begin{unit}{La Comunicación y la Lengua}{Cassany, Caceres, Cisneros}{9}{2}
\begin{topics}
      \item Fundamentos del Proceso Comunicativo: la comunicación, proceso, elementos y clases. 
      \item La Lengua como medio esencial de comunicación: La Lengua y las unidades que  la conforman, los signos linüísticos.
      \item La lengua oral y escrita: Sistema, Norma y Habla. 
      \item Las funciones del lenguaje en el proceso comunicativo: Informativa (representación de la realidad), persuasiva (apelación al oyente), expresiva (la expresión del hablante).
      \item La riqueza lónica y su importancia en la comunicación y en la formación profesional.
\end{topics}
\begin{learningoutcomes}
   \item Reconocer, definir y aplicar la naturaleza de la Comunicación como factor fónico de las relaciones sociales y profesionales.
   \item Reconocer y apreciar la Lengua como medio esencial de la comunicación,a partir del análisis de sus unidades.
   \item Diferenciar las características del código linüístico oral y escrito,
   como medio primordial de comunicación.
   \item Distinguir, valorar y aplicar las funciones del lenguaje como factor del conocimiento de la realidad y de las relaciones sociales.
   \item Incrementar su vocabulario a través de la investigación y consignación de palabras propias de la carrera elegida.
\end{learningoutcomes}

\end{unit}

\begin{unit}{El estudio de las estructuras linüísticas}{Cassany,Caceres,Cisneros,Hockett,Leahey,Saussure,Alonso}{9}{3}
\begin{topics}
   \item Relaciones entre los signos linüísticos: relaciones paradigmáticas y sintácticas.
   \item La estructura del discurso: relaciones globales en el texto, intencionalidad y propósito del autor y del texto. 
   \item La estructura del párrafo: relaciones pragmáticas, representación esquemática de las relaciones. 
   \item La Estructura de la Oración: Relaciones lineales, estructuras nominales y verbales (sintagmas).
   \item Funciones de las palabras en la oración: conectores lógicos y referencias.
   \item Operaciones de expansión, supresión, sustitución y cambio.
   \item Criterios de corrección y ejemplaridad idiomática: acento y entonación, la morfología, errores a evitar, la sintaxis: solecismos, corrección ortográfica: tildes, signos de puntuación y letras de escritura dudosa.
\end{topics}

\begin{learningoutcomes}
   \item Desarrollar la capacidad de abstracción y de relación a través del análisis de las estructuras linüísticas.
   \item Diferenciar las relaciones lógicas entre las ideas presentadas en los textos.
   \item Construir en orden lógico, diversos tipos de textos empleando ideas principales y secundarias.
   \item Analizar los contextos oracionales como elemento base de una expresión completa.
   \item Desarrollar la competencia linüística mediante la permanente ejercitación ortográfica, morfológica y sintáctica.
\end{learningoutcomes}
\end{unit}

\begin{unit}{La Lectura como Comunicación Escrita}{Cassany,Caceres,Cisneros,Hockett,Leahey,Saussure,Alonso}{9}{3}
\begin{topics}
   \item La asimilación de la lectura: El acto de la lectura: Comunicación entre el texto y el lector. 
   \item El placer de leer.  
   \item El proceso de la lectura: Estrategias de comprensivo, interpretación y comentario de textos.
   \item Clases de textos: narrativos, informativos y argumentales. 
   \item Recursos de planificación y organización para la lectura de textos.
   \item El resumen, la recensión esquemónica o gráfica. El resumen y la recensión en prosa.
   \item La corrección de la Forma (ortográfica) y del Fondo (redacción).
\end{topics}
\begin{learningoutcomes}
   \item Aplicar y analizar las estructuras linüísticas, con el fin de asimilar y valorar textos.
   \item Descubrir y manejar las diferentes estructuras de distintos discursos escritos.
   \item Alcanzar el manejo adecuado de los procesos lógicos de síntesis,
   \item Demostrar preocupación por la fase correctiva de la producción de textos.
\end{learningoutcomes}
\end{unit}

\begin{unit}{La Redacción como comunicación escrita por uno mismo}{Cassany,Caceres,Cisneros,Hockett,Leahey,Saussure,Alonso}{9}{3}
\begin{topics}
   \item La producción de textos: el proceso de redacción, fases y criterios.
   \item Aspectos redaccionales semánticos: claridad, coherencia, propiedad e integridad.
   \item Los documentos de redacción comercial o administrativa: Informe, Memorando, Solicitud, Comunicado, oficio, etc.
\end{topics}
\begin{learningoutcomes}
   \item Mejorar su redacción, formando en cuanta pauta y normas señaladas.
   \item Elaborar textos empleando micro estructuras.
   \item Iniciarse y/o perfeccionarse en la redacción del tipo administrativo o comercial.
\end{learningoutcomes}
\end{unit}

\begin{unit}{El discurso Oral}{Cassany,Caceres,Cisneros,Hockett,Leahey,Saussure,Alonso}{9}{3}
\begin{topics}
   \item La comunicación oral en grupo pequeño: características y cualidades en la conversación, el lenguaje no  verbal.
   \item La comunicación en público: manejo del auditorio (tensión), argumentación.
   \item Diferentes tipos de producciones orales: Discursos informativos, Discursos argumentativos.
   \item Formas oratorias: Tipo conferencia, Tipo deliberativo: debate, panel, mesa redonda.
\end{topics}
\begin{learningoutcomes}
   \item Distinguir entre una comunicación oral en grupo pequeño y en público, aprendiendo a manejar la persuasión, a través de la argumentación en el discurso oratorio.
   \item Reconocer y discriminar las ideas que estructuran los diferentes discursos oratorios.
   \item Evidenciar y valorar las producciones orales, demostrando respeto y tolerancia por el emisor y su mensaje.
   \item Producir diferentes discursos oratorios, aplicando las formas en que pueden estructurarse.
\end{learningoutcomes}
\end{unit}



\begin{coursebibliography}
\bibfile{GeneralEducation/FG101}
\end{coursebibliography}

\end{syllabus}
