\begin{syllabus}

\course{FG204A. Teología II}{Obligatorio}{FG204A}

\begin{justification}
Para que la formación de un buen profesional no se desligue ni se oponga sino mas bien contribuya al autentico crecimiento personal requiere de la asimilación de valores sólidos, horizontes espirituales amplios y una visión profunda del entorno cultural. La fe cristiana es uno de los elementos fundamentales de la configuración de la vida y el que hacer cultural de nuestro país, propone los más altos valores humanos y el horizonte espiritual mas rico posible.
Por esta razón dar una clara y explicita formación cristiana es indispensable.  En este curso tratamos de brindar al alumno el conocimiento básico de las razones.
\end{justification}

\begin{goals}
\item \OutcomeFH
\end{goals}

\begin{outcomes}
\ExpandOutcome{FH}{2}
\end{outcomes}

\begin{unit}{Teología fundamental}{Catecismo,Ratzinger,Biblia}{15}{2}
\begin{topics}
	\item La demanda. 
	      \begin{inparaenum}
		    \item Presentación del curso.
		    \item Análisis de la Felicidad.
		    \item Análisis del Amor.
		    \item Esquema de antropología.
	      \end{inparaenum}
	\item Ofertas intramundanas. 
	      \begin{inparaenum}
		    \item Hedonismo.
		    \item Materialismo.
		    \item Individualismo.
		    \item Ideologías de dominio.
	      \end{inparaenum}
	\item La respuesta cristiana (Parte I). 
	      \begin{inparaenum}
		    \item Conversión
		    \item Fe y Razón
	      \end{inparaenum}
\end{topics}
\begin{learningoutcomes}
	\item Que el alumno interprete las manifestaciones de su entorno cultural concreto a la luz de los elementos fundamentales de la persona humana.
	\item Que el alumno descrubra existencialmente la insuficiencia de las respuestas del mero poder, tener o placer y al mismo tiempo se proponga que es lo que realmente anhela en la vida.
	\item Que el alumno descubra la fe cristiana como una respuesta a los anhelos mas profundos del ser humano que ha creado toda una cultura y que está en la base de nuestra vida nacional.
\end{learningoutcomes}
\end{unit}

\begin{unit}{Teología dogmática}{Catecismo,Ratzinger,Biblia}{15}{2}
\begin{topics}
	\item La respuesta cristiana (Parte II). 
	      \begin{inparaenum}
		  \item Dios Unitrino.
		  \item Doctrina de la creación.
		  \item El pecado original.
	      \end{inparaenum}
	 \item La respuesta cristiana (Parte III). 
	      \begin{inparaenum}
		  \item Identidad y misión de Jesús de Nazareth.
		  \item Vida y obra de Jesús de Nazareth.
		  \item La gracia
		  \item El Espíritu Santo.
		  \item La Iglesia Una, Santa, Católica y Apostólica.
		  \item La Virgen María y la Vida Cristiana.
	      \end{inparaenum}
\end{topics}
\begin{learningoutcomes}
	\item Que el alumno tenga pleno conocimiento de lo que significa la palabra de Dios y de las formas en las que él se relaciona con nosotros con el fin de que reafirme y fortalezca su fé, enfrente dificultades y asuma el compromiso al que Dios lo invita.
\end{learningoutcomes}
\end{unit}

\begin{unit}{Teología moral y espiritual}{Catecismo,Ratzinger,Biblia}{15}{2}
\begin{topics}
      \item Fundamentos de teología moral.
      \item Fundamentos de teología espiritual.
      \item La vida espiritual.
      \item La vida de oración.
      \item El combate espiritual.
\end{topics}

\begin{learningoutcomes}
      \item Que el alumno comprenda la necesidad de una vida espiritual que le permita encontrar el sentido de su vida.
\end{learningoutcomes}
\end{unit}



\begin{coursebibliography}
\bibfile{GeneralEducation/FG204}
\end{coursebibliography}

\end{syllabus}
