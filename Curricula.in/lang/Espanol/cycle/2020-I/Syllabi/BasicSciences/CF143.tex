\begin{syllabus}

\course{CF143. Física III}{Obligatorio}{CF143} % Common.pm

\begin{justification}
El propósito del curso es que el alumno desarrolle la capacidad de usar los conceptos teóricos y aplicar estrategias adecuadas para la resolución de problemas de ingeniería, además de desarrollar su pensamiento crítico para analizar los resultados obtenidos, y que los interprete en contexto de aplicación a problemas reales. Para dar la solución debe utilizar el vocabulario propio. El alumno debe consultar bibliografiaTecnologia en inglés y ser capaz de emitir juicios fundamentados pertinentes al tema. 
\end{justification}

\begin{goals}
\item Al término del curso, el estudiante resolverá problemas sobre cargas eléctricas, campo eléctrico, ley de Gauss, potencial eléctrico, energía eléctrica y condensadores; además de problemas sobre cargas y corrientes bajo la presencia de campos magnéticos. Asimismo, el estudiante comprenderá y resolverá problemas relacionados con las leyes de Ampère, Biot-Savart, Faraday y de Lenz, y los relacionados con autoinductancia y energía magnética. El estudiante también aprenderá a resolver problemas de circuitos de corriente continua y alterna, así como los problemas referentes a ondas electromagnéticas (OEM) en el vacío.
\end{goals}

\begin{outcomes}
\item \ShowOutcome{a}{2}
\item \ShowOutcome{i}{2}
\end{outcomes}

\begin{competences}
    \item \ShowCompetence{C1,C20}{a,i}
\end{competences}

\begin{unit}{CAPÍTULO 1. Electrostática}{}{Sears2013,Serway2008}{20}{C1,C20}
\begin{topics}
      \item Introducción. Carga eléctrica.
      \item Conductores y aislantes.
      \item Ley de Coulomb.
      \item Fuerza eléctrica y campo eléctrico: principio de superposición en distribución de carga discreta y continua.
      \item Flujo de campo eléctrico y Ley de Gauss. Aplicaciones.
      \item Potencial eléctrico: potencial de una distribución de cargas discretas y continuas, campo eléctrico y potencial
      \item Energía electrostática.
      \item Condensadores: definición y capacidad, condensadores en serie y paralelo, condensadores de placas paralelas con dieléctricos.
   \end{topics}

\end{unit}

\begin{unit}{CAPÍTULO 2. Corriente continua}{}{Sears2013,Serway2008}{8}{C1,C20}
\begin{topics}
      \item Corriente eléctrica, resistencia y ley de Ohm.
      \item Fuerza electromotriz
      \item Circuitos de corriente continua: resistencias en serie y paralelo, leyes de Kirchhoff. 
      \item Circuito RC: carga y descarga.
    \end{topics}

\end{unit}

\begin{unit}{CAPÍTULO 3. Campo magnético}{}{Sears2013,Serway2008}{8}{C1,C20}
\begin{topics}
      \item Campo magnético y fuerza de Lorentz.
      \item Fuerza magnética sobre un conductor con corriente.
      \item Torque sobre una espira con corriente.
      \item Aplicaciones: espectrómetro de masas, y motor eléctrico.
      \item La Ley de Biot-Savart. 
      \item Ley de Ampère. Aplicaciones.
      \item Flujo magnético.
      \item Ley de Gauss en el magnetismo.
\end{topics}

\end{unit}

\begin{unit}{CAPÍTULO 4. Ley de Faraday e inductancia}{}{Sears2013,Serway2008}{8}{C1,C20}
\begin{topics}
      \item Ley de inducción de Faraday.
      \item Ley de Lenz. Aplicaciones.
      \item Autoinductancia.
      \item Inducción mutua.
      \item Circuito RL en corriente continua.
      \item Energía de una autoinductancia.
      \item Densidad de energía magnética.
      \item Aplicaciones.
   \end{topics}

\end{unit}

\begin{unit}{CAPÍTULO 5. Corriente alterna}{}{Sears2013,Serway2008}{8}{C1,C20}
\begin{topics}
	\item Generador de corriente alterna.
	\item Circuitos de corriente alterna con una resistencia, inductancia y capacitancia.
	\item Reactancia. Impedancia.
	\item Fasores.
	\item Circuito RLC en serie y en paralelo.
           \item Resonancia.
           \item Valores medios y eficaces.
           \item Potencia en corriente alterna.
           \item Transformador ideal.
   \end{topics}

\end{unit}

\begin{unit}{CAPÍTULO 6.Ondas electromagnéticas}{}{Sears2013,Serway2008}{4}{C1,C20}
   \begin{topics}
	\item La Ley de Ampère generalizada.
	\item Ecuaciones de Maxwell.
	\item Ecuaciones de Maxwell en el espacio libre.
	\item Ondas electromagnéticas planas.
	\item Vector de Poynting.
	\item Presión de radiación.
	\item El espectro electromagnético.
   \end{topics}
\end{unit}



\begin{coursebibliography}
\bibfile{BasicSciences/CF143}
\end{coursebibliography}

\end{syllabus}
