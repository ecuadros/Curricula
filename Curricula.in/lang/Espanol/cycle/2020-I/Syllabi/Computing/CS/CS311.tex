\begin{syllabus}

\course{CS3101. Programación Competitiva}{Obligatorio}{CS3101}
% Source file: ../Curricula.in/lang/Espanol/cycle/2020-I/Syllabi/Computing/CS/CS311.tex

\begin{justification}
    La Programación Competitiva combina retos de solucionar problemas con el añadido de poder competir con otras personas. Enseña a los participantes a pensar más rápido y desarrollar habilidades para resolver problemas, que son de gran demanda en la industria. 
    Este curso enseñará la resolución de problemas algorítmicos de manera rápida combinando la teoría de algoritmos y estructuras de datos con la práctica la solución de los problemas.
  \end{justification}
  
  \begin{goals}
    \item Que el alumno utilice técnicas de estructuras de datos y algoritmos complejos.
    \item Que el alumno aplique los conceptos aprendidos para la aplicación sobre un problema real.
    \item Que el alumno investigue la posibilidad de crear un nuevo algoritmo y/o técnica nueva para resolver un problema real.
  \end{goals}
  
  \begin{outcomes}{V1}
      \item \ShowOutcome{a}{2}
      \item \ShowOutcome{b}{2}
      \item \ShowOutcome{h}{2}
  \end{outcomes}
  
  \begin{outcomes}{V2}
      \item \ShowOutcome{1}{2}
      \item \ShowOutcome{2}{2}
      \item \ShowOutcome{4}{2}
      \item \ShowOutcome{5}{2}
  \end{outcomes}
  
  \begin{competences}{V1}
      \item \ShowCompetence{C1}{a,b}
      \item \ShowCompetence{C2}{a,b}
      \item \ShowCompetence{C24}{h}
  \end{competences}
  
  \begin{competences}{V2}
      \item \ShowCompetence{C1}{1,2}
      \item \ShowCompetence{C2}{1,2}
      \item \ShowCompetence{C24}{4,5}
  \end{competences}
  
  \begin{unit}{Introducción}{}{Cormen2009,Steven09,Kulikov09,SkienaRevilla:PC:2003,Laaksonen17,aziz2012elements}{20}{a,b,h}
      \begin{topics}
        \item Introducción a la Programación competitiva
        \item Modelo computacional
        \item Complejidad algoritmica
        \item Problémas sobre búsqueda y ordenamiento
        \item Recursión y recurrencia 
        \item Estrategia divide y conquista
    \end{topics}
    \begin{learningoutcomes}
        \item Reconocer y sabes como usar los recursos del modelo de computación RAM (Random Access Machine). [\Usage]
        \item Determinar el tiempo y espacio de complejidad de algoritmos. [\Usage]
        \item Determinar relaciones de recurrencia para algoritmos recursivos.[\Usage]
        \item Resolver problemas de búsqueda y ordenamiento.[\Usage]
        \item Aprender a seleccionar los algoritmos adecuados para problemas de tipo divide y conquista.[\Usage]
        \item Diseñar nuevos algoritmos para la resolución de problemas.[\Usage]
    \end{learningoutcomes}
  \end{unit}
  
  \begin{unit}{Estructuras de datos}{}{Cormen2009,Steven09,Kulikov09,SkienaRevilla:PC:2003,Laaksonen17,aziz2012elements}{20}{a,b,h}
    \begin{topics}
      \item Problemas sobre arrays y strings
      \item Problemas sobre listas enlazadas
      \item Problemas sobre pilas, colas
      \item Problemas sobre arboles
      \item Problemas sobre Hash tables
      \item Problemas sobre Heaps 
    \end{topics}
    \begin{learningoutcomes}
        \item Reconocer las distintas estructuras de datos sus complejidades usos y restricciones. [\Usage]
        \item Identificar el tipo de estructura de datos adecuado a la resolución del problema. [\Usage]
        \item Reconocer tipos de problemas asociado a operaciones sobre estructuras de datos como búsqueda, inserción, eliminación y actualización.[\Usage]
    \end{learningoutcomes}
  \end{unit}
  
  \begin{unit}{Paradigmas de diseño}{}{Cormen2009,Steven09,Kulikov09,SkienaRevilla:PC:2003,Laaksonen17,aziz2012elements}{20}{a,b,h}
    \begin{topics}
      \item Fuerza bruta
      \item Divide y conquista
      \item Backtracking
      \item Greedy
      \item Programación Dinamica
    \end{topics}
    \begin{learningoutcomes}
        \item Aprender los distintos paradigmas de resolución de problemas.[\Usage]
        \item Aprender a seleccionar los algoritmos adecuados para distintos problemas según el tipo de paradigma.[\Usage]
    \end{learningoutcomes}
  \end{unit}
  
  \begin{unit}{Gráfos}{}{Cormen2009,Steven09,Kulikov09,SkienaRevilla:PC:2003,Laaksonen17,aziz2012elements}{20}{a,b,h}
    \begin{topics}
      \item Recorrido de gráfos 
      \item Aplicaciones y problemas sobre gráfos
      \item Camino mas corto
      \item Redes y flujos 
    \end{topics}
    \begin{learningoutcomes}
        \item Identificar problemas clasificados como problemas de grafos. [\Usage]
        \item Aprender a seleccionar los algoritmos adecuados para problemas de grafos (recorrido, MST, camino mas costo, redes y flujos) y conocer sus soluciones eficientes. [\Usage]
    \end{learningoutcomes}
  \end{unit}
  
  \begin{unit}{Tópicos avanzados}{}{Cormen2009,Steven09,Kulikov09,SkienaRevilla:PC:2003,Laaksonen17,aziz2012elements}{20}{a,b,h}
    \begin{topics}
      \item Teoria de números
      \item Probabilidad y combinaciones
      \item Algoritmos para manejos de strings (tries, string hashing, z-algorithm)
      \item Geometria y sweep line algorithms, segment trees
    \end{topics}
    \begin{learningoutcomes}
        \item Aprender a elegir los algoritmos adecuados para problemas sobre teoria de números y matemáticas ya que son importantes en programación competitiva. [\Usage]
        \item Aprender a seleccionar los algoritmos adecuados para problemas sobre probabilidades y combinaciones,  manejos de strings y geometría computacional. [\Usage]
    \end{learningoutcomes}
  \end{unit}
  
  \begin{unit}{Problemas de dominio especifico}{}{Cormen2009,Steven09,Kulikov09,SkienaRevilla:PC:2003,Laaksonen17,aziz2012elements}{20}{a,b,h}
    \begin{topics}
      \item Latencia y rendimiento
      \item Paralelismo
      \item Redes
      \item Almacenamiento
      \item Alta disponibilidad
      \item Caching
      \item Proxies
      \item Equilibradores de carga
      \item Almacenamiento clave-valo
      \item Replicar y compartir
      \item Elección del líder
      \item Limitación de la tasa
      \item Registro y monitoreo
    \end{topics}
    \begin{learningoutcomes}
        \item Aprender a diseñar sistemas para diferentes problemas de dominio especifico aplicando conocimiento sobre redes, computación distribuida, alta disponibilidad, almacenamiento y arquitectura de sistemas. [\Usage]
    \end{learningoutcomes}
  \end{unit}
  
  \begin{coursebibliography}
  \bibfile{Computing/CS/CS311}
  \end{coursebibliography}
  
  \end{syllabus}
  