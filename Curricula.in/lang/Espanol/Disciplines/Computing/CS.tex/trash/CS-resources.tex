\chapter{Evaluación de Recursos}\label{chap:cs-resources}

Los recursos considerados para la creación de esta nueva carrera están relacionados con la plana docente, infraestructura, laboratorios y financiamiento.

\OnlyUNSA{\input{\currentarea-staff-\siglas}}
\OnlySPC{%
\section{Plana Docente}
En este punto debemos especificar la plana docente con la que se cuenta para esta carrera. 
}

\section{Infraestructura}\label{sec:cs-infraestructura}
Esta nueva carrera podría inicialmente compartir la infraestructura de la Facultad de Producción y Servicios de la UNSA. Sin embargo, en un mediano o largo plazo es necesario poder contar con un espacio propio ya que una de las principales tareas de esta nueva propuesta es la de servir como incubadora de empresas de base tecnológica. 

Desde este punto de vista, el tema de infraestructura no debe representar un problema, al menos en la etapa inicial.
Por otro lado, se cuenta con el apoyo tanto de la EPIS como del DAISI para resolver este punto en aprticular.

\section{Recursos para dictado de clases}
Un profesional innovador debe estar al tanto de los últimos avances de su área siempre. Los últimos avances de esta área no son presentados en los libros necesariamente. Debemos utilizar publicaciones de revistas indexadas de circulación mundial. Por esa razón tomamos como base la suscripción institucional a la ACM y a la IEEE-CS. Es recomendado que el docente use este material para discutir en clase las tendencias en todas las áreas.

\section{Laboratorios}\label{sec:cs-labs}
Al igual que el item anterior, los laboratorios podrían ser inicialmente compartidos con la Facultad de Ingenieríade Producción y Servicios. Para años posteriores es necesario considerar la implementacion de 3 laboratorios de 25 máquinas cada uno. El recurso físico en cuanto a laboratorios para cada curso está detallado en el Capítulo \ref{chap:laboratorios}.

\section{Presupuesto}\label{sec:cs-budget}
\subsection{Justificación Económica}
La propia naturaleza de la carrera de Ciencia de la Computación orientada a la innovación y creación de tecnología abre un gran abanico de posibilidades para generar recursos para la universidad y el país.

Es a partir de este tipo de profesionales que podemos generar tecnología que resulte atractiva para a la inversión extranjera y consecuentemente ayude al desarrollo de la industria de software que aún es precaria en nuestro país.

Sólamente como referencia podemos mencionar que una única empresa de este rubro como es Google tiene ingresos anuales que bordean los 9 mil millones de dólares mientras que el presupuesto nacional de nuestro país esta al borde de los 63 mil millones.

Vale la pena resaltar en este aspecto la colaboración abierta de profesionales de Google hacia este proyecto considerando que es la única forma de generar industria de software de calidad internacional.

\subsection{Consideraciones de orden social}
En relación a las consecuencias sociales de este proyecto debemos resaltar que, generar tecnología de nivel competitivo internacionalmente, provocaría mayores inversiones extranjeras y consecuentemente una mayor cantidad de puestos de trabajo.

La proyección internacional de este perfil internacional también permite aumentar nuestras redes de colaboración con el extranjero. Esto ayuda a que podamos seguir enviando alumnos a estudiar a otros países y también facilita el intercambio y capacitación de docentes. Este tipo de movimiento internacional de nuestro recurso humano hará posible contar con un capital profesional altamente capacitado que permita desarrollar proyectos de gran envergadura a nivel nacional e internacional.

Este perfil profesional permitiría también resolver problemas complejos que nuestra ciudad o región presentan como el modelamiento del tráfico vehicular, detección de desplazamiento de la población para planear rutas de transporte, planeamiento de crecimiento urbano mediante simulaciones, entre otros.

\OnlyUNSA{\input{base-tex/\currentarea-UNSA-Plan-de-funcionamiento}}

