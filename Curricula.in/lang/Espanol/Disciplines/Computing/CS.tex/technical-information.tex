\section{Ficha Técnica de la Encuesta}\label{sec:cs-ficha-tecnica-de-la-encuesta}

La presente encuesta tuvo como asesora a la profesional en estad­stica Auria Julieta Flores Luna, Docente del curso de Estad­stica Experimental de la Universidad Nacional de San Agust­n, Arequipa. 

Nuestra encuesta tuvo un per­odo de ejecución desde Noviembre hasta Diciembre de 2006 de acuerdo a la Tabla \ref{tab:cronograma-de-la-encuesta} (ver Pág. \pageref{tab:cronograma-de-la-encuesta}).

\subsection{Planteamiento del Problema}
Nuestro problema es determinar si: 

\begin{itemize}
\item ?`Es necesaria la creación de la carrera de Ciencia de la Computación en nuestro medio?
\item ?`El perfil profesional de los egresados de las universidades, en las ramas de la computación, es suficiente para generar una industria competitiva de software de calidad internacional?
\item ?`Existe una distinción clara de los perfiles profesionales, de las ramas de la computación, por parte del mercado de trabajo?
\end{itemize}

\subsection{Justificación}
Los mayores responsables de la era tecnológica que nos ha tocado vivir son profesionales que tienen la formación universitaria orientada hacia la generación e innovación de nuevas tecnolog­as.

Para ser competitivo en el mundo actual, la generación de dicha tecnolog­a debe ser de acuerdo a estándares internacionales. Este perfil permite que nuestra tecnolog­a no conozca fronteras y tenga como mercado todo el planeta.

Un perfil orientado a estándares internacionales también ayuda a la movilidad internacional y a la permanente actualización de nuestros alumnos, egresados y plana docente.

La creación de un perfil profesional orientado a la innovación e investigación nos da la oportunidad de pasar de un pa­s netamente consumidor de tecnolog­a extranjera a un pa­s que produzca su propia tecnolog­a y la pueda exportar en igualdad de condiciones de calidad con un producto foráneo.

\subsection{Objetivos de la encuesta}

\textbf{Objetivos Generales}
\begin{enumerate}
\item Determinar si es necesaria la existencia de una carrera de Ciencia de la Computación dado el contexto peruano.
\item Determinar requerimientos no cubiertos de profesionales en el mercado laboral local/nacional para la creación de una industria competitiva de software de calidad internacional.
\end{enumerate}

\textbf{Objetivos Espec­ficos}
\begin{enumerate}
 \item Determinar el grado de conocimiento de los empresarios con relación a las diferentes ramas de la computación.
\end{enumerate}

\subsection{Hipótesis}
Este estudio tiene las siguientes hipótesis:

\begin{itemize}
 \item Es probable que el mercado laboral de nuestro medio necesite profesionales con formación en Ciencia de la Computación.
\item Es probable que la formación universitaria existente sea insuficiente para generar una industria de software competitiva a nivel internacional.
\item Es probable que el mercado laboral no esté en condiciones de distinguir los perfiles profesionales del área de computación.
\item Es probable que las empresas del medio necesiten, en una gran mayor­a, profesionales con formación en investigación e innovación permanente.
\item Es probable que las empresas del medio no sean competitivas tecnológicamente.
\item Es probable que el mercado laboral considere mayoritariamente que la responsabilidad de la correcta difusión de los perfiles profesionales es de las universidades que los forman.
\item Es probable que la innovación permanente sea muy importante en las empresas del medio.
\item Es probable que las empresas tengan gran necesidad de profesionales innovadores.
\item Es probable que las empresas del medio tengan mayores ventajas competitivas debido a la formación de innovadores y creadores de tecnolog­a computacional por parte de los profesionales del medio.
\item Es probable que se pueda contar con financiamiento en nuestro pa­s para desarrollar tecnolog­a con calidad de exportación.
\end{itemize}


\subsection{Presupuesto y Financiamiento}
El costo de la ejecución de esta encuesta fue reducido debido al uso del internet como herramienta de env­o y recolección de los datos. 

\subsection{Recursos}
Los recursos humanos requeridos fueron obtenidos de manera voluntaria por iniciativa de profesionales de diversas instituciones interesados en ayudar a esta iniciativa. Toda la ejecución de la encuesta fue coordinada por el Dr. Ernesto Cuadros-Vargas y el Ing. Johan Chicana D­az.

Para el procesamiento de datos se utilizó apenas un microcomputador con conexión a internet y un teléfono para las coordinaciones de env­o y recepción de las encuestas.

\subsection{Cronograma}
Esta encuesta se ha planeado de acuerdo al cronograma presentado en el Cuadro \ref{tab:cronograma-de-la-encuesta}.

\begin{center}
\begin{table}[h!]
\begin{tabularx}{\textwidth}{|X|c|c|c|c|}\hline
\textbf{Acciones}       & \textbf{Nov 1-15} & \textbf{Nov 16-30} & D\textbf{ic 1-15} & \textbf{Dic 16-29} \\ \hline
Elaboración del Plan    &  $\surd$ &           &          &           \\ \hline
Elaboración de la ficha &  $\surd$ &           &          &           \\ \hline
Recolección de datos    &          & $\surd$   &  $\surd$ &           \\ \hline
Procesamiento de datos  &          &           &  $\surd$ &  $\surd$  \\ \hline
Análisis e interpretación&          &           &          &  $\surd$  \\ \hline
Redacción del informe   &          &           &          &  $\surd$  \\ \hline
\end{tabularx}
\caption{Cronograma de ejecución de la encuesta de mercado sobre Ciencia de la Computación a empresarios a nivel Perú.}
\label{tab:cronograma-de-la-encuesta}
\end{table}
\end{center}
