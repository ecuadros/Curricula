La referencia más sólida a nivel mundial en cuanto a la propuesta de carreras de
computación para nivel de pregrado es la que fue propuesta en conjunto por la
\ac{ACM}, \ac{IEEE-CS} y la \ac{AIS}. Estas tres organizaciones propusieron la
Computing Curricula en el documento denominado: {\it Joint Task Force for Computing Curricula 2005,
Computing Curricula 2005. Overview Report}\cite{ComputingCurricula2005}.

La \acl{CC} es un término de origen estadounidense \ac{CS}. Este término es
conocido también como informática en el ámbito europeo\footnote{El término
europeo es derivado del vocablo francés {\it Informatique}.}.

Según el diccionario de la Real Academia de la Lengua Española (http://www.rae.es)
ambos términos también son sinónimos.

A nivel internacional, la computación presenta 5 perfiles claramente definidos:
\begin{itemize}
\item Ciencia de la Computación (\textit{Computer Science}) \\ \cite{CS2013},
\item Ingenierí­a de Computación (\textit{Computer Engineering}) \cite{ComputerEngineering2004},
\item Sistemas de Información (\textit{Information Systems}) \cite{InformationSystemsCurricula2010, InformationSystems2002Journal},
\item Ingenierí­a de Software (\textit{Software Engineering}) \cite{SoftwareEngineering2004},
\item Tecnologí­a de la Información (\textit{Information Technology}) \cite{InformationTechnology2005}
\end{itemize}

La Figura \ref{fig.cs} es tomada de la definición propuesta en la \textit{Computing Curricula}
\cite{ComputingCurricula2005} en el área de \ac{CC} cuyo última versión es la denominada CS2013 (cs2013.org) \cite{CS2013}.
La \ac{CC} cubre la mayor parte entre el extremo superior y el extremo inferior, porque el
profesional en \ac{CC} no trata ``solamente con el hardware'' que utiliza un software o de
``solamente la organización'' que hace uso de la información que la computación le puede proveer.

\begin{figure}[ht]
   \centering
   \includegraphics[width=13cm]{\OutputFigsDir/\currentarea}
   \caption{Campo acción de la Ciencia de la Computación}
   \label{fig.cs}
\end{figure}

%\begin{quote}
Ciencia de la Computación cubren un amplio rango, desde sus fundamentos teóricos y algorí­tmicos hasta 
los \'ultimos desarrollos en robótica, visión por computadora, sistemas inteligentes, bioinformática, y
otras áreas emocionantes. Podemos pensar que el trabajo de un cientí­fico de la computación pertenece
a las siguientes tres categorí­as:

\begin{itemize}
\item \textbf{Diseño e implementación de software}. Los cientí­ficos de computación se encargan de
desafiantes labores de programación. También supervisan otros programadores, haciéndolos concientes
de nuevas aproximaciones.

\item \textbf{Instrumentación de nuevas formas para usar computadoras}. El progreso en las áreas
de ciencias de la computación como redes, bases de datos, e interfaces humano-computadora permitieron
el desarrollo de la www y actualmente se trabaja en el desarrollo de metasistemas Grid. Además,
los investigadores trabajan ahora en hacer que los robots sean ayudantes prácticos y demuestren
inteligencia, utilizan las bases de datos para crear nuevos conocimientos, y están utilizando
computadoras para decifrar los secretos de nuestro ADN.

\item \textbf{Desarrollo de formas efectivas de resolver problemas de computación.}
Por ejemplo, los cientí­ficos de la computación desarrollan las mejores formas posibles
de almacenar información en bases de datos, enviar datos a través de la red, y
desplegar imágenes complejas. Sus bases teóricas les permiten determinar el
mejor desempeño posible, y su estudio de algoritmos les ayuda a desarrollar
nuevas aproximaciones para proveer un mejor desempeño.
\end{itemize}

La Ciencia de la Computación cubren todo el rango desde la teorí­a hasta la programación. Mientras otras disciplinas pueden producir titulados mejor preparados para trabajos especí­ficos, las ciencias de la computación ofrecen un amplio fundamento que permite a sus titulados adaptarse a nuevas tecnologí­as y nuevas ideas.
%\end{quote}

El profesional en \ac{CC} se preocupa por casi todo en medio de estas áreas. En dirección hacia el hardware, este profesional llega a desarrollar software que permite el funcionamiento de dispositivos {\it devices}. En dirección a aspectos organizacionales, el profesional de \ac{CC} ayuda a que los sistemas de información operen correctamente en las organizaciones. Él genera la tecnologí­a que permite que otras áreas como los sistemas de información se desarrollen adecuadamente.

El profesional en \ac{CC} diseña y desarrolla todo tipo de software, desde infraestructura de plataformas (sistemas operativos, programas de comunicación, etc.) hasta aplicación de tecnologí­as (navegadores de Internet, bases de datos, motores de búsqueda, etc.). Este profesional crea estas capacidades, pero no está orientado al uso de las mismas. Por lo tanto, el área sombreada (fig. \ref{fig.cs}) para \ac{CC} se estrecha y finaliza en la medida que nos movamos hacia la aplicación y configuración de productos.
