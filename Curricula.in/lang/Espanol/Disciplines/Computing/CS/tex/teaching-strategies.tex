\section{Estrategias de Enseñanza-Aprendizaje}
El avance progresivo de las tecnologí­as y redes móviles, habilita nuevas herramientas para ser exploradas y 
experimentadas en escenarios de aprendizaje. Es así­, como el uso de dispositivos móviles, añade nuevas 
dimensiones al proceso de enseñanza aprendizaje, como la movilidad y personalización. La evolución de los 
componentes de E-Learning (aprendizaje soportado por medios electrónicos) para el aprendizaje móvil y aprendizaje ubicuo, 
abre el espacio conceptual y técnico para el desarrollo de la Internet de los objetos (IoT), de sus siglas en inglés de {\it Internet of Things}, en el aprendizaje.

La tendencia generalizada de parte de los estudiantes a aceptar positivamente las herramientas que 
impliquen una novedad en el proceso de enseñanza-aprendizaje, nos indica que la inclusión de elementos nuevos, es favorable para el ánimo frente al aprendizaje.

Teniendo en cuenta que las actividades intensivas en cuanto al uso de dispositivos especializados, 
deben ser cuidadosamente diseñadas, sobre todo en cuanto a imprudencias o excesos de parte de los usuarios, 
lo cual puede llevar a fallas y posibles pérdidas de información, entonces podemos hacer uso de IoT 
para hacer más agradable el proceso de aprendizaje.

Por lo que, consideramos implementar los siguientes conceptos estratégicos:

\begin{itemize}
\item Horizontalidad educativa.
\item Material audiovisual, aprendamos con documentales y con todo lo que nos da hoy la tecnologí­a.
\item interactividad grupal.
\item Motivación de la curiosidad.
\item Fomento de la creatividad.
\item Pensamiento lateral.
\item Uso adecuado de las tecnologí­as.
\item Autoaprendizaje guiado.
\item Estimulación de la inteligencia en lugar de estimular la memoria.
\item Diversión incorporada, mientras aprendemos ó enseñamos.
\end{itemize}