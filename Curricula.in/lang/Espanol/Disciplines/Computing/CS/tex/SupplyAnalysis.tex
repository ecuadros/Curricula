\section{Análisis de Oferta}\label{sec:cs-analisis-de-oferta}
El análisis de oferta de profesionales en esta línea puede ser claramente visto desde dos perspectivas: nacional e internacional.

A nivel nacional existen sólo dos universidades que ofrecen el perfil de Ciencia de la Computación, la primera es la Universidad Nacional de Trujillo que otorga el grado de Bachiller en Ciencia de la Computación desde hace varios años. La segunda y más reciente es la Universidad Católica San Pablo en Arequipa que desde 2006 ha comenzado a ofrecer este perfil debido al proceso de acreditación que se ha iniciado.

A nivel internacional, Bolivia tiene una fuerte tendencia hacia Ciencia de la Computación. Esta tendencia es reforzada por la reunion anual que tienen en el mayor congreso Boliviano de Computación denominado: ``Congreso Boliviano de Ciencia de la Computación"~que en 2006 se realizó en su 13$^{ava}$ edición.

En Brasil, este proceso de estandarización internacional comenzó aproximadamente hace 20 años y en la actualidad son muy pocas las carreras que escapan a la estandarización. Esta información puede ser ampliada a través de la \ac{SBC} \footnote{http://www.sbc.org.br} y en el Ministerio de Ciencia y Tecnología de Brasil \footnote{http://www.mec.gov.br}. Esta estandarización internacional de parte de Brasil conjugada con una mejora de las universidades han provocado que grandes empresas se instalen en ese país como es el caso de Google que recientemente ha abierto Google Brasil en Belo Horizonte-MG. Esta inversión se realizó debido a la gran cantidad de producción científica del grupo de Bases de Datos de la Universidad federal de Minas Gerais-UFMG\footnote{http://www.ufmg.br}.

En Chile, el caso está más claro y está liderado por la \ac{SCCC} hace más de 20 años. Este cambio les ha permitido como país ser un líder junto con Brasil y Uruguay en la exportación de Software latinoamericano. Nuevamente, la alta producción intelectual en el área de Ciencia de la Computación de la Universidad de Chile fue la responsable de que la transnacional Yahoo! abra su rama de Yahoo! en Santiago de Chile desde 2005. En Chile es difícil encontrar carreras de computación que no estén orientadas al estandar de Ciencia de la Computación.

Uruguay tiene una única universidad nacional y es un país de 3 millones de habitantes. Sin embargo, exporta 8 veces más software que nuestro país. La razón fundamental es que la universidad está orientada a generar conocimiento de forma permanente y sus egresados están orientados a la exportación de software de calidad. La carrera es denominada Informática, sin embargo la formación doctoral de más de 15 docentes en Europa y USA en Ciencia de la Computación es la mayor responsable del éxito en sus exportaciones de software.

El contexto estadounidense está fuertemente regido por profesionales de Ciencia de la Computación y también está tomando importancia el área de Ingeniería de Software. Este último, es un esfuerzo por crear metodologías en el proceso de producción de software. El perfil de Ciencia de la Computación es el principal responsable de la existencia de empresas líderes como Google fundada por Sergey Brin y Larry Page quienes son egresados de la maestria de Ciencia de la Computación de la Universidad de Stanford. Inclusive en la oferta laboral para ingenieros de software puede ser observado que el profesional que se pide debe tener formación en Ciencia de la Computación\footnote{http://www.google.com/support/jobs/bin/topic.py?dep\_id=1056\&loc\_id=1100}.

En el contexto europeo, la carrera de Ciencia de la Computación es denominada Informática y también tiende a estandarizarse hace varios años debido al Acuerdo de Bologna. Este acuerdo busca facilitar de movilidad de egresados entre todos los países de la Unión Europea. Esta exigencia obliga a todos los países a uniformizar sus carreras y por esta razón la estandarización internacional es algo vital para seguir generando profesionales competitivos.

Basado en este contexto nacional e internacional de profesionales de esta línea consideramos que se debe recibir un número de 80 ingresantes por año.
