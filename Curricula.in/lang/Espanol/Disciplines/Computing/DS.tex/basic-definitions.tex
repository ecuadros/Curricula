\section{Definiciones básicas}\label{sec:ds-definiciones-basicas}

La Ciencia de Datos comprende la ciencia de planificación, adquisición, gestión, análisis e inferencia a partir de datos. Los fundamentos teóricos de Ciencia de Datos se basan principalmente en Estadí­stica, Ciencia de la Computación y Matemáticas. Los significados prácticos del mundo real provienen de la interpretación de los datos en el contexto del dominio en el que surgieron los datos.

El ciclo recursivo de datos para obtener, discutir, curar, administrar y procesar datos, explorar datos, resolver preguntas, realizar análisis y comunicar los resultados es el núcleo de Ciencia de Datos. Los estudiantes de pregrado deben comprender y practicar en todos los pasos de este ciclo y de esta forma participar en preguntas de investigación sustantivas.

La Ciencia de Datos es necesariamente \emph{altamente experiencial}, cuyo experiencia es adquirida a través de la práctica. Los estudiantes de Ciencia de Datos deben encontrar aplicaciones frecuentes del mundo real basadas en proyectos con datos reales para complementar los algoritmos y modelos fundamentales.

La Ciencia de Datos ofrece la oportunidad de integrar y utilizar el \textbf{pensamiento computacional} y \textbf{estadí­stico} para resolver problemas en lugar de enfatizar uno sobre el otro.