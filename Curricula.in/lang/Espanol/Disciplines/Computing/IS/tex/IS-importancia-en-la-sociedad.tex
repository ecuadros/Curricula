\section{Importancia de la carrera en la sociedad}\label{sec:cs-importancia-en-la-sociedad}

Los Sistemas de Información basados en el computador se han tornado una parte escencial de los productos, servicios, operaciones y administración de las organizaciones. El uso efectivo y eficiente de las tecnologí­as de información y de comunicación es un elemento importante en el logro de ventajas competitivas para las organizaciones de negocios y la excelencia en el servicio para organizaciones gubernamentales y sin fines de lucro. 

La estrategia de los sistemas de información y la tecnologí­a de información es una parte integral de la estrategia organizacional. Los sistemas de información soportan los procesos de administración en todos los niveles: operacional, táctico y estratégico. Así­ mismo, son vitales para la identificación y análisis de problemas así­ como para la toma de decisiones. La importancia de la tecnologí­a de información y de los sistemas de información para las organizaciones y la necesidad de profesionales competentes en el campo es la base para un enlace fuerte entre los programas educacionales y la comunidad profesional de Sistemas de Información.

El camino lógico que se espera que siga un profesional de esta área es que el se dedique a 
producir software, que se integre a las empresas productoras de software o al área de desarrollo de sistemas
dentro de cualquier organización. En Perú, la entidad que agrupa a las empresas dedicadas a la producción de software es la \ac{APESOFT}. Esta asociación ha tomado como polí­tica principal dedicarse a la producción de software para exportación. Siendo así­, no tendrí­a sentido preparar a nuestros alumnos sólo para el mercado local o nacional. Nuestros egresados deben estar preparados para desenvolverse en el mundo globalizado que nos ha tocado vivir.
