\section{Campo y mercado ocupacional}\label{sec:job-positions}
Nuestro egresado podrá prestar sus servicios profesionales en empresas e instituciones públicas y privadas 
que requieran sus capacidades en función del desarrollo que oferta, entre ellas:

\begin{itemize}
\item Empresas dedicadas a la producción de software con calidad internacional orientado a asuntos organizacionales.
\item Empresas, instituciones y organizaciones que requieran software de calidad para mejorar sus actividades y/o 
      servicios ofertados.
\end{itemize}

Nuestro profesional puede desempeñarse en el mercado laboral sin ningún problema ya que, en general, la 
exigencia del mercado y campo ocupacional está mucho más orientada al uso de herramientas. 
Nuestro profesional se diferencia debido a que el considera la tecnologí­a como una herramienta fundamental
para el eficiente funcionamiento de una organización en un ambiente globalizado como el que tenemos en estos momentos.

El profesional en Sistemas de Información cuenta con rasgos de formación en Ciencia de la Computación por 
lo que está en condiciones de complementarse con dicha profesión en beneficio de la organización donde se desempeña.

A medida que la informatización de las empresas del paí­s avanza, la necesidad de personas 
capacitadas para resolver los problemas de mayor complejidad aumenta. Las organizaciones, sin ninguna duda, 
también son fuertemente impactadas por un aumento de información que ya no es posible tratar de forma manual.
En este escenario, el profesional en Sistemas de Información se perfila como alguien fundamental en la toma 
de decisiones basada en información que el mismo es capaz de producir en la organización. 
Esta caracterí­stica representa una enorme ventaja en relación a profesionales de otras áreas como 
Administración de Empresas o Ingenierí­a Industrial quienes, en muchos casos, toman decisiones empresariales 
con información procesada en baja escala o en forma manual debido a que no carecen del componente 
tecnológico en la misma profundidad que este profesional si tiene a su disposición. 

Debido al correcto entendimiento de la computación como una herramienta que puede ser utilizada para 
generar y procesar gran volumen de información, el profesional de Sistemas de Información presenta 
la ventaja de poder tomar decisiones apoyadas en información concreta y a gran escala que el mismo es 
capaz de producir y que otras profesiones no relacionadas a la computación no podrí­an hacer con 
la misma velocidad y eficiencia.
